%----------------------------------------------------------------------------------------
%Paragraph Coherence
%----------------------------------------------------------------------------------------

\chapter{Paragraph Coherence}

\section{What is paragraph coherence?}

A paragraph is a unit of text devoted to developing an idea. The idea is usually stated in a topic sentence.

A paragraph is coherent when every sentence is devoted to developing the main point of the paragraph, the sentences are in a logical order, the connections between sentences are clear, unnecessary repetition is avoided, and the reader can easily follow the development of your point from your first sentence to your last.

It is the opposite of incoherence, which occurs when a paragraph is disorganized and repetitive and makes little sense to the reader.

\section{How can I make my paragraphs coherent?}

\begin{enumerate}

\item Identify the main idea of the paragraph then remove any sentence that does not develop that main idea or repeats ideas offered in another sentence.

\item Consider the order of your sentences. Is there a reason why they are in the order they are, or do they need to be rearranged to make sense to a reader?

\item Think about employing some transitions (see \emph{Transitions}) at the beginning of some of the sentences. These will help you pinpoint the relationships between your ideas/sentences and thus clarify these relationships for your reader.

\item Repeat key words.

\item Be concise (see \emph {Concision}) and precise (see \emph{Precision}).
\end{enumerate}
 
 
\textbf{For example}:

\begin{quote}
\textbf{An incoherent paragraph}: 

Ethos is the ethical appeal. This author talks about a lot of different arguments for legalizing organ sale. In rhetoric, ethos is very important. You shouldn't be able to sell your organs. In China there have been some human rights investigations into the executions of political prisoners and the way their organs were sold. The author thinks you should be able to sell your organs, but I think you shouldn't be able to.  I don't know why the author has to be so hostile when he writes, calling his opponents ``small-minded.''  Just because he thinks you should be able to sell your organs, he is not being ethical.  And why is he so sarcastic in passage 5?  A lot of people don't agree with selling organs; they think we should just donate them. He is not persuading me because of his ethos.
\end{quote}

\begin{quote}
\textbf{Revision}:
          
When discussing the complicated issue of  selling organs, this author fails to persuade his audience that such sales should be legal because he fails to construct a strong ethos.  Ethos is the impression a writer creates of himself through his written text. To be persuasive, the writer must employ the ethical appeal in such a way that he appears educated and well-intentioned. As well, he should demonstrate that he shares the audience's values. Consequently, when this author is sarcastic (as he is in passage 5), or hostile (when he belittles his opponents by calling them small-minded),  he alienates those he most wishes to persuade, those who disagree with him, like me. While I might have been persuaded if he had offered more data (for instance on the supposed black market in China) and dealt gracefully with his opponents, in this instance, I was not.

\end{quote}

%----------------------------------------------------------------------------------------
% END OF SECTION
%----------------------------------------------------------------------------------------

