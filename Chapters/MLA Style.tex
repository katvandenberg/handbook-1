
%----------------------------------------------------------------------------------------
%	MLA
%----------------------------------------------------------------------------------------
\chapter{MLA style} % Sub-section


\section{Formatting the MLA essay}
When setting up your word processor for an MLA-formatted document, use the 
following settings:

\begin{itemize}
\item Set 1" margins on all sides of the document.
\item Double-space the entire document, including block quotes.
\item In the top left portion of the first page, type your name, instructor's name, 
course title, and date on separate, double-spaced lines.
\item Include your last name and a page number on each page in the top right corner 
of the header.
\item Include a centered title on the first page.
\item Indent the first line of each paragraph with a tab set to 0.5".
\end{itemize}

When a quotation runs more than four typed lines, use a \textbf{block quote}. A block 
quote is a free-standing block of text, set apart from the rest of the text. (See the 
following page for a visual example of block quote formatting in MLA). When formatting 
blockquotes in MLA, use the following rules: 

\begin{itemize}
\item Begin the block quote on a new line. 
\item Indent every line of the quote 1" from the left margin (two tabs). 
\item Do not use quotation marks around the quoted material. 
\item Place the parenthetical citation \emph{after} the final punctuation of the quoted 
passage. For example:
\end{itemize}

The following page is a model for the MLA-formatted document:

%-----------MLA FIRST PAGE EXAMPLE--------
\newpage

\thispagestyle{empty}
\begin{flushright}Johnson 1\end{flushright}
\bigskip
\begin{Spacing}{2}
Jeff Johnson\\
Profesor Smith\\
English 130\\
January 28, 2010
\end{Spacing}
\begin{center}
America's Foreign Policy Future?
\end{center}
\begin{Spacing}{2}

\hspace{.4in}While all of the world may indeed be a stage, we must be decisive in the 
measures we take in order to prevent the curtain from closing too soon. We currently 
live in an era of great political uncertainty. After the Cold War, as Dave Smith has argued, 
America suddenly became the stand-alone world power, dominating international 
treaties, coalitions, and movements (34). This "unipolarity" has resulted in one of the 
most wholly economically prosperous and peaceful periods of time in history. As one 
economist explains,

\hspace{.8in}The most remarkable result of the fall of the Soviet Union has been the

\hspace{.8in}incredible proliferation of capital markets to every corner of the globe.

\hspace{.8in}From Bangalore to Brazil, Columbia to China, capital flows have  

\hspace{.8in}circled the globe, all protected by the military forces of the world's last 

\hspace{.8in}remaining superpower. (Johnson 77)

However, this economic prosperity comes at a great cost. American military 
deployments at bases across the world cost the American taxpayers dearly. As much as


\end{Spacing}
\newpage
%------DONE------------------

\section{In-text citations}
The MLA style uses parenthetical citations to indicate the author and page number of 
sources. These parenthetical citations take two forms. The first form is used when the 
source you are citing \emph{is known or understood} by your audience. The second 
form is used when the author being cited is \emph{unknown or unclear}. 

In the following sentence, the author of the source in question is obvious:

\begin{quote}According to scholar James Frey, "Americans eat five pounds of ice cream 
annually" (78).
\end{quote}

Since the author, James Frey, is understood, the citation uses \textbf{only the page 
number} of the source.

In these versions of the sentence, however, the author is not stated:

\begin{quote}
According to one scholar, "Americans eat five pounds of ice cream annually" (Frey 78).
\end{quote}
Or:

\begin{quote}
Studies have shown that the American people consume an average of five pounds of ice 
cream every year (Frey 78).
\end{quote}
In the first sentence, the author is referred to only as a generic "scholar." To give the 
reader information on which scholar is being cited, Frey's name is included in the 
citation. In the second sentence, the author describes the report, not its author; as a 
result, the student has included Frey's name to indicate whose report is being referenced. 

%----DONE-----------

\section{Bibliography}

MLA requires a bibliography at the conclusion of the essay that includes the full 
citation of the sources cited within the essay. In MLA, the bibliography is known as the 
Works Cited page. When setting up a Works Cited page, use the following rules and 
characteristics:

\begin{itemize}
\item Center the words "Works Cited" at the top of the page.
\item Use your last name and the page number on the right side of the page's header.
\item Double-space the Works Cited entries.
\item Alphabetize the entries by the author's last name.
\item If an entry runs more than one typed line, indent the second (and any 
subsequent) line with a .5" tab.
\item If two or more works by the same author are used, list the entries alphabetically 
by title. After the first entry, replace the author's name with three dashes followed by a 
period. (See the entries from St. Augustine on the following page).
\end{itemize}


Here is an example of a MLA formatted Works Cited page:

%-----------MLA Works Cited EXAMPLE--------
\newpage
\thispagestyle{empty}
\begin{flushright}Johnson 23\end{flushright}
\begin{center}
Works Cited
\end{center}
\begin{Spacing}{2}

Adams, Robert M. \emph{James Joyce: Common Sense and Beyond}. Woodward: 

\hspace{.4in}Classic Nonfiction Library, 1967. Print.

Augustine. \emph{Tractates on the Gospel of John}. Trans. J. W. Rettig. Washington D.C.:

\hspace{.4in}Catholic U of America P, 1988. Print.

- - -. \emph{Confessions}. Trans. Henry Chadwick. New York: Oxford UP, 1992. Print.

Carver, Craig. "Molly: Bloom's Preservative; Correspondence and Function in \emph{Ulysses}."

\hspace{.4in}\emph{James Joyce Quarterly} 12 (1975): 414-22. Print.

Garrett, Peter K. Introduction. \emph{Twentieth Century Interpretations of} Dubliners.

\hspace{.4in}Ed. Peter K. Garrett. New York: Penguin, 1968. 1-17. Print.

\end{Spacing}

\newpage




%------DONE------------------
										                

Although there are hundreds of MLA forms for citing everything from books to email 
conversations, they may all be reduced to two basic forms. Let's call them "book'" and 
"periodical." All of the other forms are merely variations of these two.

\section{Book form}



Last Name, First Name. \emph{Title}. City: Publisher, Year. Medium of 

\hspace{.4in}Publication (Web or Print).
\medskip

Taylor, Alan. \emph{William Cooper's Town: Power and Persuasion on the Frontier of the 
Early} 

\hspace{.4in}\emph{American Republic}. Penguin Books, 2001. Print.
\medskip

Variations on this form include a \textbf{book by multiple authors}, a \textbf {book with 
an editor}, a \textbf {translated book}, an \textbf {edition}, and so on. Each of these 
additional forms will require slight alterations or additions to the basic form. 

\subsection{Commonly used book forms}

\textbf{A book with an editor}: 
\begin{quote}
James, Henry. \emph{Portrait of a Lady}. Ed. Leon Edel. Boston: Houghton,

\hspace{.4in}1963. Print.
\end{quote}

\textbf{A book with a translator}: 
\begin{quote}
McDougle, Astrid. \emph{The Basics of Gaelic}. Trans. Paddy Maloney. New

\hspace{.4in}York: Vintage, 1990. Print.
\end{quote}


\textbf{An edition (other than the first)}:
\begin{quote}
Thompson, Fred. \emph{Why I Fight}. 3rd ed. New York: Vanity Publications,

\hspace{.4in}2000. Print.
\end{quote}

\textbf{An anthology}: 
\begin{quote}
Savage, Fred, ed. \emph{American Literature about the Great Depression}. 12th

\hspace{.4in}ed. New York: Longman, 2012. Print.
\end{quote}

\textbf{A work in an anthology}: 
\begin{quote}
Smith, Noel. "The Dustbowl Aesthetic." \emph{American Literature about the}

\hspace{.4in}\emph{Great Depression}. Ed. Fred Savage. 12th ed. New York:

\hspace{.4in}Longman,  2012. Print.
\end{quote}

\textbf{A republished book}: 
\begin{quote}
James, Esther. \emph{My Life}. 1946. New York: Random House, 2001. Print.
\end{quote}


\section{Periodical form}
\begin{quote}
\textbf{Last Name, First Name. "Title." Journal Name Volume.Issue  (Year):}

\hspace{.4in}\textbf{Page Number range. Medium of Publication.}
\end{quote}

\begin{quote}
Luck, Chad. "Re-Walking the Purchase: \emph{Edgar Huntly}, David Hume, and

\hspace{.4in}the Origins of Ownership." \emph{Early American Literature} 44.2 (2009):  

\hspace{.4in}271-306. Print.
\end{quote}

Common variations on the basic periodical form include a \textbf{newspaper article}, 
an \textbf{essay in a scholarly journal}, an \textbf{essay in a collection or edited volume}, 
an \textbf{article in a magazine}, and an \textbf{article on a website}.

\subsection{Commonly used periodical forms}

\textbf{An article in a scholarly journal with volume and issue numbers}:
\begin{quote}
Taylor, James. "The Indian Matter of Charles Brockden Brown's

\hspace{.4in}\emph{Edgar Huntly}." \emph{American Literature} 45.6 (1998): 432-45. Print.
\end{quote}


\textbf{An article in a scholarly journal with only volume numbers}:
\begin{quote}
Johnston, Johanna. "A Reading of \emph{Moby Dick}." \emph{North Dakota Quarterly}  

\hspace{.4in}45 (1978): 45-56. Print.
\end{quote}

\textbf{An article in a newspaper}:
\begin{quote}
McKinley, Robert. "Cat Saved from Dog." \emph{New York Times} 7 Oct. 2011:

\hspace{.4in}A12+. Print.
\end{quote}

\textbf{An article in a magazine}:
\begin{quote}
Smith, Jim. "Remembering Tony." \emph{New Yorker} Jan. 2010: 12-18. Print.
\end{quote}

\textbf{An article in an online database}:
\begin{quote}
Taylor, Hayden. "\emph{Moby Dick} and the Cold War." \emph{American Literature} 45.6 

\hspace{.4in}(2010): 45-57. JSTOR. Web. 12 July 2012.
\end{quote}

%----------------------------------------------------------------------------------------
% END OF SECTION
%----------------------------------------------------------------------------------------