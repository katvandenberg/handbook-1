
%----------------------------------------------------------------------------------------
%	Plagiarism
%----------------------------------------------------------------------------------------

\chapter{Plagiarism}

\section{Definition of Plagiarism}

Boston University's \href{http://www.bu.edu/academics/resources/academic-conduct-code/}{\{Academic Conduct Code\}} adopts a definition of plagiarism from Harold Martin and Richard Ohmann's \emph{The Logic and Rhetoric of Exposition} (1963). They define plagiarism as follows:

\begin{quote}The academic counterpart of the bank embezzler and of the manufacturer who mislabels products is the plagiarist, the student or scholar who leads readers to believe that what they are reading is the original work of the writer when it is not. If it could be assumed that the distinction between plagiarism and honest use of sources is perfectly clear in everyone’s mind, there would be no need for the explanation that follows; merely the warning with which this definition concludes would be enough. But it is apparent that sometimes people of goodwill draw the suspension of guilt upon themselves (and, indeed, are guilty) simply because they are not aware of the illegitimacy of certain kinds of “borrowing” and of the procedures for correct identification of materials other than those gained through independent research and reflection...

The spectrum is a wide one. At one end there is a word-for-word copying of another’s writing without enclosing the copied passage in quotation marks and identifying it in a footnote, both of which are necessary. (This includes, of course, the copying of all or any part of another student’s paper.) It hardly seems possible that anyone of college age or more could do that without clear intent to deceive. At the other end there is the almost casual slipping in of a particularly apt term which one has come across in reading and which so admirably expresses one’s opinion that one is tempted to make it personal property. Between these poles there are degrees and degrees, but they may be roughly placed in two groups. Close to outright and blatant deceit—but more the result, perhaps, of laziness than of bad intent—is the patching together of random jottings made in the course of reading, generally without careful identification of their source, and then woven into the text, so that the result is a mosaic of other people’s ideas and words, the writer’s sole contribution being the cement to hold the pieces together. Indicative of more effort and, for that reason, somewhat closer to honest, though still dishonest, is the paraphrase, an abbreviated (and often skillfully prepared) restatement of someone else’s analysis or conclusion, without acknowledgment that another person’s text has been the basis of the recapitulation.\end{quote}

Although there is a wide "spectrum" of plagiarism, Martin and Ohmann argue that all examples  of plagiarism may be placed within three broad categories: word-for-word copying, the adoption of certain "apt" phrases or words, 
 %---------------------------------------------------------------------------------------
% END OF SECTION
%----------------------------------------------------------------------------------------
