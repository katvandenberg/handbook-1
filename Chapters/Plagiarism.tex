
%----------------------------------------------------------------------------------------
%	Plagiarism
%----------------------------------------------------------------------------------------

\chapter{Plagiarism}
\textbf{This chapter is taken directly from Boston University's \href{http://www.bu.edu/academics/resources/academic-conduct-code/}{\{Academic Conduct Code\}}}

\section{Examples of Plagiarism}

[From H. Martin and R. Ohmann, \emph{The Logic and Rhetoric of Exposition}, revised edition, Holt, Rinehart and Winston, 1963.]

The examples given below should distinguish between dishonest and the proper use of source material. If instances occur which these examples do not seem to serve as a model, conscience will, in all likelihood, be prepared to supply advice.

\subsection{The Source}

\begin{quote}“The importance of the Second Treatise of Government printed in this volume is such that without it we would miss some of the familiar features of our own government. It is safe to assert that the much criticized branch known as the Supreme Court obtained its being as a result of Locke’s insistence upon the separation of power; and that the combination of many powers in the hands of the executive under the New Deal has still to encounter opposition because it is contrary to the principles enunciated therein, the effect of which is not spent, though the relationship may not be consciously traced. Again we see the crystallizing force of Locke’s writing. It renders explicit and adapts to the British politics of this day the trend and aim of writers from Languet and Bodin through Hooker and Grotius, to say nothing of the distant ancients, Aristotle and the Stoic School of natural law. It sums up magisterially the arguments used through the ages to attack authority vested in a single individual, but it does so from the particular point of view engendered by the Revolution of 1688 and is in harmony with the British scene and mental climate of the growing bourgeoisie of that age. Montesquieu and Rousseau, the framers of our own Declaration of Independence, and the statesmen (or should we say merchants and speculators?) who drew up the Constitution have re-echoed its claims for human liberty, for the separation of powers, for the sanctity of private property. In the hands of these it has been the quarry of liberal doctrines; and that is has served the Socialist theory of property based on labor is final proof of its breadth of view.”

--Charles L. Sherman, “Introduction” to \emph{Treatise of Civil Government and A Letter Concerning Toleration by John Locke}

\end{quote}

\subsubsection{Word-for-Word Plagiarizing}

\begin{quote}“It is not hard to see the importance of the Second Treatise of Government to our own democracy. Without it we would miss some of the most familiar features of our own government. It is safe to assert that the much criticized branch known as the Supreme Court obtained its being as a result of Locke’s insistence upon the separation of powers; and that the combination of many powers in the hands of the executive is contrary to the principles enunciated therein; the effect of which is not spent, though the relationship may not be consciously traced. The framers of our own Declaration of Independence and the statesman who drew up the Constitution have re-echoed its claims for human liberty, for the separation of powers, for the sanctity of private property. All these are marks of influence of Locke’s Second Treatise on our own way of life.”
\end{quote}

In this example, after composing half of the first sentence, the writer copies exactly what is in the original text, leaving out the center section of the paragraph and omitting the names of Montesquieu and Rousseau where he takes up the text again. The last sentence is also the writer’s own.

If the writer had enclosed all the copied text in quotation marks and had identified the source in a footnote, he would not have been liable to the charge of plagiarism; a reader might justifiably have felt, however, that the writer’s personal contribution to the discussion was not very significant.

\subsubsection{The Mosaic}

\begin{quote}“The crystallizing force of Locke’s writing may be seen in the effect his Second Treatise of Government had in shaping some of the familiar features of our own government. That much criticized branch known as the Supreme Court and the combination of many powers in the hands of the executive under the New Deal are modern examples. But even the foundation of our state—the Declaration of Independence and the Constitution—have re-echoed its claims for human liberty, for the separation of powers, for the sanctity of private property. True, the influence of others is also marked in our Constitution—from the trend and aim of writers like Languet and Bodin, Hooker and Grotius to say nothing of Aristotle and the Stoic school of natural law; but the fundamental influence is Locke’s Treatise, the very quarry of liberal doctrines.”
\end{quote}

Note how the following phrases have been lifted out of the original text and moved into new patterns:
\begin{quote}
“crystallizing force of Locke’s writing”

“some of the familiar features of our own government”

“much criticized branch known as the Supreme Court”

“combination of many powers in the hands of the executive under the New Deal”

“have re-echoed its claims for human liberty…property”

“from the trend and aim…Grotius”

“to say nothing of Aristotle and…natural law”

“quarry of liberal doctrines”
\end{quote}
As in the first example, there is really no way of legitimizing such a procedure. To put every stolen phrase within quotation marks would produce an almost unreadable, and quite worthless, text.

\subsubsection{The Paraphrase}

\textbf{Paraphrase}
\begin{quote}
“Many fundamental aspects of our own government are apparent in the Second Treatise of Government. One can safely say that the oft-censured Supreme Court really owes its existence to the Lockean demand that powers in government be kept separate; equally one can say that the allocation of varied and widespread authority to the President during the era of the New Deal has still to encounter opposition because it is contrary to the principles enunciated therein… Once more it is possible to note the way in which Locke’s writing clarified existing opinion.”
\end{quote}

\textbf{Original}
\begin{quote}
“Many familiar features of our own government are apparent in the Second Treatise of Government. It is safe to assert that the much criticized… Court obtained its existence upon separation of powers; and that the combination of many powers in the hand of the executive under the New Deal has still to encounter opposition because it is contrary to the principles enunciated therein… Again we see the crystallizing force of Locke’s writing.”
\end{quote}

The preceding comparison shows how the writer has simply traveled along with the original text, substituting approximately equivalent terms except where his or her understanding falters, as it does with “crystallizing,” or where the ambiguity of the original requires too much ingenuity to decipher, as it apparently does as in “ to encounter opposition…consciously traced” in the original.

Such a procedure has its uses; for one thing, it is of value to the reader. How, then, may it properly be used? The writer might begin second sentence with “As Sherman notes in the introduction to his edition of the Treatise, one can safely say…” and conclude the paraphrase passage with a footnote giving the additional identification necessary. Or he or she might indicate directly the exact nature of what is being done, in this fashion: “To paraphrase Sherman’s comment…”and conclude that also with a footnote indicator.

In point of fact, this course of action does not particularly lend itself to honest paraphrase, with the exception of that one sentence, which the paraphrase above copied without change except for abridgement. The purpose of paraphrase would be to simplify, or to throw new and significant light on a text; it requires much skill if it is to be used honestly, and should be used rarely by the student except for the purpose, as suggested above, of personal enlightenment.

\subsubsection{The “Apt” Text}

\begin{quote}
“The Second Treatise of Government is a veritable quarry of liberal doctrines. In it the crystallizing force of Locke’s writing is markedly apparent. The cause of human liberty, the principle of separation of powers, and the inviolability of private property—all three major dogmas of American constitutionalism—owe their presence in our Constitution in large part to the remarkable Treatise which first appeared around 1685 and was destined to spark within three years a revolution in the land of its author’s birth and, ninety years later, another revolution against that land.”
\end{quote}

Here the writer has not been able to resist the appropriation of two striking terms—“quarry of liberal doctrines” and “crystallizing force”; a perfectly proper use of the terms would have required only the addition of a phrase: “The Second Treatise of Government is, to use Sherman’s suggestive expression, a “quarry of liberal doctrines.” In it the “crystallizing force”—the term again is Sherman’s—of Locke’s writing is markedly apparent.”

Other phrases in the text above—“the cause of human liberty,” “the principle of the separation of powers,” “the inviolability of private property”—are clearly drawn directly from the original source, but are so much matters in the public domain, so to speak, that no one could reasonably object to their reuse in this fashion.

Since one of the principal aims of college education is the development of intellectual honesty, it is obvious that plagiarism is a particularly serious offence, and the punishment for it is commensurately severe. What a penalized student suffers can never really be known by anyone except that student. The student who plagiarizes and “gets away with it” suffers something less public, and probably less acute, but the corruptness of the act, the disloyalty and baseness it entails, must inevitably leave a mark on him or her, as well as on the institution.
 %---------------------------------------------------------------------------------------
% END OF SECTION
%----------------------------------------------------------------------------------------
