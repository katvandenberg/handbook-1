
%----------------------------------------------------------------------------------------
%	Plagiarism
%----------------------------------------------------------------------------------------

\chapter{Plagiarism}

\section{Definition of Plagiarism}

Boston University's
\href{http://www.bu.edu/academics/resources/academic-conduct-code/}{\{Academic
Conduct Code\}} adopts a definition of plagiarism from Harold Martin and Richard
Ohmann's \emph{The Logic and Rhetoric of Exposition} (1963). They define plagiarism
as follows:

\begin{quote}The academic counterpart of the bank embezzler and of the manufacturer
who mislabels products is the plagiarist, the student or scholar who leads readers
to believe that what they are reading is the original work of the writer when it
is not. If it could be assumed that the distinction between plagiarism and honest
use of sources is perfectly clear in everyone’s mind, there would be no need
for the explanation that follows; merely the warning with which this definition
concludes would be enough. But it is apparent that sometimes people of goodwill
draw the suspension of guilt upon themselves (and, indeed, are guilty) simply
because they are not aware of the illegitimacy of certain kinds of “borrowing”
and of the procedures for correct identification of materials other than those
gained through independent research and reflection...

The spectrum is a wide one. At one end there is a word-for-word copying of
another’s writing without enclosing the copied passage in quotation marks and
identifying it in a footnote, both of which are necessary. (This includes, of
course, the copying of all or any part of another student’s paper.) It hardly
seems possible that anyone of college age or more could do that without clear
intent to deceive. At the other end there is the almost casual slipping in of a
particularly apt term which one has come across in reading and which so admirably
expresses one’s opinion that one is tempted to make it personal property. Between
these poles there are degrees and degrees, but they may be roughly placed in two
groups. Close to outright and blatant deceit—but more the result, perhaps, of
laziness than of bad intent—is the patching together of random jottings made in
the course of reading, generally without careful identification of their source,
and then woven into the text, so that the result is a mosaic of other people’s
ideas and words, the writer’s sole contribution being the cement to hold the
pieces together. Indicative of more effort and, for that reason, somewhat closer
to honest, though still dishonest, is the paraphrase, an abbreviated (and often
skillfully prepared) restatement of someone else’s analysis or conclusion,
without acknowledgment that another person’s text has been the basis of the
recapitulation.\end{quote}

Although there is a wide "spectrum" of plagiarism, Martin and Ohmann argue
that all examples of plagiarism may be placed within three broad categories:
the outright copying of another's words, the adoption of certain "apt" phrases 
or words from another without proper attribution, and a paraphrase that does 
not properly give credit to another author. Here are some examples of each of the
three categories of plagiarism.

Below you will find examples of all three kinds of plagiarism using the following original source text, taken from page 35 of Patrica Nelson Limerick's book, \emph{The Legacy of Conquest}:

\section{Original Source}

\begin{quote}
When academic territories were parceled out in the early twentieth century, anthropology got the tellers of tales and history got the keepers of written records.  As anthropology and history diverged, human differences that hinged on literacy assumed an undeserved significance.  Working with oral, preindustrial, prestate societies, anthropologists acknowledged the power of culture and of a received worldview; they knew that the folk conception of the world was not narrowly tied to proof and evidence.  But with the disciplinary boundary overdrawn, it was easy for historians to assume that literacy, the modern state, and the commercial world had produced a different sort of creature entirely--humans less inclined to put myth over reality, more inclined to measure their beliefs by the standard of accuracy and practicality (35).
\end{quote}

\section{Word-for-word copying}

\begin{Spacing}{1.5}
\begin{quote}
As we all know, when academic territories were parceled out in the early twentieth century, anthropology got the tellers of tales and history got the keepers of written records. This made historians assume that literacy, the modern state, and the commerical world had produced a different sort of creature entirely--humans less inclined to put myth over reality, more inclined to measure their beliefs by the standard of accuracy and practicality. 
\end{quote} 

\end{Spacing}

\ding{96} Word-for-word copying such as this is completely unacceptable, whether it occurs through a mistake or by design. Care must be taken when taking notes and typing quotations to avoid representing the words of other authors as your own. 

\section{Adoption of "apt" phrases}

\begin{Spacing}{1.5}
\begin{quote}
When the \hl{academic territories were parceled out in the early 1900s}, the disciplines \hl{diverged}. This made the differences that human beings had with regard to literacy \hl{assume an undeserved significance}. By \hl{overdrawing this disciplinary boundary}, historians began to believe that the subjects they studied \hl{were less inclined to put myth over reality and more likely to measure belief through the excellent standards of accuracy and practicality} (Limerick 35). 
\end{quote}
\end{Spacing}

\ding{96} This form of plagiarism is often the result of sloppiness in the taking of notes or during the drafting process. Using apt phrases from a source text is perfectly reasonable; however, make sure that quotation marks are used and citations are given.

\section{Paraphrase without attribution}
\begin{Spacing}{1.5}
\begin{quote}
During the early part of the last century the disciplines of anthropology and history separated. While anthropology focused on unlettered and illiterate communities, history became the study of societies who produced texts and records. Within the field of anthropology, a firm belief developed that oral cultures were characterized by mythological worldviews and superstitious beliefs; on the other hand, historians assumed that literate cultures were filled with individuals who only used reason and evidence to guide their thinking.
\end{quote}
\end{Spacing}

\ding{96} This paraphrase would be perfectly acceptable were it to have a citation indicating that the ideas are taken from another author's work.


 %---------------------------------------------------------------------------------------
% END OF SECTION
%----------------------------------------------------------------------------------------


