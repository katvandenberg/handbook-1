
%----------------------------------------------------------------------------------------
% Plagiarism
%----------------------------------------------------------------------------------------

\chapter{Plagiarism}

\section{Definition of Plagiarism}

Boston University's \href{http://www.bu.edu/academics/resources/academic-conduct-code}{\{Academic Conduct Code\}}  defines plagiarism as:

\begin{quote}Representing the work of another as one's own. Plagiarism includes but is 
not limited to the following: copying the answers of another student on an examination,
 copying or restating the work or ideas of another person or persons in any oral or 
written work (printed or electronic) without citing the appropriate source, and 
collaborating with someone else in an academic endeavor without acknowledging his or 
her contribution. Plagiarism can consist of acts of commission\textemdash appropriating the words 
or ideas of another\textemdash or omission failing to acknowledge / document / credit the source 
or creator of words or ideas \dots. It also includes colluding with someone else in an 
academic endeavor without acknowledging his or her contribution, using audio or video 
footage that comes from another source (including work done by another student) 
without permission and acknowledgement of that source.
\end{quote}

Consult the \href{http://www.bu.edu/academics/resources/academic-conduct-code}{\{Academic Conduct Code\}} 
for more information on academic dishonesty and a description of its severe consequences.

\section{Examples of Plagiarism}

In their \emph{The Logic and Rhetoric of Exposition} (1963), Harold Martin and Richard
Ohmann claim that while there is a large "spectrum" of plagiarism, we may comfortably
place them all within three broad categories: 1) the outright copying of another's words, 
2), the adoption of certain "apt" phrases or words from another without proper attribution, 
and 3), a paraphrase that does not properly give credit to another author. 

Below you will find examples of all three kinds of plagiarism. Each example plagiarizes a passage
taken from Patricia Nelson Limerick's book, \emph{The Legacy of Conquest}:

\subsection{Original source}

\begin{quote}
When academic territories were parceled out in the early twentieth century, anthropology got the tellers of tales and history got the keepers of written records.  As anthropology and history diverged, human differences that hinged on literacy assumed an undeserved significance.  Working with oral, preindustrial, prestate societies, anthropologists acknowledged the power of culture and of a received worldview; they knew that the folk conception of the world was not narrowly tied to proof and evidence.  But with the disciplinary boundary overdrawn, it was easy for historians to assume that literacy, the modern state, and the commercial world had produced a different sort of creature entirely\textemdash humans less inclined to put myth over reality, more inclined to measure their beliefs by the standard of accuracy and practicality (35).
\end{quote}

\subsection{Word-for-word copying}

\begin{Spacing}{1.5}
\begin{quote}
As we all know, when academic territories were parceled out in the early twentieth century, anthropology got the tellers of tales and history got the keepers of written records. This made historians assume that literacy, the modern state, and the commerical world had produced a different sort of creature entirely\textemdash humans less inclined to put myth over reality, more inclined to measure their beliefs by the standard of accuracy and practicality. 
\end{quote} 

\end{Spacing}

\ding{96} Word-for-word copying such as this is completely unacceptable, whether it occurs through a mistake or by design. Care must be taken when taking notes and typing quotations to avoid representing the words of other authors as your own. 

\subsection{Adoption of "apt" phrases}

\begin{Spacing}{1.5}
\begin{quote}
When the \hl{academic territories were parceled out in the early 1900s}, the disciplines \hl{diverged}. This made the differences that human beings had with regard to literacy \hl{assume an undeserved significance}. By \hl{overdrawing this disciplinary boundary}, historians began to believe that the subjects they studied \hl{were less inclined to put myth over reality and more likely to measure belief through the excellent standards of accuracy and practicality} (Limerick 35). 
\end{quote}
\end{Spacing}

\ding{96} This form of plagiarism is often the result of sloppiness in the taking of notes or during the drafting process. Using apt phrases from a source text is perfectly reasonable; however, make sure that quotation marks are used and citations are given.

\subsection{Paraphrase without attribution}
\begin{Spacing}{1.5}
\begin{quote}
During the early part of the last century the disciplines of anthropology and history separated. While anthropology focused on unlettered and illiterate communities, history became the study of societies who produced texts and records. Within the field of anthropology, a firm belief developed that oral cultures were characterized by mythological worldviews and superstitious beliefs; on the other hand, historians assumed that literate cultures were filled with individuals who only used reason and evidence to guide their thinking.
\end{quote}
\end{Spacing}

\ding{96} This paraphrase would be perfectly acceptable were it to have a citation indicating that the ideas are taken from another author's work.


 %---------------------------------------------------------------------------------------
% END OF SECTION
%----------------------------------------------------------------------------------------


