
%----------------------------------------------------------------------------------------
%	DOCUMENTATION OF SOURCES / PLAGIARISM
%----------------------------------------------------------------------------------------

\section{\textcolor{ForestGreen}{Documentation of Sources}}

Accurately documenting sources is a vital aspect of any process of inquiry. If you fail to properly document your sources your readers will be unable to follow your research, validate your claims, or judge the quality of your argument. Furthermore, failing to properly cite a source (whether summarized, paraphrased, or quoted) opens you to the charge of \textbf{plagiarism}, a serious academic offense.

\subsection{Definition of plagiarism}

Boston University's \emph{Academic Conduct Code} defines plagiarism as:

\begin{quote}Representing the work of another as one's own. Plagiarism includes but is not limited to the following: copying the answers of another student on an examination, copying or restating the work or ideas of another person or persons in any oral or written work (printed or electronic) without citing the appropriate source, and collaborating with someone else in an academic endeavor without acknowledging his or her contribution. Plagiarism can consist of acts of commission---appropriating the words or ideas of another---or omission failing to acknowledge/document/credit the source or creator of words or ideas \dots. It also includes colluding with someone else in an academic endeavor without acknowledging his or her contribution, using audio or video footage that comes from another source (including work done by another student) without permission and acknowledgement of that source.
\end{quote}
Consult the \emph{Academic Conduct Code} for more information on academic dishonesty and a description of its severe consequences:

\tab \url{http://www.bu.edu/academics/resources/academic-conduct-code/}

\subsection{Citing sources}

Scholars avoid plagiarism and give credit to the thinking and writing of others using a variety of citation formats, or ``styles.'' As you work to complete your degree in college you will encounter a number of these citation formats. In fact, each discipline has a preferred style. The humanities use MLA, psychology uses APA, history and other social sciences use Chicago. There are many others. As you begin to specialize in a particular field of study, you will be required to master the citation regime used by your discipline. In this brief handbook, however, we will introduce you to two of the most common styles: MLA and Chicago.

Although citation formats differ significantly, they all have two primary components: \textbf{in-text citations} and a \textbf{final bibliography}. As the name suggests, in-text citations are used to reference the work of others within the text itself; the bibliography contains an ordered list of all the in-text citations contained within a piece of writing.

%DONE------------------------
