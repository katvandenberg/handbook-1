%----------------------------------------------------------------------------------------
%Glossing/Reverse Outlining
%----------------------------------------------------------------------------------------

\chapter{Glossing/Reverse Outlining}

\section{What is it?}

Glossing is a method of creating an outline from a rough draft of an essay in order to 
check that draft's paragraph unity and coherence as well as the paper's overall 
arrangement and the accuracy of the thesis statement.

\section{How do I do it?}

Count the number of body paragraphs in your draft and create a corresponding 
numbered list.

For each paragraph, write one, and only one, sentence that starts with "This paragraph,"
and then  precisely states the claim or idea being developed in that paragraph.
        	
\textbf{Avoid vague sentences like these}:

\begin{quote} "This paragraph is about dogs." \end{quote}

Or:

\begin{quote} "This paragraph is about dogs and people who need service animals." 
\end{quote}
        	
\textbf{Instead, try sentences like these:}
                    	     	
\begin{quote} "This paragraph points out that service dogs are in short supply despite 
all the people willing to train them."\end{quote}
 
 Or:
                                	
\begin{quote} "This paragraph presents a synthesis of the arguments for and against 
using  small dogs as service animals." \end{quote}
                 
  	        
\section{What is the point of a gloss?}

It provides you with a quick outline of your paper. After constructing your gloss, you 
can glance at the list of sentences and determine the following:

        
\textbf{Whether or not your repeat yourself}: If you see the same point being made in 
more than one paragraph, think about deleting the extra paragraph(s), 
combining all the paragraphs sharing the same point (if you think each has 
something of value), or making the distinctions between the paragraphs more 
clear to your reader.
        
\textbf{Whether there is a logic to the order of the sentences, and thus to your paper}:
If your gloss sentences appear to be in a random order, think about how you would 
move them around so that there is logic to their arrangement.  Plug in transitional 
words (i.e. “In addition,” “However,” etc.)  in the spaces between the sentences to 
see if you can find ones that explain the relationship between the sentences (and thus 
the paragraphs). Once you are happy with the list, go back and cut/paste your 
paragraphs into the new order and insert transitional sentences, employing the words 
from your list, at the beginnings or ends of the paragraphs so that the order is clear to 
your reader.

\textbf{Whether your thesis matches your paper}: Are the points in the gloss clear in 
the thesis? Are they in the same order in the gloss as they are in the thesis? If the 
thesis and the gloss do not match up, consider which one you should rearrange. Also, a 
gloss can be a good way for you to construct a thesis in the first place. Just glance at 
your gloss and compose a 1-4 sentence statement covering the points outlined in the 
gloss.

\textbf{Whether your paragraphs are unified and coherent}: If, when constructing your 
gloss, you have trouble locating the main idea in a paragraph, or you find multiple main 
ideas in a paragraph, this is a sign that you need to revise that paragraph so that it 
develops one, and only, clear point (see \emph{Paragraph Coherence}).


%----------------------------------------------------------------------------------------
% END OF SECTION
%----------------------------------------------------------------------------------------

