
%----------------------------------------------------------------------------------------
%	Preamble
%----------------------------------------------------------------------------------------

\documentclass[12pt, hidelinks]{article} % Default font size is 12pt, it can be changed here


\usepackage{geometry} % Required to change the page size to A4
\usepackage{multirow}
%\usepackage{graphicx} % Required for including pictures
\usepackage{setspace}
\usepackage{hyperref}
\hypersetup{
    colorlinks,
    citecolor=black,
    filecolor=black,
    linkcolor=black,
    urlcolor=blue}
    
%\usepackage{float} % Allows putting an [H] in \begin{figure} to specify the exact location of the figure
\usepackage{wrapfig} % Allows in-line images such as the example fish picture
\usepackage{parskip}
\usepackage[usenames,dvipsnames,svgnames,table]{xcolor}


 \newcommand{\tab}{\hspace*{2em}}
\linespread{1.2} % Line spacing

\setlength\parindent{0pt}

%\graphicspath{{./Pictures/}} % Specifies the directory where pictures are stored

\begin{document}

%----------------------------------------------------------------------------------------
%	TITLE PAGE
%----------------------------------------------------------------------------------------

\begin{titlepage}

\newcommand{\HRule}{\rule{\linewidth}{0.3mm}} % Defines a new command for the horizontal lines, change thickness here

\center % Center everything on the page

\textsc{\LARGE Boston University}\\[1.5cm] % Name of your university/college
\textsc{\Large College of General Studies}\\[0.5cm] % Major heading such as course name
\textsc{\large Rhetoric Division}\\[0.5cm] % Minor heading such as course title

\HRule \\[0.4cm]
{ \huge \bfseries Team H Handbook}\\[0.4cm] % Title of your document
\HRule \\[1.5cm]

By

Professors Taylor and Vandenberg



\vfill % Fill the rest of the page with whitespace
\emph{2012}\\[.5cm] 
\end{titlepage}

%----------------------------------------------------------------------------------------
%	TABLE OF CONTENTS
%----------------------------------------------------------------------------------------

\tableofcontents % Include a table of contents

\newpage % Begins the essay on a new page instead of on the same page as the table of contents 



%----------------------------------------------------------------------------------------
%	INTRODUCTION
%----------------------------------------------------------------------------------------

\section{\textcolor{ForestGreen}{Rhetoric And Composition}}


\subsection{What is rhetoric?}

\textbf{At a glance}: 

Rhetoric is the study of oral, written, and visual persuasion. Oral speeches were an important part of social, political, and judicial life in ancient Greece, which gave birth to democratic government. Aristotle's work, \emph{The Art of Rhetoric}, is one of the earliest known books on the subject.

 
For the next 2,300 years, classical rhetoric was an integral part of Western culture: most educated citizens were required to take classes in it.
 
\textbf{Why do we need it in this class?} 

Rhetoric helps us determine how to use reason, emotion, and the strength of our own character to persuade others to change their views. It allows us to think critically about how others attempt to persuade us. It further provides methods for generating, arranging, and styling content. It directs our attention, as writers, to the importance of rhetorical situations, audience awareness, and voice.

\textbf{Where can I find more online information about rhetoric?}

Try the website \emph{Silva Rhetoricae}, or ``The Forest of Rhetoric,'' which explains the rhetorical proofs of ethos, pathos, and logos and lists and defines figures of speech.

\url{http://rhetoric.byu.edu/} 
%-------done-------

\subsection{What is composition?}

\textbf{At a glance}: 

Composition is a discipline concerned with how to write or compose language. To compose is to select and arrange words to achieve your purposes. There are many reasons to write--for example, you might write to express yourself, to entertain others, to persuade, to maintain contact, to clarify, or to synthesize material.
 
\textbf{Brief history}: 

Composition classes grew out of rhetoric classes. By the mid 19th century, due to the increase in students attending college, rhetoric courses shifted from emphasizing speaking to focusing on writing. During the 1960s in America, the modern professional study of composition began, with an emphasis on research that explained how best to teach college writing.
 
\textbf{Why take a composition class}?

Writing well is crucial for success in college and in the professional world. Learning to compose your thoughts in language that is clear, organized, sophisticated, precise, and educated will allow you to communicate effectively with peers, professors, colleagues, and employers.


\textbf{Are there any good websites that will help me with my writing}?
Visit Purdue University's Online Writing Lab, which has information on formatting and common writing errors, research tips, worksheets and exercises:

\url{http://owl.english.purdue.edu/} 

%--------------DONE`

%----------------------------------------------------------------------------------------
%	TYPES OF WRITING AT THE COLLEGE LEVEL
%----------------------------------------------------------------------------------------

\section{\textcolor{ForestGreen}{Types of Writing at the College Level}}

\begin{quote} [\textbf{Note}: there are seldom clear boundaries between the following types of papers. For instance, a research paper often includes summary, synthesis, and argument; an analysis can be an argument; a proposal is an argument often based in research; arguments often include narrative features, etc..] \end{quote}


\subsection{Expository}
A paper in which you report on, define, summarize, clarify and/or explain a concept or a process. The purpose of this paper is to provide information to an audience unfamiliar with the subject or to demonstrate for a professor that you have understood course material.  Research is often required. Clarity and organization are key.

\subsection{Synthesis}

A paper in which you pull together information from different sources on one topic or in which you examine the connections between the perspectives of various authors on one issue. It is essential that you move beyond summarizing each source and/or author discreetly, and clarify the relationships between ideas (whether from different sources or different authors). Thus, you must focus on where authors agree and disagree, or where ideas conflict or complement each other. Transitional words and phrases are extremely important for synthesis papers.

\subsection{Analysis/close reading}
Analysis requires you to look at a work (a reading, film, musical piece, etc.), a process, a person, or an issue and break it into smaller units in order to explain how those units function independently and how they work together to create the whole. You might, for example, do a rhetorical analysis to analyze how successfully an author persuades his audience, a visual analysis to explain how a film creates a certain mood or communicates a theme, or a poem to examine how figures of speech work to convey meaning at the level of the sentence.

\subsection{Argument}

A paper that requires you to make a claim about a debatable issue, which you then back up with evidence (examples, data, quotes from experts, etc.). An argument also usually requires you to recognize (and refute, concede, or undermine) opposing arguments and to qualify your claim.

\subsection{Response}

A paper written in response to a specific question or prompt. It may involve any of the other modes of writing. You should be sure you have a strong sense of the professor's expectations as well as a good grounding in the subject matter (which often comes from lectures, course readings, and discussions).

\subsection{Proposal}

An argument in which you propose that something should be done in the future. This often requires you to anticipate possible constraints and obstacles, qualify your claims, and pay close attention to the needs of the audience.

\subsection{Research paper}

Any paper that requires you to draw on sources other than your own thinking. Be sure you are familiar with the citation format required by each discipline/professor/class (i.e. APA, MLA, and Chicago Style), and that you understand what plagiarism is and how it can be avoided.

\subsection{Narrative}

A paper that tells a story, whether fictional, nonfictional, or a blend. Narratives can be used to entertain, inform, or persuade (or some blend).

%----------------------------------------------------------------------------------------
%	DOCUMENTATION OF SOURCES
%----------------------------------------------------------------------------------------

\section{\textcolor{ForestGreen}{Documentation of Sources}}

Accurately documenting sources is a vital aspect of any process of inquiry. If you fail to properly document your sources your readers will be unable to follow your research, validate your claims, or judge the quality of your argument. Furthermore, failing to properly cite a source (whether summarized, paraphrased, or quoted) opens you to the charge of \textbf{plagiarism}, a serious academic offense.

\subsection{Definition of plagiarism}

Boston University's \emph{Academic Conduct Code} defines plagiarism as:

\begin{quote}Representing the work of another as one's own. Plagiarism includes but is not limited to the following: copying the answers of another student on an examination, copying or restating the work or ideas of another person or persons in any oral or written work (printed or electronic) without citing the appropriate source, and collaborating with someone else in an academic endeavor without acknowledging his or her contribution. Plagiarism can consist of acts of commission---appropriating the words or ideas of another---or omission failing to acknowledge/document/credit the source or creator of words or ideas \dots. It also includes colluding with someone else in an academic endeavor without acknowledging his or her contribution, using audio or video footage that comes from another source (including work done by another student) without permission and acknowledgement of that source.
\end{quote}
Consult the \emph{Academic Conduct Code} for more information on academic dishonesty and a description of its severe consequences:

\tab \url{http://www.bu.edu/academics/resources/academic-conduct-code/}

%Done---------------------------------


%----------------------------------------------------------------------------------------
%	CITING SOURCES
%----------------------------------------------------------------------------------------
\subsection{Citing sources}

Scholars avoid plagiarism and give credit to the thinking and writing of others using a variety of citation formats, or ``styles.'' As you work to complete your degree in college you will encounter a number of these citation formats. In fact, each discipline has a preferred style. The humanities use MLA, psychology uses APA, history and other social sciences use Chicago. There are many others. As you begin to specialize in a particular field of study, you will be required to master the citation regime used by your discipline. In this brief handbook, however, we will introduce you to two of the most common styles: MLA and Chicago.

Although citation formats differ significantly, they all have two primary components: \textbf{in-text citations} and a \textbf{final bibliography}. As the name suggests, in-text citations are used to reference the work of others within the text itself; the bibliography contains an ordered list of all the in-text citations contained within a piece of writing.

%DONE------------------------

%----------------------------------------------------------------------------------------
%	MLA
%----------------------------------------------------------------------------------------

\subsection{MLA style} % Sub-section


\subsubsection{Formatting the MLA essay}
When setting up your word processor for an MLA-formatted document, use the following settings and rules:

\begin{itemize}
\item Set 1'' margins on all sides of the document.
\item Double-space the entire document, including block quotes.
\item Include your last name and page number on each page in the top right corner of the header.
\item In the top left portion of the first page, type your name, instructor's name, course title, and date on separate, double-spaced lines.
\item Include a centered title on the first page.
\item Indent the first line of each paragraph with a tab set to 0.5''.
\item When a quotation runs more than four typed lines, use a block quote. A block quote is a free-standing block of text, set apart from the rest of the text. Use the following four rules to format the quote: 

1) Begin the block quote on a new line. 

2) Indent every line of the quote 1'' from the left margin (two tabs). 

3) Do not use quotation marks around the quoted material. 

4) Place the parenthetical citation \emph{after} the final punctuation of the quoted passage. For example:

\begin{quote}
\textbf{Normal in-text citation:}

He argues that ``there is no such thing as `safety' scissors'' (Smith 89).

\textbf{Block quote}: 


Wolfgang Sapir explains the dangers of scissors:
\begin{doublespace}

\tab \tab Despite their comfort-inducing name, there is no such thing \\ \tab \tab as ``safety'' scissors. Running with scissors claims the lives \\ \tab \tab of hundreds of school children each year, and is statistically \\ \tab \tab one of the most dangerous threats faced by kindergartners \\ \tab\tab in all 50 states. (Sapir 89)

\end{doublespace}
\end{quote}

\end{itemize}

The following page is a model for the MLA-formatted document:

%-----------MLA FIRST PAGE EXAMPLE--------
\newpage
\thispagestyle{empty}
\begin{flushright}Johnson 1\end{flushright}
\bigskip
\begin{doublespace}
Jeff Johnson\\
Profesor Smith\\
English 130\\
January 28, 2010
\end{doublespace}
\begin{center}
America's Foreign Policy Future?
\end{center}
\begin{doublespace}

\tab While all of the world may indeed be a stage, we must be decisive in the measures we take in order to prevent the curtain from closing too soon. We currently live in an era of great political uncertainty. After the Cold War, America suddenly became the stand-alone world power, dominating international treaties, coalitions, and movements. This ``unipolarity'' has resulted in one of the most wholly economically prosperous and peaceful periods of time in history. The coexistence and international functionality that we enjoyed during the 90s for a large part can be credited to the United States' influential methods of maintaining world order.
\end{doublespace}
\newpage
%------DONE------------------

\subsubsection{In-text citations}
The MLA style uses parenthetical citations to indicate the author and page number of sources. These parenthetical citations take two forms. The first form is used when the source you are citing \emph{is known or understood} by your audience. The second form is used when the author being cited is \emph{unknown or unclear}. 

In the following sentence, the author of the source in question is obvious:

\begin{quote}According to scholar James Frey, ``Americans eat five pounds of ice cream annually" (78).
\end{quote}

Since the author, James Frey, is understood, the citation uses \textbf{only the page number} of the source.

In these versions of the sentence, however, the author is not stated:

\begin{quote}
According to one scholar, ``Americans eat five pounds of ice cream annually" (Frey 78).
\end{quote}
Or:

\begin{quote}
Studies have shown that the American people consume an average of five pounds of ice cream every year (Frey 78).
\end{quote}
In the first sentence, the author is referred to only as a generic ``scholar.'' To give the reader information on which scholar is being cited, Frey's name is included in the citation. In the second sentence, the author describes the report, not its author; as a result, the student has included Frey's name to indicate whose report is being referenced. 

%----DONE-----------

\subsubsection{Bibliography}

MLA requires a bibliography at the conclusion of the essay that includes the full citation of the sources cited within the essay. In MLA, the bibliography is known as the Works Cited page. When setting up a Works Cited page, use the following rules and characteristics:

\begin{itemize}
\item Center the words ``Works Cited'' at the top of the page.
\item Use your last name and the page number on the right side of the page's header.
\item Double-space the Works Cited entries.
\item Alphabetize the entries by the author's last name.
\item If an entry runs more than one typed line, indent the second (and any subsequent) line with a .5'' tab.
\item If two or more works by the same author are used, list the entries alphabetically by title. After the first entry, replace the author's name with three dashes followed by a period. (See the entries from St. Augustine on the following page).
\end{itemize}


Here is an example of a MLA formatted Works Cited page:

%-----------MLA Works Cited EXAMPLE--------
\newpage
\thispagestyle{empty}
\begin{flushright}Johnson 23\end{flushright}
\begin{center}
Works Cited
\end{center}
\begin{doublespace}

Adams, Robert M. \emph{James Joyce: Common Sense and Beyond}. Woodward: Classic \tab Nonfiction Library, 1967. Print.

Augustine. \emph{Tractates on the Gospel of John}. Trans. J. W. Rettig. Washington D.C.: \tab Catholic U 
of America P, 1988. Print.

- - -. \emph{Confessions}. Trans. Henry Chadwick. New York: Oxford UP, 1992. Print.

Carver, Craig. ``Molly: Bloom's Preservative; Correspondence and Function in \tab\emph{Ulysses}.'' \emph{James Joyce Quarterly} 12 (1975): 414-22. Print.

Garrett, Peter K. Introduction. \emph{Twentieth Century Interpretations of} Dubliners. \tab Ed. Peter K. Garrett. New York: Penguin, 1968. 1-17. Print.

\end{doublespace}

\newpage




%------DONE------------------
										                



Although there are hundreds of MLA forms for citing everything from books to email conversations, they may all be reduced to two basic forms. Let's call them ``book'' and ``periodical.'' All of the other forms are merely variations of these two.

\subsubsection{Book form}


\begin{quote}
{\bf Last Name, First Name. \emph{Title}. City: Publisher, Year. Medium \tab of Publication (Web or Print).}
\end{quote}

\medskip
\medskip

\begin{quote}
Taylor, Alan. \emph{William Cooper's Town: Power and Persuasion on the \tab Frontier of the Early American Republic}. Penguin Books, 2001. \tab Print.
\end{quote}

Variations on this form include a {\bf book by multiple authors}, a {\bf book with an editor}, a {\bf translated book}, an {\bf edition}, and so on. Each of these additional forms will require slight alterations or additions to the basic form. 

\subsubsection{Commonly used book forms}

\textbf{A book with an editor}: 
\begin{quote}
James, Henry. \emph{Portrait of a Lady}. Ed. Leon Edel. Boston: Houghton, \tab 1963. Print.
\end{quote}

\textbf{A book with a translator}: 
\begin{quote}
McDougle, Astrid. \emph{The Basics of Gaelic}. Trans. Paddy Maloney. New \tab York: Vintage, 1990. Print.
\end{quote}


\textbf{An edition (other than the first)}:
\begin{quote}
Thompson, Fred. \emph{Why I Fight}. 3rd ed. New York: Vanity Publications, \tab 2000. Print.
\end{quote}

\textbf{An anthology}: 
\begin{quote}
Savage, Fred, ed. \emph{American Literature about the Great Depression}. 12th \tab ed. New York: Longman, 2012. Print.
\end{quote}

\textbf{A work in an anthology}: 
\begin{quote}
Smith, Noel. ``The Dustbowl Aesthetic.'' \emph{American Literature about the \tab Great Depression}. Ed. Fred Savage. 12th ed. New York: Longman, \tab 2012. Print.
\end{quote}

\textbf{A republished book}: 
\begin{quote}
James, Esther. \emph{My Life}. 1946. New York: Random House, 2001. Print.
\end{quote}


\subsubsection{Periodical form}
\begin{quote}
\textbf{Last Name, First Name. ``Title.'' Journal Name Volume.Issue \tab (Year): Page Number range. Medium of Publication.}
\end{quote}
\medskip
\medskip
\begin{quote}
Luck, Chad. ``Re-Walking the Purchase: \emph{Edgar Huntly}, David Hume, and \tab the Origins of Ownership.'' \emph{Early American Literature} 44.2 (2009): \tab 271-306. Print.
\end{quote}

Common variations on the basic periodical form include a \textbf{newspaper article}, an \textbf{essay in a scholarly journal}, an \textbf{essay in a collection or edited volume}, an \textbf{article in a magazine}, and an \textbf{article on a website}.

\subsubsection{Commonly used periodical forms}

\textbf{An article in a scholarly journal with volume and issue numbers}:
\begin{quote}
Taylor, James. ``The Indian Matter of Charles Brockden Brown's \emph{Edgar \tab Huntly}.'' \emph{American Literature} 45.6 (1998): 432-45. Print.
\end{quote}


\textbf{An article in a scholarly journal with only volume numbers}:
\begin{quote}
Johnston, Johanna. ``A Reading of \emph{Moby Dick}.'' \emph{North Dakota Quarterly} \tab 45 (1978): 45-56. Print.
\end{quote}

\textbf{An article in a newspaper}:
\begin{quote}
McKinley, Robert. ``Cat Saved from Dog.'' \emph{New York Times} 7 Oct. 2011: \tab A12+. Print.
\end{quote}

\textbf{An article in a magazine}:
\begin{quote}
Smith, Jim. ``Remembering Tony.'' \emph{New Yorker} Jan. 2010: 12-18. Print.
\end{quote}

\textbf{An article in an online database}:
\begin{quote}
Taylor, Hayden. ``\emph{Moby Dick} and the Cold War.'' \emph{American Literature} \tab 45.6 (2010): 45-57. 
JSTOR. Web. 12 July 2012.
\end{quote}

%-----------Done--------

%----------------------------------------------------------------------------------------
%	THE CHICAGO STYLE
%----------------------------------------------------------------------------------------

\subsection{Chicago style}

\subsubsection {Formatting the Chicago essay}

When setting up your word processor for a Chicago-formatted document, use the following settings and rules:

\begin{itemize}

\item Use 1'' margins on all sides of the document.
\item Place your last name and page number on the right side of each page in the document's header.
\item Double-space throughout the document (except for block quotes, which are single-spaced).
\item Block quotes are formed when a quote runs five lines or more. Single-space the block quote. Indent the entire block of text with a 0.5'' tab from the left margin.
\item Endnotes and bibliographic entries are single-spaced with a blank line separating them.
\item Indent the first line of a note entry with a .5'' tab.
\item The Chicago form requires a title page. The title of the essay is centered about 1/3 down the page. Place your name at the center of the document. Near the bottom of the page place your course information and date on three separate, centered lines about 1/3 up from the bottom of the page.

\textbf{*}\textbf{Note}: the title page is \emph{counted but not numbered}. Therefore, begin your actual essay with page 2.
\end{itemize}

Models of the title page and first page of the Chicago-formatted essay may be found on the following pages:

%----------CHICAGO TITLE PAGE EXAMPLE
\newpage
\thispagestyle{empty}
\begin{doublespace}
\vspace* {3cm}
\begin{center}The War to End All Wars\end{center}
\vspace {4cm}
\begin{center}Jeff Smith\end{center}
\vspace {7.2cm}
\begin{center}History 101\\
Professor Crito\\
January 22, 2012\end{center}
\end{doublespace}
\newpage

%First Page of Chicago Essay


\thispagestyle{empty}
\begin{flushright}Johnson 2\end{flushright}
\bigskip
\begin{doublespace}

\tab While all of the world may indeed be a stage, we must be decisive in the measures we take in order to prevent the curtain from closing too soon. We currently live in an era of great political uncertainty. After the Cold War, America suddenly became the stand-alone world power, dominating international treaties, coalitions, and movements. This ``unipolarity'' has resulted in one of the most wholly economically prosperous and peaceful periods of time in history.\textsuperscript{1} The coexistence and international functionality that we enjoyed during the 90s for a large part can be credited to the United States' influential methods of maintaining world order.
\end{doublespace}
\newpage
%--------------DONE-------



\subsubsection {In-text citations}
The Chicago format uses either \textbf{endnotes} or \textbf{footnotes} to cite sources within the text. While the choice between them is left to the author, \emph{we require that you use endnotes} in the Chicago form since they prove less distracting and do not affect the page count of your written work. 

In the Chicago form, an in-text citation is indicated by a superscript number resembling the following:

\begin{quote}
Recent scholarship on the concept of sovereignty has displayed a remarkable lack of interest in the role of private property.\textsuperscript{7}
\end{quote}
This in-text reference will correspond to a citation at the conclusion of the document, such as this one:

\begin{quote}
\tab 7. Giorgio Agamben, \emph{Homo Sacer: Sovereign Power and Bare Life} (Stanford: Stanford UP, 1998), 96.
\end{quote}


\subsubsection{The notes page}





In the Chicago style, the endnotes appear on what is known as the notes page--a separate page that directly follows the conclusion of the essay. The notes page is organized as a numbered list that presents each citation in the order that it appears within the essay. Thus, your first citation will be endnote 1, your second will be endnote 2, and so on.

When setting up the notes page, use the following rules and characteristics:

\begin{enumerate}
\item Center the word ``Notes'' at the top of the page.
\item Single-space each individual endnote. Leave a blank line between entries.
\item Indent the first line of an endnote entry with one .5'' tab.
\end{enumerate}

In the interest of efficiency, the Chicago form uses a number of rules to streamline the work involved in presenting the essay's footnotes or endnotes. While this design ultimately means less typing, a number of strict rules must be followed:

\begin{enumerate}
\item Present the citations in the numerical order as they appear within the text.

\item The first time a source is cited, use the full Chicago notes form.

\item If the same source is used more than once, the shorthand version of the Chicago notes form is used the second (and each subsequent) time. The shorthand version contains \textbf{a)} the author's last name, \textbf{b)} a shortened version of the title, and \textbf{c)} the page number of the citation.

\item If a single source is used twice or more in a row, the Latin abbreviation ``ibid.'' is used along with the page number, rather than the shorthand version of the form. (Ibid. means ``in the same place.'')

\item If the same source is used \emph{twice in a row} and the citation \emph{is from the same page as the previous citation}, ibid. is used by itself \emph{without} the page number. 
\end{enumerate}

Here is an illustration of the formatting of the notes page:
\newpage

\thispagestyle{empty}
\begin{center}Notes\end{center}
\smallskip

\tab 1. Jeff Goldberg and Robert Smith, ``The Anger of the Average Joe,'' \emph{American History} 28, no. 5 (1987): 345-66.
\smallskip

\tab 2. Grady Little, \emph{An Uncivil Game} (New York: Random House, 2001), 230.
\smallskip

\tab 3. Goldberg and Smith, ``Average Joe,'' 350.
\smallskip

\tab 4. Little, \emph{Uncivil}, 110.
\smallskip

\tab 5. Ibid., 111.
\smallskip

\tab 6. Ibid.

\newpage





\subsubsection{The bibliography page}
Some professors may ask you to include both a notes page and a full bibliography page when using the Chicago format. As you will see in the next section, the bibliography form differs slightly from the notes form, so take care to use the correct one.

\begin{itemize}
\item Center the word ``Bibliography'' at the top of the page.
\item Place the bibliography page after the notes page.
\item Single-space each citation. Leave a blank line between entries.
\item Alphabetize by the author's last name.
\item Indent the second (and any subsequent) line of an entry with a .5'' tab.
\end{itemize}

The following page contains an example of the formatting for a Chicago bibliography page.
\newpage



%-----CHICAGO BIBLIOGRAPHY    PAGE--------


\thispagestyle{empty}
\begin{center}Bibliography\end{center}

Carter, Jimmy. \emph{Life as a Peanut Farmer: Words of Wisdom for the Southern Man}. \tab Atlanta: Peach Press, 1984.

Graves, Hugo. \emph{Living with Armadillos, a Practical Guide}. Austin: Dunder-Mifflin, \tab 1998.

Hide, Frank. ``My Life in a South African Prison: How I Survived.'' \emph{Penal System \tab Quarterly} 
77, no. 3 (1987): 12-34.

Zither, Les. ``Preserving the Land of the Oswegos.'' \emph{New York Times}, October 5, \tab 2012. http://www.nytimes.com/2012/10/5/frontpage/445683.html


\newpage




\subsubsection{Common Chicago forms}

In each of the following, the first item is the note form; the second item is the bibliography form.
	 	 	
\textbf{Book by one author}:

\begin{quote}
\tab 1. James McClintock, \emph{The Greek Polis: A History} (New York: Knopf, 2000), 23.

McClintock, James. \emph{The Greek Polis: A History}. New York: Knopf, \tab 2000.
\end{quote}


\textbf{Edited collection}:
\begin{quote}
\tab 12. Russell Simmons, ed., \emph{Of Moose and Men: Reflections on Hunting New Hampshire} (Concord: Granite Press, 1997), 123.

Simmons, Russell, ed. \emph{Of Moose and Men: Reflections on Hunting New \tab Hampshire}. Concord: Granite Press, 1997.
\end{quote}

\textbf{A Book with an author and editor}:

\begin{quote}
\tab 2. Grace Helen, \emph{The City of Granite}, ed. Daniel Miller (New York: Pantheon, 2001), 50.

Helen, Grace. \emph{The City of Granite}. Edited by Daniel Miller. New York: \tab Pantheon, 2001.
\end{quote}

\textbf{A translation}:
\begin{quote}
\tab 1. Esther Buffon, \emph{Tiny Deaths}, trans. John Smith (New York: Francophone Press, 1987), 43.

Buffon, Esther. \emph{Tiny Deaths}. Translated by John Smith. New York: \tab Francophone Press, 1987.
\end{quote}


\textbf{An edition other than the first}:

\begin{quote}
\tab 13. Daniel Graves, \emph{Interdisciplinary Moves}, 2nd ed. (New York: Jazz Press, 1999), 56.

Graves, Daniel. \emph{Interdisciplinary Moves}. 2nd ed. New York: Jazz Press, \tab 1999.
\end{quote}

\textbf{A selection from an anthology}:
\begin{quote}

\tab 45. Johanna Burden, ``Women's Muck,'' in \emph{Southern Women Writers}, ed. Jeff Goldblume (Nashville: Magnolia Press, 2005), 34-78.

Burden, Johanna. ``Women's Muck.'' In \emph{Southern Women Writers}, \tab edited by Jeff Goldblume, 
34-78. Nashville: Magnolia Press, 2005.
\end{quote}

\textbf{Article in scholarly journal}:
\begin{quote}
\tab 16. Jeff London, ``The War of 1812,'' \emph{Journal of American History} 87, no. 1 (1998): 10.

London, Jeff. ``The War of 1812.'' \emph{Journal of American History} 87, no. \tab 1 (1998): 1-24.
\end{quote}

\textbf{Journal article from an online database}:
\begin{quote}

\tab 16. Jeff Bags, ``Teaching Rules for Educators,'' \emph{Teacher's Quarterly} 45 (1999): 34, doi: 11.2288/1128767384937.

Bags, Jeff. `Teaching Rules for Educators.' \emph{Teacher's Quarterly} 45 \tab (1999): 30-49. doi: 11.2288/1128767384937.
\end{quote}

\textbf{*}If the article does not have a doi number listed, use the stable url to the article in its place. If that is not available, simply list the name of the database in question (JSTOR, Academic Search Complete, EBSCO, etc.).


\textbf{Magazine article}:
\begin{quote}
\tab 17. Brian Taylor, ``Making a Business,'' \emph{Atlantic}, June 10, 2012, 23.

Taylor, Brian. ``Making a Business.'' \emph{Atlantic}, June 10, 2012, 20-24.
\end{quote}


\textbf{Newspaper article}:
\begin{quote}
\tab 18. Don Osborne, ``The Trouble with Children,'' \emph{New York Times}, February 26, 2011, A1.

Osborne, Dan. ``The Trouble with Children.'' \emph{New York Times}, February \tab 26, 2011, A1.
\end{quote}

%----------------------------------------------------------------------------------------
%	WORKING WITH SOURCES
%----------------------------------------------------------------------------------------

\section{\textcolor{ForestGreen}{Working with Sources}}
\subsection{Quotation}
 
\subsubsection{When should I quote something?}

Quoting is word-for-word borrowing from source material. Only quote material when:
 
\begin{enumerate}

\item You are going to be discussing the language used in a specific part of a text (i.e., if you are interpreting a line of poetry, or analyzing a portion of a literary text).
 
\item You are calling on the words of a known authority, on whose credibility you are depending (i.e., if you are discussing MLK Jr's stance on equality).
 
\item You are referring to highly technical data or specific numbers, dates, or statistics.
 
\item You are relying on a selection of text that is uniquely or precisely worded.
 
 \end{enumerate}
 
\subsubsection{How do I integrate quoted material?}
 
 \begin{itemize}       	
\item Introduce it with a signal phrase (see \emph{Signal Phrases}), so the quote is not just ``dropped'' into the paper.

\item Use quotation marks.

\item Use either an in-text citation (MLA) or a footnote/endnote (Chicago).
        	
\item Use brackets to insert your own words into a quote (this may be necessary to ensure that the grammar of the quote matches the grammar of your text).

\textbf{For example}:

\begin{quote} Simon Frith believes we ``use music to organize our sense of time [and] to make our private feelings public'' (87). 
\end{quote}

\item Use ellipsis to condense a passage of which you only need parts. Use a period after a sentence and before the ellipsis if you are omitting a whole sentence (for a total of 4
dots).

\textbf{For example}:
 If the source reads as follows:
\begin{quote}
Rock music depends on the myth of authenticity. Paglia is wrong to believe that only contemporary rock musicians play music to make money. Rock music has never been authentic and spontaneous, or particularly revolutionary.
\end{quote}
         
You might write:
\begin{quote}
Gracyk believes that ``Rock music depends on the myth of authenticity. \ldots Rock music has never been authentic\ldots  or particularly revolutionary'' (23).
 \end{quote}
 \end{itemize}
 
\subsubsection{What should I avoid?}

\begin{itemize}
\item Avoid relying excessively on quotes. If you quote too often it can make it appear that you have not fully read, understood, and ``digested,'' the source material or that you are not comfortable putting it in your own words.

\item Relying excessively on long quotes (see above).

\item Inserting a quote and then doing nothing to discuss it, expand on it, clarify it, or connect it to surrounding material.

\end{itemize}


\subsection{Signal Phrases/Integrating Research}
 
 
\subsubsection{What are they?}
 
Signal phrases are words used to signal that the material you are incorporating is borrowed.
 
\subsubsection {Why use them?}
 
They make it clear that you are transitioning from your own material/thinking to the material/ideas of another. This makes your paper more coherent.
 
They make it clear when you have begun to paraphrase or summarize (unlike quotations, paraphrases and summaries are not formatted with quotation marks; therefore, it is difficult for readers to know when/where you have begun to paraphrase or summarize unless you include these phrases).
 
They make the tone of your paper more academic/make your use of research clear.
 
They compel you to articulate, precisely, the stances of those from whom you have borrowed. This will direct your attention to the precise ways in which authors agree or disagree with you or each other, and allow you to make these intersections clear to your reader.
 
Using these signal phrases will help you to avoid plagiarism.
 
 
\subsubsection{How do you construct a signal phrase?}
 
\textbf{Use the author's name}. The first time you mention an author, include the author's name, the title of his or her work, and perhaps a brief statement indicating the author's credentials. (Once you've introduced an author in your paper, just use his or her last name if you mention him or her again).

\textbf{Use a strong verb to characterize what the author has “done.”} See the list below for suggestions. Be sure to pick the verb which most precisely articulates the author's action:
 
\begin{quote}asserts, argues, believes, claims, emphasizes, insists, observes, reports, suggests, acknowledges, admires, agrees, corroborates, endorses, extols, praises, verifies, illustrates, expands on, rejects, complicates, contends, contradicts, denies, disagrees, refutes, questions, warns, proposes, implores, exhorts, demands, calls for, recommends, urges, advocates, wonders, asks, rejects, encourages.\end{quote}
 
 
\textbf{Example}:
 
 \begin{quote}
In his article, ``Writing is Thinking,'' Dr. Benjamin Warhol, a composition scholar, rejects the idea that one must have an outline prepared before one drafts a paper. While recognizing that some students must prepare their thoughts beforehand, Warhol warns that too much time spent on outlines could actually prevent students from completing papers on time (22-25). He is correct--writing  is a process, and often one must draft before one  can realize what one has to say much less in what order one wants to say it. However, Sally T. Osmond, poet and  creative writing instructor at Fordham, disagrees, noting that... \end{quote}

\subsection{Paraphrasing}

A paraphrase involves taking a paragraph or passage from a source, putting it in your own words, and arranging it within your own original sentence structures. Although a paraphrase is similar in length to the original passage, it is sometimes employed to condense and clarify the main points of that passage, and, therefore, may be somewhat shorter in length.

\subsubsection{Why are paraphrases important?}

They allow you to demonstrate your knowledge of and comfort with other sources and may provide strong support for your argument or analysis or help you clarify an idea.

\subsubsection{How do I incorporate them?}

\begin{itemize}
\item Employ a signal phrase (see \emph{Signal Phrases}) to introduce them.

\item End each with an in-text citation (see \emph{MLA Style}) or (\emph{Chicago Style}).
\end{itemize}

\subsubsection {What should I avoid?}

\begin{itemize}

\item Plagiarizing--remember, paraphrased material is still borrowed material, even though you have put it in your own words (see \emph{Definition of plagiarism}).

\item Merely changing a word or a phrase here or there. Instead, read the passage until you can put it aside and write your paraphrase without having to look back at it.
\end{itemize}


\subsection{Summarizing}

\begin{itemize}

\item Reviewing the main idea(s) of a page, pages, chapter/scene/stanza, or entire work.

\end{itemize}
\subsubsection {How is it different than paraphrasing?}

\begin{itemize}
\item A summary is much shorter than the source.
\end{itemize}
\subsubsection {Why are summaries important?}

\begin{itemize}
\item They demonstrate that you have read and comprehended source material. They can provide context for your argument or help you clarify a concept.
\end{itemize}

\subsubsection{How do I incorporate them?}

\begin{itemize}
\item Employ a signal phrase (see \emph{Signal Phrases}).

\item End each with an in-text citation (MLA) or footnote/endnote (Chicago), noting the pages summarized.
\end{itemize}

\subsubsection{What should I avoid?}
\begin{itemize}

\item Plagiarizing--remember, summarized material is still borrowed material, even though you have greatly condensed it and put it entirely in your own words.
 \end{itemize}

%----------------------------------------------------------------------------------------
%	Introductions
%----------------------------------------------------------------------------------------

\section{\textcolor{ForestGreen}{Introductions}}


 
\subsection {Why is the introduction important?}
 
\begin{itemize}
\item It is your first chance to establish your ethos [ethos=the sense of yourself that you create through your writing/the impression your reader gets of you due to your writing].
 
\item It is the part of the paper where you orient your reader to what is to follow in the rest of your paper.
 
\item It is your opportunity to hook/engage your audience, to make them interested.
\end{itemize}

\subsection{How do I write a strong introduction?}

\begin{enumerate}
\item \textbf{Hook your reader with a vivid and engaging opening sentence by avoiding general or vague sentences}.
        	               	        	
\begin{quote}

\textbf{Instead of}: ``Drugs are a big problem in the U.S. today.''

\textbf{Try}:``When Shay Howard was 5, she watched her mother
overdose on  the bathroom floor.''
\end{quote}
        	    	
\item {\textbf{Begin with a question, provocative statement, interesting data or statistic.}}

\begin{quote}
                            	                   
\textbf{Instead of}: ``There are a lot of debates about the sale of cells today.''

\textbf{Try}: ``Although Henrietta Lacks died in the 1950s, trillions
      of her living cells have since gone to the moon and  
      been used to cure cancer, all without her prior
      consent.''     
\end{quote}
                    	 
\item \textbf{Offer a brief introduction to the material in your paper, probably around 4-8 lines}. 

If you have not done so in your opening sentence, and will not do so in your thesis, be sure to mention any works, authors, or concepts central to the paper, though you should avoid doing so in a list-like fashion.
 
\item \textbf{Write your thesis} (see: \emph{Thesis Statements}).

After you have written your thesis, put it next to the prompt/assignment question to see if it addresses exactly what it is the professor has asked you to address.  
  
 \end{enumerate}

 
\subsection{Helpful hints on introductions}

\begin{itemize}
\item Avoid the clich\'ed phrase ``In today's world.''
        	
\item Do not employ a dictionary definition.
       	
\item In general, work to keep your introduction to half a page or less.    	  
 
\item Proofread your introduction very carefully and work on your style--remember, this is where you are making your first impression. You want to sound educated, on-topic, clear, and coherent.
 \end{itemize}

%----------------------------------------------------------------------------------------
%Conclusions
%----------------------------------------------------------------------------------------


\section{\textcolor{ForestGreen}{Conclusions}}  
\subsection{Why is the conclusion important?}

\begin{itemize}
\item It is the place where you draw connections between all your points and where you emphasize the most significant material covered.
 
\item On a practical note, it may be the last thing your professor reads before assigning a grade to the paper, so end on a strong note. Your conclusion should not read like an afterthought or an attempt to meet a minimum page requirement. 
\end{itemize}

\subsection{How do I write a strong conclusion?}
        	
\begin{itemize}

\item Provide a transition from the body of your paper (see \emph{Transitions}).
 
\item Emphasize the main points of your paper, drawing connections between them as appropriate to create coherence (see: \emph{Transitions}).
 
\item End with a strong final sentence (see advice above regarding the opening sentence)

\end{itemize}

\subsection{What should I avoid?}

\begin{itemize}        	
\item \textbf{Avoid tacking on an inappropriately happy ending}. 

For instance, if you have just written a whole paper explaining how serious the drug problem in the U.S. is, it is not appropriate to end with something  upbeat and off-topic.
 
\begin{quote} \textbf{For example}: ``Everything usually works out in the end, and we can figure this out.''
 
\end{quote}
 
\item \textbf{Do not end a persuasive piece by including a feel-good and inaccurate statement that undermines your entire argument.}

\begin{quote}
\textbf{For example}:
        	``But that's just my opinion, and everyone is entitled to their own opinion.''
 \end{quote}
        	
A personal opinion is different than an academic opinion. If you have made a claim, backed it, and anticipated reasonable opposing arguments with the intent to persuade, this is quite different than just expressing your beliefs in the interest of sharing, not   persuading. The sample sentence above refers to the latter type of opinion and thus would be an inappropriate way to end an academic argument.
 
\item \textbf{Avoid merely rephrasing your thesis statement}.

Since you have already introduced your points in the introduction and developed them in the body of your paper, you should use the conclusion to word these main points in such a way that they are particularly striking and memorable. The point is to emphasize them as you conclude. You might consider using figures of speech such as anaphora, alliteration, polysyndeton, etc.. For a comprehensive list and description of rhetorical figures, see the following website:

\url{http://rhetoric.byu.edu/}
 
 
\item \textbf{Do not introduce a totally new idea or go off topic}.
 
\item \textbf{Avoid merely listing a bunch of points from your paper}.

You should be able to discern which points are the most significant. Emphasize these.
 
\item \textbf{Do not tack on material just to meet a length requirement}.

If your paper needs more material, go back to your body paragraphs and see where you can or should expand on a point or add necessary points.

\end{itemize}

%----------------------------------------------------------------------------------------
%Thesis Statements
%----------------------------------------------------------------------------------------

\section{\textcolor{ForestGreen}{Thesis Statements}}

\subsection{Why do I need a thesis statement?}

A thesis statement is your way of introducing your readers to the main ideas of your paper while indicating to them the order in which you will address these ideas. It serves as a miniature outline for your paper.

\subsubsection{Where does it belong?}

Readers are typically expecting to see the thesis statement at the end of the introduction. For papers under 10-12 pages, this usually means it will be on the first page of your paper.

\subsubsection{When should I write it?}

Unless your professor has required you to submit a thesis before a draft is due, you may write it at any point during the drafting process. Many students do not know exactly what they want their paper to cover until they have completed a first draft. Thus, do not feel obligated to have a thesis composed before you begin drafting.

\subsubsection{How long can it be?}

Many high school students are taught to keep thesis statements to one sentence, listing 3 ideas for 3 main body paragraphs (i.e. ``The death penalty should be eliminated because it is expensive, ineffective, and immoral.'') However, most college papers are much longer and more complex and thus require a more lengthy thesis statement.

As a general guideline, 3-4 sentences should be plenty. You don't want to overwhelm the reader or go into too much detail before it is necessary.

\subsubsection{What should it do?}

\begin{itemize}
\item List the main ideas of your paper as precisely as possible.

\item List those points in the order in which they appear in the paper.

\end{itemize}

\subsubsection{What makes a thesis statement strong?}

\begin{itemize}
\item Being precise (see \emph{Precision}) and concise (see \emph{Concision}).

\item Using parallel structure (see \emph{Parallel structure}.

\item Employing transitions.
\end{itemize}

\subsubsection{What should I avoid?}

\begin{itemize}
\item Avoid being vague about your actual points or narrating what you will do in the paper (i.e., ``In this paper I will make some points about how the film works and the many effects of the film style'').

\item Don't raise points that you do not develop in your paper.

\end{itemize}

\textbf{For example}:

\textbf{A weak thesis}:  “All of these authors discuss many points. I will talk about which points I agree with. There are also many points with which I disagree. These will also be talked about in this paper.”


\textbf{Revised}: While Pauline Erera believes a family can and should take many forms, Wade Horn and James Wilson disagree, arguing for the importance of the nuclear family to the proper upbringing of children. Although Erera's willingness to embrace diversity is admirable, Horn and Wade are correct in asserting that the traditional two-parent home ensures safe, happy, and educated children in a way nontraditional families cannot.

%----------------------------------------------------------------------------------------
%Paragraph Coherence
%----------------------------------------------------------------------------------------

\section{\textcolor{ForestGreen}{Paragraph Coherence}}

\subsection{What is paragraph coherence?}

A paragraph is a unit of text devoted to developing an idea. The idea is usually stated in a topic sentence.

A paragraph is coherent when every sentence is devoted to developing the main point of the paragraph, the sentences are in a logical order, the connections between sentences are clear, unnecessary repetition is avoided, and the reader can easily follow the development of your point from your first sentence to your last.

It is the opposite of incoherence, which occurs when a paragraph is disorganized and repetitive and makes little sense to the reader.

\subsection{How can I make my paragraphs coherent?}

\begin{enumerate}

\item Identify the main idea of the paragraph then remove any sentence that does not develop that main idea or repeats ideas offered in another sentence.

\item Consider the order of your sentences. Is there a reason why they are in the order they are, or do they need to be rearranged to make sense to a reader?

\item Think about employing some transitions (see \emph{Transitions}) at the beginning of some of the sentences. These will help you pinpoint the relationships between your ideas/sentences and thus clarify these relationships for your reader.

\item Repeat key words.

\item Be concise (see \emph {Concision}) and precise (see \emph{Precision}).
\end{enumerate}
 
 
\textbf{For example}:

\begin{quote}
\textbf{An incoherent paragraph}: 

Ethos is the ethical appeal. This author talks about a lot of different arguments for legalizing organ sale. In rhetoric, ethos is very important. You shouldn't be able to sell your organs. In China there have been some human rights investigations into the executions of political prisoners and the way their organs were sold. The author thinks you should be able to sell your organs, but I think you shouldn't be able to.  I don't know why the author has to be so hostile when he writes, calling his opponents ``small-minded.''  Just because he thinks you should be able to sell your organs, he is not being ethical.  And why is he so sarcastic in passage 5?  A lot of people don't agree with selling organs; they think we should just donate them. He is not persuading me because of his ethos.
\end{quote}

\begin{quote}
\textbf{Revision}:
          
When discussing the complicated issue of  selling organs, this author fails to persuade his audience that such sales should be legal because he fails to construct a strong ethos.  Ethos is the impression a writer creates of himself through his written text. To be persuasive, the writer must employ the ethical appeal in such a way that he appears educated and well-intentioned. As well, he should demonstrate that he shares the audience's values. Consequently, when this author is sarcastic (as he is in passage 5), or hostile (when he belittles his opponents by calling them small-minded),  he alienates those he most wishes to persuade, those who disagree with him, like me. While I might have been persuaded if he had offered more data (for instance on the supposed black market in China) and dealt gracefully with his opponents, in this instance, I was not.

\end{quote}


%----------------------------------------------------------------------------------------
%Transitions
%----------------------------------------------------------------------------------------
\section{\textcolor{ForestGreen}{Transitions}}

\subsection {What are transitions?}

Transitions are words or phrases that connect ideas within or between sentences--transitions allow readers to follow your ideas across a paper.

\subsection{Why are they important?}

\begin{itemize}
\item They make your writing coherent (see \emph{Paragraph Coherence}). 

\item They can help you vary your sentence structure, which often improves your style.

\item Their employment often prompts you to consider, carefully, the logic behind the order of your points.

\end{itemize}
         

\textbf{If you want to clarify a causal relationship between ideas}, you might employ the following transition words: \emph{since}, \emph{so}, \emph{consequently}, \emph{as a result}, \emph{therefore}, \emph{then}, \emph{thus}, \emph{because}, or \emph{due to}.
 
 \tab \textbf{Example}: ``As a result of his hard work, he aced the exam.''

\textbf{If you want to contrast one idea with another}, you might use these words: \emph{however}, \emph{on the other hand}, \emph{in contrast}, \emph{on the contrary}, \emph{yet}, \emph{rather}, or \emph{while}.

\tab \textbf{Example}: ``Silver endorses human cloning, while Krauthammer does not.''

\textbf{If you wish to show that ideas are similar}, you might depend on the following words: \emph{likewise}, \emph{similarly}, \emph{in much the same way}, or \emph{also}.

\tab \textbf{Example}:  ``In much the same way that Frith values pop music, Berry loves rap.''      

\textbf{If you wish to use an example}, you might employ these words: \emph{for example}, \emph{for instance}, or \emph{in illustration}.

\tab \textbf{Example}: ``He neglected his paperwork; for instance, he forgot to file his taxes.''

\textbf{If you wish to show how multiple points build on each other}, you might want these: \emph{Futhermore}, \emph{in addition}, \emph{also}, \emph{and}, \emph{moreover}

\tab \textbf{Example}: ``It important to analyze film.  And it is essential to attend to film style.''

\textbf{If you wish to elaborate on a point}, you might try these: \emph{in short}, \emph{ultimately}, \emph{by extension}, \emph{that is}, \emph{in other words}.

\tab \textbf{Example}: ``Cloning human is immoral. That is, it undermines the moral value humans place on life.''
        
\textbf{If you wish to concede a point}, you might employ the following: \emph{although}, \emph{admittedly}, \emph{granted}, \emph{of course}, \emph{naturally}.

\tab \textbf{Example}: ``Granted, not all tax breaks are good for the economy.''
        	
\textbf{Finally, if you wish to sum up an idea}, these will be of use: \emph{in conclusion}, \emph{as a result}, \emph{in sum}, \emph{finally}, \emph{therefore}, \emph{thus}, \emph{in short}.

\tab \textbf{Example}: ``In sum, transitions are an important part of a coherent essay.''
              	
%----------------------------------------------------------------------------------------
%Glossing/Reverse Outlining
%----------------------------------------------------------------------------------------

\section{\textcolor{ForestGreen}{Glossing/Reverse Outlining}}

\subsection{What is it?}

Glossing is a method of creating an outline from a rough draft of an essay in order to check that draft's paragraph unity and coherence as well as the paper's overall arrangement and the accuracy of the thesis statement.

\subsection{How do I do it?}

Count the number of body paragraphs in your draft and create a corresponding numbered list.

For each paragraph, write one, and only one, sentence that starts with ``This paragraph,'' and then  precisely states the claim or idea being developed in that paragraph.
        	
\textbf{Avoid vague sentences like these}:

\begin{quote} ``This paragraph is about dogs.'' \end{quote}

Or:

\begin{quote} ``This paragraph is about dogs and people who need service animals.'' \end{quote}
        	
\textbf{Instead, try sentences like these:}
                    	     	
\begin{quote} ``This paragraph points out that service dogs are in short supply despite all the people willing to train them.''\end{quote}
 
 Or:
                                	
\begin{quote} ``This paragraph presents a synthesis of the arguments for and against using  small dogs as service animals.'' \end{quote}
                 
  	        
\subsection{What is the point of a gloss?}

It provides you with a quick outline of your paper. After constructing your gloss, you can glance at the list of sentences and determine the following:
 \begin{itemize}
        
\item \textbf{Whether or not your repeat yourself}: If you see the same point being made in 
more than one paragraph, think about deleting the extra paragraph(s), 
combining all the paragraphs sharing the same point (if you think each has 
something of value), or making the distinctions between the paragraphs more 
clear to your reader.
        
\item \textbf{Whether there is a logic to the order of the sentences, and thus to your paper}:
If your gloss sentences appear to be in a random order, think about how you would move them around so that there is logic to their arrangement.  Plug in transitional words (i.e. “In addition,” “However,” etc.)  in the spaces between the sentences to see if you can find ones that explain the relationship between the sentences (and thus the paragraphs). Once you are happy with the list, go back and cut/paste your paragraphs into the new order and insert transitional sentences, employing the words from your list, at the beginnings or ends of the paragraphs so that the order is clear to your reader.

\item \textbf{Whether your thesis matches your paper}: Are the points in the gloss clear in the thesis? Are they in the same order in the gloss as they are in the thesis? If the thesis and the gloss do not match up, consider which one you should rearrange. Also, a gloss can be a good way for you to construct a thesis in the first place. Just glance at your gloss and compose a 1-4 sentence statement covering the points outlined in the gloss.

\item \textbf{Whether your paragraphs are unified and coherent}: If, when constructing your 
gloss, you have trouble locating the main idea in a paragraph, or you find multiple main ideas in a paragraph, this is a sign that you need to revise that paragraph so that it develops one, and only, clear point (see \emph{Paragraph Coherence}).
\end{itemize}

 
 

 
%----------------------------------------------------------------------------------------
%Proofreading for Errors
%----------------------------------------------------------------------------------------
 
\section{\textcolor{ForestGreen}{Proofreading for Errors}}

\begin{enumerate}

\item \textbf{Comma splice}

A comma splice occurs when the writer combines two or more independent clauses (i.e., clauses that could stand alone as sentences) with only a comma. To revise, use a comma with a coordinating conjunction (and, or, but, for, so, yet); employ a semicolon; or break the clauses up and punctuate them as separate sentences.
\begin{quote}
\textbf{[Wrong]}: The Cherokee Indians once ruled the Tennessee valley, they are all gone now.

\textbf{[Revised]}:  The Cherokee Indians once ruled the Tennessee valley, but they are all gone now.

\textbf{[Revised]}: The Cherokee Indians once ruled the Tennessee valley; they are all gone now.
\end{quote}
\item \textbf{Run-on}

A run-on is very similar to a comma splice; it is also created when the writer combines two or more independent clauses incorrectly, in this case by not providing any punctuation between them.  To revise, do as you would with a comma splice and use a comma with a coordinating conjunction; employ a semi-colon; or break the clauses up and punctuate them as separate sentences.
\begin{quote}
\textbf{[Wrong]}: The medication had a side effect it caused severe dry mouth.

\textbf{[Revised]}: The medication had a side effect; it caused severe dry mouth.

\textbf{[Revised]}: The medication had a side effect. It caused severe dry mouth.
\end{quote}
\item \textbf{Sentence Fragment}

A sentence fragment is a dependent clause treated as an independent clause or sentence. It does not express a complete thought. Often, though not always, these start with the following words: Although, While, Because, Whether, Such, and If.
\begin{quote}
\textbf{[Wrong]}: BU offers many classes. Such as Accounting and English.

\textbf{[Revised]}: BU offers many classes, such as Accounting and English.
\end{quote}
\item \textbf{Misplaced or dangling modifier}

A misplaced or dangling modifier occurs when a word or phrased is not placed next to the word it modifies, thus altering the meaning of a sentence, often in comical ways.
\begin{quote}
\textbf{[Wrong]}: Covered with hot, melted cheese, we ate the pizza.

\textbf{[Revised]}: We ate the pizza, which was covered in hot, melted cheese.
\end{quote}

\item \textbf{Mixed construction}

Mixed construction occurs when you start with one sentence structure and then shift to another.
\begin{quote}
\textbf{[Wrong]}: There is so much going on in the world today is why it is so hard to keep up with everything.

\textbf{[Revised]}: With so much going on in the world today, it is hard to keep up with everything.
\end{quote}

\item \textbf{Wrong preposition}

Many prepositions have developed a particular usage that will be expected by readers.

\begin{quote}
\textbf{[Wrong]}: The candidate compared his opponent with an orangutan.

\textbf{[Revised]}: The candidate compared his opponent to an orangutan.

\textbf{[Wrong]}: The tree was at the background of the photograph.

\textbf{[Revised]}: The tree was in the background of the photograph.

\end{quote}

\item \textbf{Shifts in tense, person, number, and mood}

Avoid unexpected changes in tense, person, number, and mood.

\emph{Tense}:
\begin{quote}
\textbf{[Wrong]}: She ran to the store and picks up some milk. 

\textbf{[Revised]}:  She ran to the store and picked up some milk.
\end{quote}

\emph{Person}:
\begin{quote}
\textbf{[Wrong]}:  When one visits Boston's Museum of Fine Arts, you get an education.

\textbf{[Revised]}:  When one visits Boston's Museum of Fine Arts, one gets an education.
\end{quote}

\emph{Number} (shifting between singular and plural):
\begin{quote}
\textbf{[Wrong]}: Because people make errors, he or she receives poor grades.

\textbf{[Revised]}:  Because people make errors, they receive poor grades.
\end{quote}

\emph{Mood} (indicative=make a statement/pose a question; imperative=request/command; subjunctive=making a wish):
\begin{quote}
\textbf{[Wrong]}: The teacher gave the students two guidelines: do not chew gum in class (imperative) and students should not be late for class (indicative).

\textbf{[Revised]}: The teacher gave the students two guidelines: do not chew gum in class and be on time.
\end{quote}

\item \textbf{Missing Commas}

\emph{After an introductory element}:
 \begin{quote}
\textbf{[Wrong]}: To tell the truth I never really liked the Yankees.

\textbf{[Revised]}: To tell the truth, I never really liked the Yankees.
\end{quote}

\emph{With a nonrestrictive element} (a part of the sentence not essential to the meaning of the sentence):

\begin{quote}
\textbf{[Wrong]}: Jeff who owned the corporation was a big gambler.

\textbf{[Revised]}: Jeff, who owned the corporation, was a big gambler.
\end{quote}

\emph{
In a series}:
\begin{quote}
\textbf{[Wrong]}: He bought eggs, milk, cheese and shampoo.

\textbf{[Revised]}: He bought eggs, milk, cheese, and shampoo. 
\end{quote}


\item \textbf{Unclear pronoun reference}
\begin{quote}
\textbf{[Wrong]}: The teacher gave her notes to her. (Whose notes are they?)

\textbf{[Revised]}: The teacher gave Jane's notes to her.
\end{quote}
\item \textbf{Misspellings [don't count on spell check]/Wrong words}
\begin{quote}
\textbf{[Wrong]}: He drank the bear.

\textbf{[Revised]}: He drank the beer.
\end{quote}
\end{enumerate}

%----------------------------------------------------------------------------------------
% Active vs. Passive Voice
%----------------------------------------------------------------------------------------

\section{\textcolor{ForestGreen}{Active vs. Passive Voice}}



In the active voice, the subject of the sentence is performing the action expressed in the verb; in the passive voice, the subject of the sentence is acted upon. 

While the passive voice is often a requirement in scientific or technical writing---where the focus is on a particular process or experiment---it should be avoided in most other forms of academic writing. As the following sentences illustrate, the active voice produces less wordy and more precise sentences:

\begin{quote}
 
\textbf{Passive voice}: My first visit to San Francisco will always be remembered by me.

\textbf{Active voice}: I will always remember my first visit to San Francisco.

\textbf{Passive voice}: 	An A was given to Johnnie by professor James.

\textbf{Active voice}: 	Professor James gave Johnnie an A.
\end{quote}

\begin{quote}
\textbf{Passive voice}: The take-home exam was cheated on by over 100 Harvard students.

\textbf{Active voice}: Over 100 Harvard students cheated on the take-home exam.

\textbf{Passive voice}: Mistakes were made.

\textbf{Active voice}: We made mistakes.

\end{quote}

%----------------------------------------------------------------------------------------
% Concision
%----------------------------------------------------------------------------------------
\section{\textcolor{ForestGreen}{Concision}}
 
\subsection{What is it?}
 
In his famous book, \emph{The Elements of Style}, William Strunk writes that a ``sentence should contain no unnecessary words, a paragraph no unnecessary sentences, for the same reason that a drawing should have no unnecessary lines and a machine no unnecessary parts'' (24). To be concise, eliminate wordiness and redundancy in your writing. This is a matter of style, not correctness. It requires that you compose your sentences by making choices about the selection and arrangement of words, rather than just writing them as they come out of your head.
 
\subsection{Why practice concision?}
 
Think rhetorically--when you write academically, professionally or artistically, you write for an audience. Be concise because you want to communicate your ideas clearly to your reader.
 
\subsection{How can I be more concise?}
 
 \begin{enumerate}
 
\item Use precision (see \emph{Precision}).
 
\item Avoid redundancy. \textbf{For example}:
 
\textbf{No need to write}: ``It's raining outside.''
[Where else would it rain?]
 
\textbf{No need to write}: ``I went to the ATM machine but forgot my PIN number.'' 
 [ATM= Automatic Teller Machine PIN= Personal Identification Number.]
 
\textbf{No need to write}: ``We should all collaborate together.'' 
[One cannot collaborate alone.]
 
 \item Eliminate unnecessary expletives at the beginning of sentences. (When grammarians refer to ``explicatives'' they mean: ``There is,'' ``There are,'' ``There was,'' ``There were,'' ``It is,'' ``It was''). \textbf{For example}:
 
  \textbf{Instead of}: ``It is my belief that all the research is wrong.''
 
                  	\textbf{Write}:		``I believe all the research is wrong'' \emph{or} ``All the research is wrong.''
 
 
             	\textbf{Instead of:}    	``There is another chapter that explains the theories.''
 
                    	\textbf{Write}:       	``Chapter 5 explains the theories.''
 
 
\item Cut down on ``to be'' verbs (is, are, was, were, being, been, be, am). \textbf{For example}:
 
      	\textbf{Instead of}:  	``Cheating would be a violation of my moral code.''
 
      	\textbf{Write}:          	``Cheating violates my moral code.''
 
 
\item Simplify sentence structure. \textbf{For example}:
 
    	\textbf{Instead of}: 	``We visited Central Park, which is where the museum is located.''
 
    	\textbf{Write}:          	``We visited Central Park, home to the museum.''
 
    	\textbf{Instead of}: ``She missed the lecture because of the fact that she was late.''
 
    	\textbf{Write}: ``Because she was late, she missed the lecture.''
 
    	\textbf{Instead of}:  	``The teacher demonstrated some of the various ways and methods
                               	for cutting words from my essay that I had written for class.''
 
    	\textbf{Write}:           	``The teacher demonstrated how to make my essay more concise.''
 
    	\textbf{Instead of:}  	``Due to the fact that it was late at night, it was not safe for him to
                                	be out.''
 
    	\textbf{Write}:                    	``It was unsafe for him to be out past dark.''
 
 
    	\textbf{Instead of:}  	``His white shirt, made of cotton, looked comfortable to me.''
 
    	\textbf{Write}:        	          	``His white, cotton shirt looked comfortable.''

\end{enumerate}

\section{\textcolor{ForestGreen}{Precision}}
 
\subsection{What is it?}
 
In his definitive work on style, William Strunk argues that ``the surest method of arousing and holding the attention of the reader is by being specific, definite, and concrete'' (22). In other words, your writing should be precise. To be precise, pick words that most clearly, accurately, and appropriately convey your ideas.  It is a good idea to pay attention to the language employed by various disciplines so that you may use the most appropriate vocabulary required. The idea is to communicate, as effectively as possible, exactly what you are thinking.
 
Being precise is a matter of style, not correctness. You must already have used the \emph{right} word before you can focus on choosing the most precise word.
 
\subsection{Why be precise?}
 
Again, think rhetorically--consider your audience; how can you best get them to understand exactly what you mean?
 
\subsection{How can I achieve precision?}
 
 \begin{enumerate}

\item Review your writing and look for words that seem vague or could be interpreted in many different ways or force your reader to do a lot of work to figure out, exactly, what you mean.
 
\item Use a dictionary or thesaurus (or the language tools on Word, which suggest synonyms)  to find a word that is more precise and appropriate, paying attention to the connotations of the words you choose.
 
\textbf{For example}:
 
 \begin{quote}
\textbf{Instead of}:``The author says that changing the drinking laws could be a problem.''
 
\textbf{Try}:	``The author warns that changing drinking laws increases rates of binge drinking.''
 
\textbf{Instead of:}	``The author did not come off very well at all in his article.''
 
        	\textbf{Try}:	``The author was inarticulate, illogical, and hostile.''
 
\textbf{Instead of}:	``The movie looked very nice with lots of bright colors and a bunch of
                    	changes happening.''
 
        	\textbf{Try}:   ``Beautifully saturated with reds and oranges, and employing frequent
                    	 attention-grabbing  wipes and jump cuts, \emph{The Isle} keeps the viewer entranced.''
 
\textbf{Instead of}:   ``The authors kind of say the same things. Sometimes they don't.''
 
        	\textbf{Try}:  ``John Smith believes apples are essential to a good diet, and Jane Brown concurs.
                   	However, they disagree what it comes to the number of apples one should       
                         consume, with Smith advocating eating one a day and Brown arguing for                         two.''

\end{quote}
\end{enumerate}

%----------------------------------------------------------------------------------------
% Parallel Structure
%----------------------------------------------------------------------------------------

\section{\textcolor{ForestGreen}{Parallel Structure}}

\subsection{What is it?}

Parallel structure is a way to construct your sentences so that related ideas and items in a series are presented in the same grammatical form.

\subsection{Why employ it?}

It helps indicate connections between your ideas, and thus improves the flow and coherence of your paper.

It can help you emphasize certain points and make them more memorable.

\subsection{Where should I use it?}

Use it in a thesis statement, because these often offer a pair or series of connected ideas/points.

It can be used anywhere in your paper where you want to draw a connection between a pair or series of ideas, or make these ideas particularly striking.

\textbf{For example}:

\begin{quote}
\textbf{Instead of}: ``Kissing can communicate a lot of different things.''

\textbf{ Try this}: ``A kiss can be a comma, a question mark, or an exclamation 
                   point.''--Mistinguett

\textbf{Instead of:} ``We shouldn't always just be thinking about ourselves and what we want from our
                  	government. Why don't we ever think about what we can be doing to help it''?

  \textbf{	Try this}: ``Ask not what your country can do for you, but what you can do for your
                               country.'' --JFK

\textbf{Instead of}: ``We went fishing and hiking. And then we went for a bike ride.''

\textbf{	Try this}: ``We went fishing, hiking, and biking.''

\textbf{Instead of}: ``Conversing with John Updike is kind of like it is when you go somewhere new:
                    proceed with caution.''

	\textbf{Try this}: ``To converse with John Updike is to enter uncharted territory: proceed with
                              caution.''

\textbf{Instead of}:  ``Writing well involves drafting, reviewing, and the need to edit your sentences.''
 
	\textbf{Try this}: ``Writing well involves drafting, reviewing, and editing.''

\textbf{Instead of}: ``She was a strong student and in leading other students she was good.''

\textbf{Try this}: ``She was not only a strong student but a good leader.''
 \end{quote}

%----------------------------------------------------------------------------------------
% Academic Research
%----------------------------------------------------------------------------------------

\section{\textcolor{ForestGreen}{Academic Research}}

Boston University's Mugar library maintains an impressive collection of books, articles, and electronic databases for academic research. Each of the electronic databases contain hundreds of individual journals, which, in turn, contain decades of academic research.

In addition to the instruction you receive from your professors, the Mugar library has created the following guides to aid you in the research process.

\subsection{Mugar library research guides}

\textbf{The research process}: evaluating sources, finding background information, accessing academic articles and books, using online databases: 

\url {http://www.bu.edu/library/guide/guidetoresearch/}

\textbf{Topic-based research guides}:

\url{http://www.bu.edu/library/research/guides/research-guides/}

\textbf{Other research-related how-to guides}:

\url{http://www.bu.edu/library/help/how-to/}

\subsection{Recommended electronic databases}


\textbf{JSTOR}: ``Provides page images of back issues of the core scholarly journals in the humanities, social sciences, and basic sciences from the earliest issues to within a few years of current publication. Users may browse by journal title or discipline, or may search the full-text or citations/abstracts.''

\textbf{MLA}: ``Indexes critical materials on literature, languages, linguistics, and folklore. Provides access to citations from worldwide publications, including periodicals, books, essay collections, working papers, proceedings, dissertations and bibliographies.''

\textbf{Project Muse}: ``Provides digital humanities and social sciences content for the scholarly community and is a source of complete, full-text versions of journals and monographs from many of the world's leading university presses and scholarly societies.''

\textbf{Lexis-Nexis}: ``Provides access to various databases, including current news, business information, company directories, federal and state laws, regulations, legal cases, etc..''

\textbf{Google Scholar}: ``Google Scholar provides a simple way to broadly search for scholarly literature. From one place, you can search across many disciplines and sources: articles, theses, books, abstracts and court opinions, from academic publishers, professional societies, online repositories, universities and other web sites. Google Scholar helps you find relevant work across the world of scholarly research"--Information screen (viewed Jan. 10, 2012).''

\textbf{America, History and Life}: ``Provides historical coverage of American history, including the United States and Canada from prehistory to the present. Includes information abstracted from over 2,000 journals published worldwide. Database coverage is 1964 to the present.''

\textbf{H.W. Wilson}: ``Humanities full text includes many of the most important academic journals in the humanities with the full text of articles from over 300 periodicals dating back to 1995, and high-quality indexing for almost 700 journals--of which 470 are peer-reviewed--dating as far back as 1984. The database provides coverage of feature articles, interviews, bibliographies, obituaries, and original works of fiction, drama, poetry and book reviews, as well as reviews of ballets, dance programs, motion pictures, musicals, radio and television programs, plays, operas, and more.''


\begin{quote}
[Descriptions taken from \url{http://www.bu.edu/library/research/collections/databases/} and viewed on August 22, 2012]
\end{quote}

%----------------------------------------------------------------------------------------
% Grading Rubric
%----------------------------------------------------------------------------------------

\section{\textcolor{ForestGreen}{Grading Rubric}}

\begin{quote}
The following grading rubric has been adopted from Winthrop University's Writing Program:


\url{http://www2.winthrop.edu/english/WritingProgram/rubric.htm}

\end{quote}

\subsection{The A paper}
The A paper is superior work that far exceeds the requirements of the assignment. This paper tackles the topic in an innovative way, with an appropriate sense of audience and an effective plan of organization. The writer has clearly used the elements of reasoning to generate material for developing as well as structuring the paper. When evaluated, the paper meets the highest standards of critical thinking. For example, the point of view and purpose are clear. It outlines the question at issue and assumptions with precision. Its information is accurate, and its concepts are relevant. It reaches sufficient conclusions and interpretations. Its implications and consequences are deep and broad. All elements can be judged against the standards very favorably. When the writer uses source materials in the A paper, he/she demonstrates complete understanding of potential critical thinking impediments and also illustrates an insightful understanding of the elements of reasoning present in the source. The writer reveals an understanding of the stated and unstated concepts and/or assumptions that are an essential aspect of alternative perspectives and shows a sophisticated analysis of the information and conclusions used to support the source’s main argument. The writing style is energetic and precise; how the writer says things is as excellent as what the writer says. There is evidence of careful editing, since the paper contains few grammatical and/or mechanical errors. Research materials, if used, are correctly documented \dots.

\subsection{The B paper}

The B paper is above-average work that more than meets the requirements of the assignment. It has a clear sense of topic, audience, and purpose, with sound organization, good evidence, and good analysis and/or argument. As in the A paper, this writer has clearly used elements of reasoning to generate material for developing the paper, although not to the extent of the superior A paper. Similarly, all elements can be judged against the standards favorably. When the writer uses source materials, he/she demonstrates complete understanding of potential critical thinking impediments, but illustrates a less sophisticated understanding of the elements of reasoning present in the source. The writer reveals an understanding of the source’s assumptions and conclusions, but falls somewhat short in recognizing the stated and unstated concepts and/or assumptions of the argument. The writing style is clear and precise, and the paper shows evidence of careful editing, since the essay contains few grammatical and/or mechanical errors (although an otherwise superior paper can have errors that lower the overall grade). Research materials, if used, are correctly documented \dots.
\subsection{The C paper}
The C paper is adequate, average work that meets the requirements of the assignment. The paper has a clear sense of topic, audience, and purpose, with a generally sound organization, although problems with focus may exist. The evidence is adequate, but not as specific as in an A or B paper, and the analysis and/or argument is not as fully developed. This paper deals with the elements of reasoning as outlined above, although not to the extent of the A and/or B paper, and some elements may not be fully addressed. While the paper minimally meets the standards of critical thinking, it may not address some of these standards fully. When the writer uses source materials, he/she demonstrates a basic understanding of potential critical thinking impediments, but reveals only a rudimentary understanding of the elements of reasoning present in the source. The writing style is clear, although there may be problems with sentence construction or word choice. Though the writer has edited the paper, remaining errors may affect the ability to communicate. Research materials, if used, are correctly documented \dots.

\subsection{The D paper}

The D paper is below average work that demonstrates an attempt to fulfill the assignment and shows some promise, but does not meet the requirements of the assignment. The paper may have one or more of the following weaknesses. There may be problems with the sense of topic, audience, or purpose. The paper may not adequately address the elements of reasoning and the standards of critical thinking as outlined above. It may have a general or implied thesis, but the idea may be too broad, vague, or obvious. The organizational plan may be inappropriate or inconsistently carried out. Evidence may be too general, missing, irrelevant to the thesis, or inappropriately repetitive. The analysis and/or argument may be underdeveloped. The style may be compromised by repetitive or flawed sentence patterns and/or inappropriate word choice and confusing syntax. Grammatical and mechanical errors may interfere with readability and indicate a less-than-adequate attempt at editing or an unfamiliarity with some aspects of Standard Written English. Research materials, if used, may not be completely or accurately documented \dots.

\subsection{The F paper}

The F paper is unacceptable work that does not meet the requirements of the assignment. It exhibits one or more of the following weaknesses. It may not meet the purpose of the assignment. It may be an attempt to meet the requirements of the assignment, but have no apparent thesis or a self-contradictory one, or its point may be general or obvious and suggest no serious engagement with the topic. The paper fails to address the elements of reasoning and the standards of critical thinking as outlined above. It may display little or no apparent sense of organization; it may lack development; evidence may be inadequate, inappropriate, or may consist of generalizations, faulty assumptions, or errors of fact. The style suggests serious difficulties with fluency, which may be revealed in short, simple sentences and ineffective word choice. Grammatical/mechanical errors may interfere with reader comprehension or indicate problems with basic literacy or a lack of understanding of Standard English Usage. Research materials, if used, may not be handled responsibly and/or documented appropriately \dots.





\end{document}
