
%----------------------------------------------------------------------------------------
%Conclusions
%----------------------------------------------------------------------------------------


\chapter{Conclusions} 
\section{Why is the conclusion important?}

\begin{itemize}
\item It is the place where you draw connections between all your points and where you emphasize the most significant material covered.
 
\item On a practical note, it may be the last thing your professor reads before assigning a grade to the paper, so end on a strong note. Your conclusion should not read like an afterthought or an attempt to meet a minimum page requirement. 
\end{itemize}

\section{How do I write a strong conclusion?}
        	
\begin{itemize}

\item Provide a transition from the body of your paper (see \emph{Transitions}).
 
\item Emphasize the main points of your paper, drawing connections between them as appropriate to create coherence (see: \emph{Transitions}).
 
\item End with a strong final sentence (see advice above regarding the opening sentence)

\end{itemize}

\section{What should I avoid?}

\begin{itemize}        	
\item \textbf{Avoid tacking on an inappropriately happy ending}. 

For instance, if you have just written a whole paper explaining how serious the drug problem in the U.S. is, it is not appropriate to end with something  upbeat and off-topic.
 
\begin{quote} \textbf{For example}: ``Everything usually works out in the end, and we can figure this out.''
 
\end{quote}
 
\item \textbf{Do not end a persuasive piece by including a feel-good and inaccurate statement that undermines your entire argument.}

\begin{quote}
\textbf{For example}:
        	``But that's just my opinion, and everyone is entitled to their own opinion.''
 \end{quote}
        	
A personal opinion is different than an academic opinion. If you have made a claim, backed it, and anticipated reasonable opposing arguments with the intent to persuade, this is quite different than just expressing your beliefs in the interest of sharing, not   persuading. The sample sentence above refers to the latter type of opinion and thus would be an inappropriate way to end an academic argument.
 
\item \textbf{Avoid merely rephrasing your thesis statement}.

Since you have already introduced your points in the introduction and developed them in the body of your paper, you should use the conclusion to word these main points in such a way that they are particularly striking and memorable. The point is to emphasize them as you conclude. You might consider using figures of speech such as anaphora, alliteration, polysyndeton, etc.. For a comprehensive list and description of rhetorical figures, see the following website:

\url{http://rhetoric.byu.edu/}
 
 
\item \textbf{Do not introduce a totally new idea or go off topic}.
 
\item \textbf{Avoid merely listing a bunch of points from your paper}.

You should be able to discern which points are the most significant. Emphasize these.
 
\item \textbf{Do not tack on material just to meet a length requirement}.

If your paper needs more material, go back to your body paragraphs and see where you can or should expand on a point or add necessary points.

\end{itemize}

%----------------------------------------------------------------------------------------
% END OF SECTION
%----------------------------------------------------------------------------------------

