%----------------------------------------------------------------------------------------
% Parallel Structure
%----------------------------------------------------------------------------------------
\section{\textcolor{ForestGreen}{Parallel Structure}}

\subsection{What is it?}

Parallel structure is a way to construct your sentences so that related ideas and items in a series are presented in the same grammatical form.

\subsection{Why employ it?}

It helps indicate connections between your ideas, and thus improves the flow and coherence of your paper.

It can help you emphasize certain points and make them more memorable.

\subsection{Where should I use it?}

Use it in a thesis statement, because these often offer a pair or series of connected ideas/points.

It can be used anywhere in your paper where you want to draw a connection between a pair or series of ideas, or make these ideas particularly striking.

\textbf{For example}:

\begin{quote}
\textbf{Instead of}: ``Kissing can communicate a lot of different things.''

\textbf{ Try this}: ``A kiss can be a comma, a question mark, or an exclamation 
                   point.''--Mistinguett

\textbf{Instead of:} ``We shouldn't always just be thinking about ourselves and what we want from our
                  	government. Why don't we ever think about what we can be doing to help it''?

  \textbf{	Try this}: ``Ask not what your country can do for you, but what you can do for your
                               country.'' --JFK

\textbf{Instead of}: ``We went fishing and hiking. And then we went for a bike ride.''

\textbf{	Try this}: ``We went fishing, hiking, and biking.''

\textbf{Instead of}: ``Conversing with John Updike is kind of like it is when you go somewhere new:
                    proceed with caution.''

	\textbf{Try this}: ``To converse with John Updike is to enter uncharted territory: proceed with
                              caution.''

\textbf{Instead of}:  ``Writing well involves drafting, reviewing, and the need to edit your sentences.''
 
	\textbf{Try this}: ``Writing well involves drafting, reviewing, and editing.''

\textbf{Instead of}: ``She was a strong student and in leading other students she was good.''

\textbf{Try this}: ``She was not only a strong student but a good leader.''
 \end{quote}
 
 %----------------------------------------------------------------------------------------
% END OF SECTION
%----------------------------------------------------------------------------------------