
%-----------------------------------
%ANNOTATING READINGS
%-----------------------------------


\chapter{Annotating Texts}

\begin{quote}
\small
"We believe the best way to work on a difficult text is by rereading \dots but you can also work on the difficult text by writing, by taking possession of the work through sentences and paragraphs of your own, through summary, paraphrase, and quotation, by making another writer’s work part of your work." (12)

--Bartholomae and Petrosky, "Introduction." \emph{Ways of Reading: An Anthology for Writers
Critical Reading}

\end{quote}

Analysis requires breaking an argument into smaller parts so that you can understand how those parts work together to make the whole (or fail to do so). The best way to begin this process is to write while you read in the form of underlining and the use of marginal notes and a system of symbols (made on the pages of the document itself). There is no right or wrong way to mark up a text, but you should develop a system that you are comfortable with and try to stick with it. Writing while you read will help you stay focused and read critically. In fact, I would argue that if you are not writing while you read — not “taking possession” of a text by putting it into your own words through annotation, summary, paraphrase, and quotation — then you are not really reading at all.

\subsection{Reading to reduce}

Your objective in annotation is to provide a plan for reducing a long, unwieldy document into something small and useful for study. I commonly underline the thesis once I find it. And, as I read, I place a large dot next to pieces of evidence or statements/assertions that are being used to support the thesis. I always place some kind of keyword or short statement next to the main argumentative moves that the essay makes. I further put a check mark or exclamation point next to statements that I find important or noteworthy. Sometimes I draw arrows to connect parts of the essay. In addition to marking up a text for later use, I ask questions in the margins or note places where I become confused. This is helpful later, on your second reading, since you can pay more careful attention to the passages that gave you trouble. I also write my thoughts as they occur to me and state objections to things which seem problematic. Sometimes I try to connect an idea in one text with the idea(s) in another (I may find places where I can compare/contrast the thinking of two writers). As you can see, the process of annotation keeps me engaged, active, and alert--key components in critical thinking. 

Of course, this is only the system that I developed. Feel free to use whatever symbols or markings that you prefer. However, it is absolutely critical that you develop some method of marking up a text to aid in the critical analysis of the challenging, dense, readings that you will encounter in college and throughout your life as a citizen of the world.

\subsection{Reduction: making a critical outline}

After I go through this process of marking up a text, I usually have a pretty solid understanding of the author's argument and my response to it. If this is an important essay or if I plan to use it in my own writing, I will put it through a process of reduction, where I try to trim the argument down to its bare essentials. With any luck, your annotations will guide your efforts to reduce the argument. On a separate piece of paper or a computer document I create a critical outline that aims to reduce the entire argument to its main idea and brief statements of paraphrase, summary, and significant quotations. I carefully note the page numbers where these summaries, paraphrases and quotations are taken from for future use in my own writing.

Of course, as you write this critical outline you will not only try to reduce the main points of the argument, you will also ask questions and make observations of the text. You should note the argumentative points that you find yourself strongly agreeing or disagreeing with and your reasons for doing so. You might see a logical inconsistency or want the author to provide more evidence for his or her claims. You might make a note to perform some research at the library or on the Internet on an unfamiliar concept or event mentioned in the argument. Ultimately, however, you will want to determine if the argument you have read is persuasive and provide the reasons why.

At the end of this process, you should have a simplified---but objective and accurate---version of the essay that has been ruthlessly cut down to its bare essentials as well as a number of critical observations, questions, and ideas that have emerged in your process of reading. By the time you reach this stage and read over your notes, you will have taken great strides toward mastering the argument. Of course, if the essay is difficult, you may have to repeat the process until you have a breakthrough. I cannot emphasize enough how helpful this process is. It will help you come to a greater understanding of the text’s claims and weaknesses while also activating your long-term memory.

