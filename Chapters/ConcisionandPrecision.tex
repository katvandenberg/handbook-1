%----------------------------------------------------------------------------------------
% Concision and Precision
%----------------------------------------------------------------------------------------
\chapter{Concision}
 
\section{What is it?}
 
In his famous book, \emph{The Elements of Style}, William Strunk writes that a "sentence 
should contain no unnecessary words, a paragraph no unnecessary sentences, for the 
same reason that a drawing should have no unnecessary lines and a machine no 
unnecessary parts" (24). To be concise, eliminate wordiness and redundancy in your 
writing. This is a matter of style, not correctness. It requires that you compose your 
sentences by making choices about the selection and arrangement of words, rather 
than just writing them as they come out of your head.
 
\section{Why practice concision?}
 
Think rhetorically--when you write academically, professionally or artistically, you write 
for an audience. Be concise because you want to communicate your ideas clearly to 
your reader.
 
\section{How can I be more concise?}
 
 \begin{enumerate}
 
\item Use precision (see \emph{Precision}).
 
\item Avoid redundancy. 
 
\textbf{No need to write}: "It's raining outside."
[Where else would it rain?]
 
\textbf{No need to write}: "I went to the ATM machine but forgot my PIN number." 
 [ATM= Automatic Teller Machine PIN= Personal Identification Number.]
 
\textbf{No need to write}: "We should all collaborate together." 
[One cannot collaborate alone.]
 
 \item Eliminate unnecessary expletives at the beginning of sentences. (When 
grammarians refer to "explicatives" they mean: "There is," "There are," "There was," 
"There were," "It is," "It was"). 
 
  \textbf{Instead of}: "It is my belief that all the research is wrong."
 
                  	\textbf{Write}:"I believe all the research is wrong" \emph{or} 
"All the research is wrong."
 
 
             	\textbf{Instead of:}    	"There is another chapter that explains the theories."
 
                    	\textbf{Write}:       	"Chapter 5 explains the theories."
 
 
\item Cut down on "to be" verbs (is, are, was, were, being, been, be, am).
 
      	\textbf{Instead of}:  	"Cheating would be a violation of my moral code."
 
      	\textbf{Write}:          	"Cheating violates my moral code."
 
 
\item Simplify sentence structure.
 
    	\textbf{Instead of}: 	"We visited Central Park, which is where the museum is located."
 
    	\textbf{Write}:          	"We visited Central Park, home to the museum."
 
    	\textbf{Instead of}: "She missed the lecture because of the fact that she was late."
 
    	\textbf{Write}: "Because she was late, she missed the lecture."
 
    	\textbf{Instead of}:  	"The teacher demonstrated some of the various ways and methods
                               	for cutting words from my essay that I had written for class."
 
    	\textbf{Write}:           	"The teacher demonstrated how to make my essay more concise."
 
    	\textbf{Instead of:}  	"Due to the fact that it was late at night, it was not safe for him to
                                	be out."
 
    	\textbf{Write}:                    	"It was unsafe for him to be out past dark."
 
 
    	\textbf{Instead of:}  	"His white shirt, made of cotton, looked comfortable to me."
 
    	\textbf{Write}:        	          	"His white, cotton shirt looked comfortable."

\end{enumerate}

\chapter{Precision}
 
\section{What is it?}
 
In his definitive work on style, William Strunk argues that "the surest method of 
arousing and holding the attention of the reader is by being specific, definite, and 
concrete" (22). In other words, your writing should be precise. To be precise, pick 
words that most clearly and accurately convey your ideas.  It is a good idea to pay 
attention to the language employed by various disciplines so that you may use the 
most appropriate vocabulary required. The idea is to communicate, as effectively as 
possible, exactly what you are thinking.
 
Being precise is a matter of style, not correctness. You must already have used the 
\emph{right} word before you can focus on choosing the most precise word.
 
\section{Why be precise?}
 
Again, think rhetorically--consider your audience; how can you best get them to 
understand exactly what you mean?
 
\section{How can I achieve precision?}
 
 \begin{enumerate}

\item Review your writing and look for words that seem vague or could be interpreted 
in many different ways or force your reader to do a lot of work to figure out, exactly, 
what you mean.
 
\item Use a dictionary or thesaurus (or the language tools on Word, which suggest 
synonyms)  to find a word that is more precise and appropriate, paying attention to 
the connotations of the words you choose.
 
 

\textbf{Instead of}:"The author says that changing the drinking laws could be a 
problem."
 
\textbf{Try}:	"The author warns that changing drinking laws increases rates of 
binge drinking."
 
\textbf{Instead of:}	"The author did not come off very well at all in his article."
 
        	\textbf{Try}:	"The author was inarticulate, illogical, and hostile."
 
\textbf{Instead of}:	"The movie looked very nice with lots of bright colors and a 
bunch of changes happening."
 
        	\textbf{Try}:   "Beautifully saturated with reds and oranges, and employing 
frequent attention-grabbing  wipes and jump cuts, \emph{The Isle} keeps the 
viewer entranced."
 
\textbf{Instead of}:   "The authors kind of say the same things. Sometimes they don't."
 
\textbf{Try}:  "John Smith believes apples are essential to a good diet, and Jane Brown concurs. However, they disagree what it comes to the number of apples one should consume, with Smith advocating eating one a day and Brown arguing for two."


\end{enumerate}

%----------------------------------------------------------------------------------------
% END OF SECTION
%----------------------------------------------------------------------------------------

