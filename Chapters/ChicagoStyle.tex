%----------------------------------------------------------------------------------------
%	THE CHICAGO STYLE
%----------------------------------------------------------------------------------------

\chapter{Chicago style}

\section{Formatting the Chicago essay}

When setting up your word processor for a Chicago-formatted document, use the 
following settings and rules:

\begin{itemize}

\item Use 1" margins on all sides of the document.
\item Place your last name and page number on the right side of each page in the 
document's header.
\item Double-space the entire document, including endnotes and bibliography.
\item Block quotes are formed when a quote runs more than 100 words. Indent the entire 
block of text with a 0.5" tab from the left margin.
\item Endnotes and bibliographic entries are double-spaced with a blank line separating 
them.
\item Indent the first line of a note entry with a .5" tab.
\item The Chicago form requires a title page. The title of the essay is centered about 
1/3 down the page. Place your name at the center of the document. Near the bottom 
of the page place your course information and date on three separate, centered lines 
about 1/3 up from the bottom of the page.

\textbf{*}\textbf{Note}: the title page is \emph{counted but not numbered}. Therefore, 
begin your actual essay with page 2.
\end{itemize}

Models of the title page and first page of the Chicago-formatted essay may be found on 
the following pages:

%----------CHICAGO TITLE PAGE EXAMPLE
\newpage
\thispagestyle{empty}
\begin{Spacing}{2}
\vspace* {3cm}
\begin{center}The Great Irish Potato Famine\end{center}
\vspace {4cm}
\begin{center}Lauren Roche\end{center}
\vspace {5cm}
\begin{center}Rhetoric 102\\
Professor Taylor\\
January 22, 2012\end{center}
\end{Spacing}
\newpage

%First Page of Chicago Essay


\thispagestyle{empty}
\begin{flushright}Roche 2\end{flushright}
\bigskip
\begin{Spacing}{1.5}

\hspace{.4in}The Great Irish Potato Famine, also known as the potato blight, was one of the most devastating and historically altering events to take place during the nineteenth century. Lasting between 1845-52 the famine killed one million men, women, and children in Ireland, and caused over one million to emigrate from the country.\textsuperscript{1}  Years of English domination and oppression left the people of Ireland poor and greatly dependent upon the potato as their main source of nutrients and survival. When disaster struck in 1845 and the potato plants unexpectedly became inedible, a deadly crisis of hunger, struggle, and destitution began.\textsuperscript{2}  The famine’s influence left a lasting impact not only on Ireland’s history, but also on the history of the world, especially in England and North America.

\hspace{.4in}The relationship between Ireland and England was notably strained even before the famine occurred. As reflected upon by Don Nardo, Pope Adrian IV granted England ownership over Ireland in 1155. As a result of this event, King Henry II began to impose high taxes upon the poor people of Ireland. This was just the starting point for what was to become centuries of resentment over English domination. Another turning point occurred in the 1500s when King Henry VIII separated from the Roman Catholic Church. All lands under the King’s rule were ordered to join the Church of England, proving to be a major issue for the Catholics of Ireland. The King responded to the objections of the Irish by enforcing “penal laws” which placed numerous restrictions upon Catholics.\textsuperscript{3}  Cecil Woodham- Smith reflects upon these restrictions saying, “The penal laws brought lawlessness, dissimulation, and revenge in their train, and the Irish character, above all the character of the peasantry, did become degraded and debased.”\textsuperscript{4}  Catholics were not allowed to vote, hold office, or purchase land, in addition to other penalties. Catholics were forbidden from schools, making education nearly impossible. Anything related to practicing the Catholic faith was banished, leading to priests being hunted and informants encouraged. Tension between the two countries grew fervently.\textsuperscript{5} 

\newpage
\thispagestyle{empty}
\begin{flushright}Roche 3\end{flushright}

\hspace{.4in}During the mid-nineteenth century, Ireland was well known as one of the poorest countries within the western world. Over two-thirds of Ireland’s eight million-person population subsided in the countryside, where living conditions were harsh and menial.\textsuperscript{6} According to Noel Kissane, at this time England had control over all of Ireland, and confiscated the majority of its land through centuries of English policies of conquest, confiscation, and plantation. The majority of Irish property was run under a landlord system in which the head landlord, who typically resided in England, would lease his estate to local Protestant landlords known as middlemen. Kissane comments upon these absentee landlords saying, “If we add that he cared nothing as well as knew nothing, we shall not be far from the truth.”\textsuperscript{7}  Through this method, the absentee head-landlord would collect rent from his middleman who would in turn rent portions of the land to poor Catholic farmers. Peasants known as cottiers were allowed to live on the farmer’s land in exchange for their labor, which would help to pay the rent. These farmers were not allowed to lease the land, and as a result they did not cultivate it. Rather, they only produced food for their own households and for a pig, which would be used to pay the rent. Potatoes were generally the only food produced on the land and became the staple diet for the majority of people. Therefore, a large percentage of the Irish population was completely dependent upon the potato harvest alone as their means of survival.\textsuperscript{8} 

\hspace{.4in}Originating in the Americas, the potato made its way to Ireland where several varieties of potatoes, known as “Irish potatoes,” began evolving as a reaction to the different soil and climate conditions there. One variety of potato that was especially common was the “lumper” which was well known for its ability to produce a good crop even in poor soils.\textsuperscript{9}  According to Eileen Moore Quinn, over 75\% of the Irish population was completely dependent upon the lumper alone when the famine hit Ireland. The other food sources available in Ireland were for export and thus unavailable for the Irish people themselves.\textsuperscript{10}  The potato proved itself to be an extremely beneficial and popular crop for the Irish for several reasons. For instance, farmers found it to be a very convenient for their circumstances because it was a 

\newpage
\thispagestyle{empty}
\begin{flushright}Roche 4\end{flushright}

high yield crop and could therefore produce a large amount of food in a small area. Also, because potatoes grow underground, they were safe from the rebellions and warfare that frequently took place. One of the most crucial advances of the potato for the poor Irish farmers was that it required very little skill and money to produce. Furthermore, potatoes contain traces of almost all of the elements required for a healthy diet, providing the majority of proteins, calories, and minerals consumed by the Irish. Over time, the potato replaced the less nutritious option of grain as the main source of food for the poor.\textsuperscript{11} 

\hspace{.4in}As Cormac O’ Grada writes, the surprising arrival of potato blight was first noted in August 1845. Primarily beginning in the US, potato blight arrived in Ireland after crossing the ocean through an unknown route. A variety of theories were suggested as explanations for the mysterious disease. Ironically, the correct diagnosis, which was proposed by Rev. M.J. Berkley, was pushed aside by most experts. Berkley identified the mold on the potatoes as a fungus that fed off of healthy potatoes.\textsuperscript{12} Another explanation for the disease was proposed by Dr. John Lindley, who suggested that the blight was the result of too much rain which had caused the potatoes to absorb an over abundance of water. However, neither these explanations nor any others were accepted as the disease continued to mysteriously spread and destroy the Irish potato harvest.\textsuperscript{13}  As the blight carried on, people began going days, weeks, and eventually months without food and people began starving to death. \textsuperscript{14} 

%\hspace{.4in}Panic and surprise erupted as the Irish farmers began to realize the gruesome and destructive outcomes of the disease. Peasants discovered “wrecked crop fields, withering and slimy potato stocks, the tubers beneath blotched and ulcerated, their sudden decomposition producing a strong stench”. [15] Desperate attempts to cure and preserve the potatoes were made by the helpless farmers. Unfortunately their efforts were made in vain as potato blight can only be prevented and not cured. As the digging of the crop began in October 1845, calls were made for the government to become involved and to assist in ending the crisis. Prime Minister Peel of the Conservative party believed that in order to relieve the hunger of the large number of people it would be necessary to import cheap corn. However, such an action would require England’s Corn Laws, which placed high tariffs on imported corn, to be repealed. British farmers and consumers strongly opposed this notion for fear that British farmers would go bankrupt and the economy would collapse if cheap foreign goods were permitted into the country.[16] Despite this resistance, Peel made the decision to secretly buy and import a huge quantity of Indian corn from the United States. Peel used this transaction as a way around the Corn Laws because Britain had no trade in Indian corn, and thus would not be affected by the Corn Laws. The amount of corn purchased was enough to feed five hundred thousand people for three months.[17]

%\hspace{.4in}In 1946, Peel began to institute a number of public works projects for the Irish. The Irish Board of Works was able to put over one hundred thousand people to work by summer. The work was extremely difficult, and laborers would often need to walk many miles with empty stomachs to reach work sites. They would work many hours without any food, and when they became unable to work, their children and wives would take over for them. Wages were extremely low, and families were unable to support themselves due to rising food prices. In some instances, workers would go weeks without receiving their pay. Meanwhile, Peel continued his work to end the Corn Laws in order to stabilize food prices. By June the Corn Laws were indeed repealed; however, Peel’s triumph caused him to lose immense support from the Conservative party, and he resigned from his position as Prime Minister.[18]

%\hspace{.4in}The spring of 1846 brought new hope to the people of Ireland, as most people did not believe that the blight had survived the winter and could affect the new crop.[19] Lord John Russell now replaced Peel as the Prime Minister. Russell believed that the people of Ireland should be self-sufficient, and thus ended the importation of Indian corn, leaving the Irish people with far less to eat during the summer time known as the “hungry months.”[20] Sir Charles Trevelyan, head of the British treasury, greatly influenced Russell’s actions. Trevelyan was well known for his belief that “God had sent the Famine to teach the Irish people a lesson, a lesson that would result in a new and improved Ireland”.[21] He closed the public work projects during mid-August, the time to harvest the early potatoes. His actions were made prematurely though, because the blight stuck again worse than before. One of the cruelest ironies of this time period was that although the Irish people were starving because of the loss of potatoes, the land was still producing large amounts of grain. Yet the laborers could not eat the grain because it was the property of the farmers and landlords. The majority of this grain was exported to England, leaving the Irish behind to starve. [22]

\hspace{.4in}Noel Kissane discusses the diseases fostered during the famine, which were as much the cause of its high levels of death rates as starvation. Because sanitation and hygiene conditions at this time were very poor, infectious diseases spread rapidly throughout the land. The most common famine diseases were typhus and relapsing fever. Also, stomach disorders became very common because the poor would eat the diseased potatoes, causing violent vomiting and diarrhea. In addition, many were afflicted with scurvy because they were no longer consuming the rich amounts of vitamin C that was present in the potato crop. One eyewitness to the famine 

\newpage
\thispagestyle{empty}
\begin{flushright}Roche 5\end{flushright}

described the piteous scene he witnessed:

\begin{quote}I started from Cork, by the mail, for Skibbereen and saw little until we came to Clonakilty, where the coach stopped for breakfast; and here, for the first time, the horrors of the poverty became visible, in the vast number of famished poor, who flocked around the coach to beg alms: amongst them was a woman carrying in her arms the corpse of a fine child, and making the most distressing appeal to the passengers for aid to enable her to purchase a coffin and bury her dear little baby. This horrible spectacle induced me to make some inquiry about her, when I learned from the people of the hotel that each day brings dozens of such applicants into the town.\textsuperscript{15} \end{quote}

Disease escalated into epidemics in several areas, especially in west and southwest Ireland, where conditions were more congested. The Central Board of Health which was established in 1846, served to provide facilities for the sick to receive care. Between 1847 and 1850, over half a million people are recorded as receiving treatment in fever hospitals. Most of the people who developed a disease during this time were not hospitalized though, and therefore were never recorded. \textsuperscript{16} 

%\hspace{.4in}Many landlords in Ireland began to acquire a lot of debt and became unable to pay their rates as more and more poor farmers and laborers were unable to pay their rent. At this time of financial crisis, the British government demanded that rates be collected by any means necessary, even by force.[24]This announcement led to the eviction of numerous families from their homes.[25] Evictions mainly took place for two reasons. First, non-payment of rent, including rents that were not paid at all and rents that were partially or irregularly paid was grounds for eviction. Another cause for many evictions were cases when landlords were forced to pay the full poor rate after the exemption from the poor rated granted under the Poor Law legislation to many tenants.[26] In some cases, landlords would threaten other tenants to not take in evicted families. However, there were rare situations in which landlords acted in what is known as “landlord-assisted emigration.” In these instances, landlords would pay for the passage of their evicted tenants to emigrate to the United States or Canada.[27] Some tenants gave up their homes without any fight, but others sought retaliation against the landlords.[28]

%\hspace{.4in}During the famine years, Irish emigration to other countries increased dramatically. For example, during the winter of 1847, more than 215,000 people left in search of new lands from Irish ports. The journey from their homeland was not an easy task for the Irish by any means. One example that illustrates this fact is that the vessels, which carried Irish emigrants, were known as coffin ships. These boats were extremely dangerous, for they would often capsize due to complications caused by overcrowding. These ships also did not carry nearly enough food or water to sustain the large amount of passengers on board. Many did not contain any sort of sanitary convenience, or even so much as a bathroom.[29] One passenger who traveled on a coffin ship, the Elizabeth and Sarah, described the experience as “ horrible and disgusting beyond the power to describe.” [30].Five thousand Irish is the estimated number of people to have died during the process of emigration at this time. As noted by Cecil Woodham- Smith, regardless of these horribly unimaginable circumstances, “The Irish peasant’s wild desire to escape from Ireland, combined with his utter ignorance of the sea and of geography, made him eager to risk himself in any vessel.” [31]

%\hspace{.4in}Unfortunately, these passengers were unaware that their struggles would not end completely once they reached their destinations. Nardo, observes the conditions that welcomed the Irish once they arrived in countries such as the United States and Canada saying, “Unfortunately, these emigrants faced the same prejudice and hatred in their new country that they experienced in Ireland.”[32] A major fear amongst many countries was that these newcomers would bring along diseases with them. For instance, it was requirement in Quebec for any ship coming from the St. Lawrence to stop at Grosse Isle where there was a quarantine station so that passengers on the ship could receive medical inspection. Canadian citizens were greatly displeased with increasing numbers of Irish who were arriving at Canadian ports.[33] Overcrowding at the quarantines became a serious issue, as many sick passengers were forced to wait on their ship for weeks, causing death rates to increase dramatically. Due to these conditions, authorities abandoned quarantine stations and immigrants began to enter the country without being checked. Indeed many Canadians’ worst fear was confirmed as illnesses spread throughout many cities and towns as a result.[34]

%\hspace{.4in}A large number of immigrants sought refuge from the harsh conditions of Ireland in the United States. In fact, around 75 percent of people who emigrated from Ireland during the Famine went to the United States.[35] Although there was once a great national feeling of sympathy for the poor and starving Irish, once they began to arrive in America sympathetic attitudes quickly changed. Bartoletti states that many Americans had fears regarding the spread of disease similar to the Canadians. In addition, many employers in America would not hire the Irish for work because it was a common fear that the new immigrants would take over all of the American jobs and bring down the price of wages. Signs that read “NINA” which meant, “no Irish need apply” were placed in the windows of businesses and in newspapers.[36] With the coming of such great influxes of Irish-Catholics into American cities such as New York, Philadelphia, and Boston, anti-Catholic feelings of discrimination and false rumors surrounding Catholics began to spread. A common fear amongst many Americans was that Irish Catholics would dominate the country over time and choose to take orders from the Catholic Pope in Rome as opposed the president of the United States.[37]As a result, anti-Irish riots became a common occurrence in the U.S. at this time. As noted by Nardo, over time anti-Irish feelings in America began to go away as the Irish adapted into American society and proved themselves to be hard workers.[38] Despite the obstacles faced by the Irish who moved to America, many were able to move their way out of the slums where they were living and improve their social standings.

\hspace{.4in}The challenging consequences created by the Great Famine had no immediate ending. The blight reoccurred several times after 1850, however the results were never as devastating as during the years of the famine because the people were no longer completely dependent upon the crop.\textsuperscript{17}One of the most noteworthy effects of the famine was that the population or Ireland decreased significantly. Around one million deaths is the estimation for lives lost as an immediate result of the famine. The rural poor of the countryside make up the majority of this statistic. Even more people were lost from the country as the result of emigration. Between the years of 1845 to 1851, around 1.2 million people left the country, and numbers continued to rise after the Famine years were over. Today the Irish diaspora is nearly ten times 

\newpage
\thispagestyle{empty}
\begin{flushright}Roche 6\end{flushright}

greater than the population of Ireland itself, with forty million people from the United States alone claiming to have some Irish ancestry.\textsuperscript{18} In addition, Dean Braa notes that the Irish tradition of land subdivision ended due to the Irish peasantry’s strategy to commercialize and consolidate after the famine. In their attempts to minimalize land subdivision, peasants adopted a custom of people getting married at later ages than before, thus “consolidating and minimalizing of dependents on a given holding.”\textsuperscript{19} 

%\hspace{.4in}The debate over whether or not the Great Irish Famine should be considered as an instance of genocide is a longstanding and controversial issue. Although the British did not cause the potato blight to occur, the response of the British Government toward the Irish is frequently considered as a destructive act toward a group of people. Cecil Woodham-Smith notes that Oliver Cromwell’s desire to “extirpate” the Irish is often compared to Adolf Hitler’s attempts to exterminate the Jewish population.[42] England’s failure to enforce any measures of reconstruction at any point after the years of the famine is often suggested as proof for their desire for Ireland to fail. Moreover, a series of harsh laws and policies instituted by the British government are also used as evidence of genocide.[43] In contrast, many people suggest that it was not England’s desire to destroy the Irish that caused the government’s poor handling of the crisis, but rather, “obtuseness, short-sightedness, and ignorance probably contributed more.” [44] The majority of historians today would never suggest that the English were responsible for causing the famine because the famine itself was truly a series of unfortunate natural occurrences . However, as Colm Toibin and Diarmaid Ferriter observe, “The suggestion is rather that, impelled by their contempt for Ireland and their interest in land reform, the administration caused many people to die”.[45] Whether or not the treatment of the Irish people during the famine should be categorized as genocide, truly depends on one’s opinions of England’s motives.

\hspace{.4in}The famine transformed Ireland as a nation, and altered its history completely. Leaving behind a bitter resentment for England, the famine helped initiate the nationalism and revolutionary movements of Ireland, which ultimately helped it gain its independence. Today twenty-six counties are known as the Republic of Ireland, and the nation is now thriving in the twenty-first century. Yet, the Irish were able to maintain their culture and traditions throughout the famine. The strength and perseverance of the Irish people through such bitter and challenging times is quite admirable. 

\newpage

\thispagestyle{empty}
\begin{flushright}Roche 7\end{flushright}
\begin{center}Notes\end{center}

\hspace{.4in}1. Noel Kissane, \emph{The Irish Famine: A Documentary History} (Dublin: Syracuse University Press, 1995), 171.

\hspace{.4in}2. Ibid.

\hspace{.4in}3. Don Nardo, \emph{The Irish Potato Famine} (San Diego: Lucent Books Inc, 1990), 12.

\hspace{.4in}4. Cecil Woodham-Smith, \emph{The Great Hunger} (New York: Harper \& Row Publishers, 1962), 27.

\hspace{.4in}5. Ibid.

\hspace{.4in}6. Kissane, \emph{Irish Famine}, 1.

\hspace{.4in}7. Ibid., 3.

\hspace{.4in}8. Ibid.

\hspace{.4in}9. Cormac O’Grada, \emph{The Great Irish Famine} (Cambridge: Press Syndicate of the University of
Cambridge, 1995), 15-17.

\hspace{.4in}10. Eileen Moore Quinn, "Entextualizing Famine, Reconstituting Self: Testimonial Narratives
From Ireland," \emph{Massachusetts Institute of Technology} 74, no. 2 (2001): 74, accessed April 20,
2013, http://www.jstor.org/stable/3318389.

\hspace{.4in}11. O'Grada, \emph{Irish Famine}, 16.

\hspace{.4in}12. Ibid., 33.

\hspace{.4in}13. Nardo, \emph{Potato Famine}, 16.

\hspace{.4in}14. Ibid., 32.

\hspace{.4in}15. Kissane, \emph{Irish Famine}, 107.

\hspace{.4in}16. Ibid., 27-33.

\newpage
\thispagestyle{empty}
\begin{flushright}Roche 8 \end{flushright}

\hspace{.4in}17. Susan Campbell Bartoletti, \emph{Black Potatoes} (Boston: Houghton Mifflin Company, 2011), 40.

\hspace{.4in}18. Kissange, \emph{Irish Famine}, 171

\hspace{.4in}19. Dean M. Braa, "The Great Potato Famine and the Transformation of Irish Peasant Society," \emph{Science and Society} 61, no. 2 (1997): 213, accessed April 3, 2013 http://www.jstor.org/stable/40403619


\newpage

\thispagestyle{empty}
\begin{flushright}Roche 9\end{flushright}
\begin{center}Bibliography\end{center}

Bartoleti, Susan Campbell. \emph{Black Potatoes} Boston: Houghton Mifflin, 2011.

Braa, Dean M. "The Great Potato Famine and the Transformation of Irish Peasant 

\hspace{.4in}Society." \emph{Science and Society} 61, no. 2 (1997): 193-215.

Kissane, Noel. \emph{The Irish Famine: A Documentary History}. Dublin: Syracuse 

\hspace{.4in}University Press, 1995.

Nardo, Don. \emph{The Irish Potato Famine.} San Diego: Lucent Books, 1990.

O' Grada, Cormac. \emph{The Great Irish Famine}. Cambridge:Press Syndicate of the 

\hspace{.4in} University of Cambridge, 1995.

Quinn, Eileen Moore. "Entextualizing Famine, Reconstituting Self:Testimonial 

\hspace{.4in}Narratives From Ireland." \emph{Anthropological Quarterly} 74, no. 2 (2001): 72-88.

Woodham-Smith, Cecil.  \emph{The Great Hunger.} New York: Harper \& Row, 1962.


\end{Spacing}

\newpage
%--------------DONE-------



\section {In-text citations}
The Chicago format uses either \textbf{endnotes} or \textbf{footnotes} to cite sources 
within the text. While the choice between them is left to the author, 
\emph{we require that you use endnotes} in the Chicago form since they prove less 
distracting and do not affect the page count of your written work. 

In the Chicago form, an in-text citation is indicated by a superscript number resembling 
the following:

\begin{quote}
Recent scholarship on the concept of sovereignty has displayed a 

remarkable lack of interest in the role of private property.\textsuperscript{7}
\end{quote}
This in-text reference will correspond to a citation at the conclusion of the document, 
such as this one:

\begin{quote}
\hspace{.4in}7. Giorgio Agamben, \emph{Homo Sacer: Sovereign Power and Bare Life}

(Stanford: Stanford UP, 1998), 96.
\end{quote}


\section{The notes page}

In the Chicago style, the endnotes appear on what is known as the notes 
page\textemdash a separate page that directly follows the conclusion of the essay. The 
notes page is organized as a numbered list that presents each citation in the order that 
it appears within the essay. Thus, your first citation will be endnote 1, your second will 
be endnote 2, and so on.

When setting up the notes page, use the following rules:

\begin{itemize}
\item Center the word "Notes" at the top of the page.
\item Double-space endnotes.
\item Indent the first line of an endnote entry with one .5" tab.
\end{itemize}

In the interest of efficiency, the Chicago form uses a number of rules to streamline the 
work involved in presenting the essay's footnotes or endnotes. While this design 
ultimately means less typing, a number of strict rules must be followed:

\begin{itemize}
\item Present the citations in the numerical order as they appear within the text.

\item The first time a source is cited, use the full Chicago notes form.

\item If the same source is used more than once, the shorthand version of the Chicago 
notes form is used the second (and each subsequent) time. The shorthand version 
contains \textbf{a)} the author's last name, \textbf{b)} a shortened version of the title, 
and \textbf{c)} the page number of the citation.

\item If a single source is used twice or more in a row, the Latin abbreviation "Ibid." is 
used along with the page number, rather than the shorthand version of the form. 
(Ibid. means "in the same place.")

\item If the same source is used \emph{twice in a row} and the citation \emph{is from 
the same page as the previous citation}, "Ibid." is used by itself without the 
page number. 
\end{itemize}


\section{The bibliography page}
Some professors may ask you to include both a notes page and a full bibliography page 
when using the Chicago format. As you will see in the next section, the bibliography 
form differs slightly from the notes form, so take care to use the correct one.

\begin{itemize}
\item Place the bibliography page after the notes page.
\item Center the word "Bibliography" at the top of the page.
\item Double-space entries.
\item Alphabetize by the author's last name.
\item Indent the second (and any subsequent) line of an entry with a .5" tab.
\end{itemize}

The following page contains an example of the formatting for a Chicago bibliography 
page.
\newpage



%-----CHICAGO BIBLIOGRAPHY    PAGE--------


\section{The Chicago Bibliography}

The following section provides examples for citing sources that are commonly found in
academic writing. We have separated them into sections on \textbf{books}, \textbf{periodicals}, \textbf{electronic
sources}, and \textbf{other types of sources} that are less common.

In each form, the first citation is the \textbf{bibliography} form and the second citation is the \textbf{notes} form.


\section{Book forms}

\subsection{A book by one author+}

\begin{quote}

Taylor, Herman. \emph{A Tale of One City}. New York: Little and Sons, 1998.

\medskip

\hspace{.4in}1. Herman Taylor, \emph{A Tale of One City} (New York: Little and Sons, 

1998), 77.
\end{quote}


\subsection{Multiple authors+}

\begin{quote}

Roberts, John, Philip Glass and Jane Hinds. \emph{Recovering the City of Boston}. 

\hspace{.4in}Boston: U of Massachusetts P, 2000.


\medskip

\hspace{.4in}1. John Roberts, Philip Glass and Jane Hinds, \emph{Recovering the City}

\emph{of Boston} (Boston: University of Massachusetts Press, 2000), 77.
\end{quote}

\ding{96} Include up to three authors in the \textbf{notes} form. If there are four or more, give the first author's name and then use "et. al." ("and others") to replace the other authors. In the \textbf{bibliography} form, include up to ten authors.

\subsection{A book with an editor+}
\begin{quote}
James, Henry. \emph{Portrait of a Lady}. Edited by Leon Edel. Boston: Houghton,

\hspace{.4in}1963. Print.
\medskip

\hspace{.4in}12. Henry James, \emph{Portrait of a Lady}, ed. Leon Edel (Boston: Houghton,

1963), 77.
\end{quote}

\subsection{Book with editor only+}
\begin{quote}
Smith, Edward, ed. \emph{Stuck in Goshen}. Nashville: Greenwood Press,

\hspace{.4in}1963.
\medskip

\hspace{.4in}17. Edward Smith, ed., \emph{Stuck in Goshen} (Nashville: Greenwood 

Press, 1963), 77.
\end{quote}


\subsection{An edition (other than the first)+}
\begin{quote}
Thompson, Fred. \emph{Why I Fight}. 3rd ed. New York: Vanity Publications,

\hspace{.4in}2000.

\medskip

\hspace{.4in}12. Fred Thompson, \emph{Why I Fight}, 3rd. ed. (New York: Vanity 

Publications, 2000), 77.
\end{quote}

\subsection{Corporate author (written by organization or government)+}
\begin{quote}

John Bigan Society. \emph{The Religions of Kenya}. New York: John Bigan 

\hspace{.4in}Publishing, 2000.

\medskip

\hspace{.4in}10. John Bigan Society, \emph{The Religions of Kenya} (New York: John Bigan Publishing, 2000), 77.
\end{quote}

\subsection{An anthology+}
\begin{quote}

Foner, Eric, ed. \emph{An American Voice: A Collection of America's Finest}

\hspace{.4in}\emph{Essays}. Boston: McKinley and Smith, 2011.

\medskip

\hspace{.4in}15. Eric Foner, ed., \emph{An American Voice: A Collection of America's}

\emph{Finest Essays} (Boston: McKinley and Smith, 2011), 77.

\end{quote}

\subsection{Work in an anthology+}

\begin{quote}

Graves, Thomas. "The History of our National Anthem." In \emph{An American}

\hspace{.4in}\emph{Voice: A Collection of America's Finest Essays}, edited by Eric

\hspace{.4in}Foner, 77-88. Boston: McKinley and Smith, 2011. 

\medskip

\hspace{.4in}13. Thomas Graves, "The History of our National Anthem," in \emph{An}

\emph{American Voice: A Collection of America's Finest Essays}, ed. Eric Foner,

(Boston: McKinley and Smith, 2011), 77-88.

\end{quote}

\subsection{No author or editor+}

\begin{quote}

\emph{A Guide to Boston}. Boston: Beantown Publishing, 2000.

\medskip

\hspace{.4in}25. \emph{A Guide to Boston} (Boston: Beantown Publishing, 2000), 22.
\end{quote}

\subsection{Introduction, preface, forward or afterward+}
\begin{quote}

Knox, John. Introduction to \emph{The Life of James}, by Elders Johnson, 2-9.

\hspace{.4in}New York: Random House, 2009.

\medskip

\hspace{.4in}33. John Knox, introduction to \emph{The Life of James}, by Elders 

Johnson (New York: Random House, 2009), 7.

\medskip

\ding{96} Substitute the word introduction with preface, afterward, or forward.

\end{quote}

\subsection{Book with a translator+} 
\begin{quote}
McDougle, Astrid. \emph{The Basics of Gaelic}. Translated by Paddy Maloney.

\hspace{.4in}New York: Vintage, 1990.

\medskip

\hspace{.4in}45. Astrid McDougle, \emph{The Basics of Gaelic}, trans. Paddy Maloney

(New York: Vintage, 1990), 22.
\end{quote}


\subsection{Multivolume work+}
\begin{quote}

Graves, Johanna. \emph{Ronald Reagan and the Iran-Contra Affair}. Vol. 7. New 

\hspace{.4in}York: Greenstalk, 1988.

\medskip

\hspace{.4in}23. Johanna Graves, \emph{Ronald Reagan and the Iran-Contra Affair}, 

vol. 7 (New York: Greenstalk, 1988), 77.
\end{quote}

\ding{96} If the volume has a separate title, place a comma after the volume number and enter its name in italics.

\subsection{Book in a series+}

\begin{quote}

Smith, Rod. \emph{American Economic Expansion in the Gilded Age}.

\hspace{.4in} American Economic History Series. New York: Grim and Drang, 

\hspace{.4in}1988.

\medskip

\hspace{.4in}44. Rod Smith, \emph{American Economic Expansion in the Gilded Age},

American Economic History Series (New York: Grim and Drang, 1988), 78.

\end{quote}

\subsection{Article in a reference work, such as a dictionary or encyclopedia+}
\begin{quote}

\hspace{.4in}7. \emph{Merriam-Webster's Collegiate Dictionary}, 10th ed., s.v. 

"Suzerian." 

\end{quote}

\ding{96} Well-known reference sources, such as encyclopedias or dictionaries, that are arranged alphabetically
by word or arranged by term do not require page numbers or need to be included in your bibliography. Abbreviate the Latin term \emph{sub verbo}, or "under the word," in your note to indicate this.

\subsection{Sacred texts+}

For sacred texts such as the Bible, Koran, or Torah, cite the work in your notes, but not the bibliography. In the note, provide information about the chapter and verse, but not the page number. If there is a version, put that information in
parenthesis after the chapter and verse information.

\begin{quote}

\hspace{.4in}22. Genesis 2: 2-5 (New International Version).

\end{quote}

\subsection{Book with title within the title+}
If a book title contains the title of another book, remove the italics to indicate the title
of the other work:

\begin{quote}
Hixson, Fred. \emph{On Cormac McCarthy's }Blood Meridian. New York: 

\hspace{.4in}Plainspeak Press, 2000.

\medskip

\hspace{.4in}45. Fred Hixon, \emph{On Cormac McCarthy's} Blood Meridian (New

York: Plainspeak Press, 2000), 56.

\end{quote}


%--------------------------------------------------------------------------------------------
%Periodicals
%-------------------------------------------------------------------------------------------


\section{Periodical forms}

\subsection{Article in a scholarly journal+}
\begin{quote}
Taylor, James. "The Indian Matter of Charles Brockden Brown's

\hspace{.4in}\emph{Edgar Huntly}." \emph{American Literature} 45, no. 6 (1998): 432-45.

\medskip

\hspace{.4in}19. James Taylor, "The Indian Matter of Charles Brockden Brown's 

\emph{Edgar Huntly}," \emph{American Literature} 45, no. 6 (1998): 434.

\end{quote}

\subsection{Article in a newspaper+}
\begin{quote}
McKinley, Robert. "Cat Saved from Dog." \emph{New York Times}, October 28, 

\hspace{.4in}2000, early edition, sec. B.

\medskip

\hspace{.4in}21. Robert McKinley, "Cat Saved from Dog," \emph{New York Times}, 

October 28, 2000, early edition, sec. B.

\end{quote}

\ding{96} If there is no section or edition information, end with the year of publication.

\subsection{A review+}
\begin{quote}

Smith, Jerry S. Review of \emph{The Orchard Revival}, by Cormac 

\hspace{.4in}Freedman. \emph{Oregon Literary Review} 120, no. 4 (1998): 20-23.

\medskip

\hspace{.4in}43. Jerry S. Smith, review of \emph{The Orchard Revival}, by Cormac 

Freedman, \emph{Oregon Literary Review} 120, no. 4 (1998): 22.
\end{quote}

\subsection{An unsigned article+}

\begin{quote}
"A Walk Down Nostalgia Lane." \emph{Bloomington Sun} October 20, 

\hspace{.4in}2013, sec. B6.

\medskip

\hspace{.4in}21. \emph{Bloomington Sun}. "A Walk Down Nostalgia Lane." October 20,

2013, sec. B6.

\end{quote}

\ding{96} In an unsigned article, state the article title first in the note. In the bibliography, begin
with the name of the publication.



\subsection{Article in a magazine+}
\begin{quote}
Smith, Jim. "Remembering Tony." \emph{New Yorker}, January 12, 2001, 12-18.

\medskip

\hspace{.4in}9. Jim Smith, "Remembering Tony," \emph{New Yorker}, January 12, 2001, 15.

\end{quote}

%--------------------------------------------------------------------
\section{Online sources}

The Chicago form prefers that citations of online sources use a DOI (Digital Object Identifier).
A DOI is a long string of numbers and letters that provide a unique identifier for an online object, 
such as an article or book. However, many online objects do not have a DOI. In that case, the Chicago 
form asks you to use a stable URL. If no stable URL is available, and the URL for your source is very long,
you may shorten it to include only the domain name. For example: http://www.nytimes.com.

\subsection{Article in an online database+}
Cite the source as you would a print article then include the DOI or stable URL. If neither exists, end the
citation with the database name.

\begin{quote}
Taylor, Hayden. "\emph{Moby Dick} and the Cold War." \emph{American}

\hspace{.4in}\emph{Literature} 45, no. 6 (2010): 45-57. doi: 12.2398/ahl.483.1.112.

\medskip

\hspace{.4in}22. Hayden Taylor, "\emph{Moby Dick} and the Cold War," \emph{American} 

\emph{Literature} 45, no. 6 (2010): 45-57. doi: 12.2398/ahl.483.1.112.
\end{quote}

\subsection{A work from a website+}

\begin{quote}

Reagan, John. "The Judo Champion Parent." \emph{Parenthood Online}. 

\hspace{.4in}The Parent Institute of Boston. Accessed October 5, 2011. 

\hspace{.4in}http://www.pibonline.org/10-5-reg.

\medskip

\hspace{.4in}17. Reagan, John, "The Judo Champion Parent," \emph{Parenthood}

\emph{Online}, The Parent Institute of Boston, accessed October 5, 2011. 

http://www.pibonline.org/10-5-reg.
\end{quote}

\ding{96} If there is no author listed, use the name of the sponsor as the author and do not 
not repeat the sponsor after the name of the website.

\subsection{Article in an online reference work, such as Wikipedia+}
\begin{quote}

\hspace{.4in}8. Wikipedia, s.v. "Iraq," last modified May 1, 2013, http://en.wikipedia.org/wiki/Iraq

\end{quote}

\ding{96} Well-known reference sources, such as encyclopedias or dictionaries, that are arranged alphabetically
by word or arranged by term do not require page numbers or need to be included in your bibliography. Abbreviate the Latin term \emph{sub verbo}, or "under the word," in your note to indicate this.

\subsection{E-book+}
\begin{quote}

Melville, Scott. \emph{The Taste of Plum in Afghanistan}. New York: RP Johnson, 

\hspace{.4in}2012. doi: 12.44589/9904384/88223287Z.

\medskip

\hspace{.4in}44. Scott Melville, \emph{The Taste of Plum in Afghanistan} (New York:

 RP Johnson, 2012). doi: 12.44589/9904384/88223287Z.
\end{quote}

\ding{96} If no DOI is available, use a URL. If this is a downloaded book from an online vendor, indicate this 
after the year in place of the DOI or URL. For example:

\begin{quote}
Taylor, Fred. \emph{Forgiving Esther}. New York: Brace and Smith, 2012. Kindle 

\hspace{.4in}edition.

\end{quote}

\ding{96} Since pagination is often affected by text size or form factors in electronic publications, you may 
use chapter numbers or section titles in place of page numbers. For example:

\begin{quote}
Taylor, Fred. \emph{Forgiving Esther}. New York: Brace and Smith, 2012. Kindle 

\hspace{.4in}edition, chapter 2.

\end{quote}


\subsection{Article in an online scholarly journal+}

\begin{quote}

Nelson, Grady. "Electronic Literature Comes of Age." 

\hspace{.4in}\emph{e-Lit Quarterly} 8, no 2 (2012): 2-12. Accessed October 28, 2013. 

\hspace{.4in}http://www.e-litquarterlyonline/8-2

\medskip

\hspace{.4in}17. Grady Nelson, "Electronic Literature Comes of Age,"  

\emph{e-Lit Quarterly} 8, no 2 (2012): 7, accessed October 28, 2013, 

http://www.e-litquarterlyonline/8-2

\end{quote}

\subsection{Article in an online newspaper+}

\begin{quote}

Taylor, Robert C. "Harvesting Undersea Sponges." \emph{New York Times}, 

\hspace{.4in}January 5, 2012. Accessed December 12, 2013. 

\hspace{.4in}http://www.nytimes.com.

\medskip

\hspace{.4in}7. Robert C. Taylor, "Harvesting Undersea Sponges," \emph{New York}

 \emph{Times}, January 5, 2012, accessed December 12, 2013. http://www.nytimes.com.

\end{quote}

\ding{96} Extremely long URLs may be shortened to include the address to the domain name,
as in the examples above.

\subsection{Article in an online magazine+}

\begin{quote}
James, Brian. "The New War on Terror." \emph{Slate.com}, June

\hspace{.4in}20, 2012. Accessed November 10, 2013. http://www.slate.com/6-

\hspace{.4in}10-2013/jamesb4387290.

\medskip

\hspace{.4in}29. Brian James, "The New War on Terror," \emph{Slate.com}, June

20, 2012, accessed November 10, 2013, http://www.slate.com/6-

10-2013/jamesb4387290.
\end{quote}


\subsection{Blog entry+}
\begin{quote}

Tate, Larry. "Spontaneous Order." \emph{I Hate What You Just Said} (blog). 

\hspace{.4in}February 11, 2011. Accessed May 12, 2013. http://www.ihatewhatyoujustsaid.com

\hspace{.4in}/2011/02/11/spontaneous-order/.

\medskip

\hspace{.4in}12. Larry Tate, "Spontaneous Order," \emph{I Hate What You Just Said}

 blog, February 11, 2011, accessed May 12, 2013, http://www.ihatewhatyoujustsaid.com

/2011/02/11/spontaneous-order/.

\end{quote}

%Online editorial
%Online film review

\subsection{Email+}
Email sources should be placed in the notes, not the bibliography.

\begin{quote}

\hspace{.4in}41. John Coolidge, email message to author, December 21, 2012. 
\end{quote}

%Posting from an online discussion board

\subsection{Podcast+}

\begin{quote}

James Spenser. "Dark Lagers." \emph{Basic Brewing Radio}. Podcast audio. 

\hspace{.4in}January 31, 2013, http://basicbrewing/bbr01-31-13darklagers.mp3

\medskip

\hspace{.4in}21. James Spenser, "Dark Lagers," \emph{Basic Brewing Radio}, podcast audio, January 31, 2013, http://basicbrewing/bbr01-31-13darklagers.mp3

\end{quote}

\ding{96} For a podcast, include the performer's name followed by the episode title, the host's name, show title, sponsor (if any), the medium (podcast audio or podcast video), date of publication and URL.

\section{Other types of sources}

\subsection{A dissertation+}

\begin{quote}

Redburn, Marcus. \emph{A Study of Melville's Aesthetics}. PhD

\hspace{.4in} diss., Boston University, 2012.

\medskip

\hspace{.4in}22. Marcus Redburn, \emph{A Study of Melville's Aesthetics} (PhD 

diss., Boston University, 2012).

\end{quote}


\subsection{An advertisement}

\begin{quote}
\textbf{Product or Company. Advertisement. \emph{Title of Publication} Date or}

\hspace{.4in}\textbf{Volume.Issue (Year): Page(s). Medium.}

\medskip

Dove Body Wash. Advertisement. \emph{Fortune Monthly} Oct. (2012): 23. 

\hspace{.4in}Print.


\end{quote}



\subsection{Artwork}

\begin{quote} 

\textbf{Artist's Last name, First name. \emph{Title}. Medium. Year. Institution, City.}

\medskip

Freeman, Dianna. \emph{Still Life 7}. Watercolor. 2009. Hunter Museum of Art, 

\hspace{.4in}Chattanooga.

\end{quote}

\subsection{Film or video clip}

\begin{quote}

Anderson, Wes and Owen Wilson. \emph{Rushmore}. Directed by Wes 

\hspace{.4in}Anderson. 1998. Buena Vista International, 2000. DVD.

\medskip

\hspace{.4in}7. Wes Anderson and Owen Wilson, \emph{Rushmore}, directed by Wes 

Anderson (1998, Buena Vista International, 2000, DVD).

\end{quote}

\ding{96} Include the writer(s), the date the film was originally released, the studio, and the year your recording was released. At the end of your citation, include the medium of the film: DVD, videocassette, etc.)




%Broadcast interview
%Published interview
%Unpublished letter
%Published letter
%Map or chart
%Musical score
%Sound recording
%Oral presentation
%Paper from a conference
%Performance
%Television or radio program
%Pamphlet, brochure, or press release
%Legal source
%A digital file, such as .mp3, .pdf, etc.



%----------------------------------------------------------------------------------------
% END OF SECTION
%----------------------------------------------------------------------------------------
