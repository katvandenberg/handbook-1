
%----------------------------------------------------------------------------------------
%	WORKING WITH SOURCES
%----------------------------------------------------------------------------------------

\chapter{Working with Sources}

Academic writing will require you to integrate the ideas and words of
other scholars into your own writing. There are only three ways that 
the writing of others may appear in your writing: \textbf{quotation}, 
\textbf{summary}, and \textbf{paraphrase}. Writing an academic paper requires
a mastery of all three critical skills.

\section{Quotation}
 
\subsection{When should I quote something?}

Quoting is word-for-word borrowing from a source text. As we describe below, 
quotation should be used sparingly. Only quote under the following scenarios:
 
\begin{itemize}

\item When the subject you are writing about is the specific language used 
within a text, such as an interpretation of a line of poetry or a passage from 
a literary text.
 
\item You are calling on the words of a known authority, on whose credibility you 
are depending. For example, if you are writing on racial inequality, a quotation 
from Martin Luther King, Jr. is appropriate.

\item When the source text contains language that is memorable, beautiful, or particularly apt.
 
\item A quotation is often necessary when you describe legal discourse (such as 
a law or court ruling) where words cannot be paraphrased or summarized without 
altering the meaning and effect of the legal language.
 
\end{itemize}
 
\subsection{How do I integrate quoted material?}
 
 \begin{itemize}       	
\item Use a \textbf{signal phrase} to introduce the quoted passage.

\item Use quotation marks.

\item Provide a citation in your chosen format, such as MLA or Chicago.
        	
\item If necessary, use ellipsis or square brackets to alter the source, satisfy 
grammar, or provide clarifying information. 

\end{itemize}

\subsection{What should I avoid?}

\begin{itemize}
\item \textbf{Avoid excessive use of quotation}. If you quote too often it can 
make it appear that you have not fully read or understood the source material. 
It may also make your writing appear lazy and thoughtless.

\item \textbf{Avoid excessive use of block quotation}. Block quotations should 
be rare, and reserved for special language that you believe cannot be summarized, 
paraphrased, or otherwise reduced.

\item \textbf{Avoid inserting a quote within your writing without providing your 
commentary}. Expand on your quotations, clarify or critique them, and connect 
them to your other sources and your argument.

\item \textbf{Avoid inserting quotations without signal phrases}. Quotations should be introduced and
woven into your own writing. They should rarely stand alone.

\end{itemize}

\section{Altering quotations}

When you remove a quotation from its original context and place it within your 
own writing, you must ensure that the resulting sentence(s) remain grammatical 
and comprehensible. Occasionally, you may need to slightly alter the original 
source material to satisfy grammar or give your readers information that will 
help them follow your meaning. To alter a source, make use of \textbf{brackets} 
and \textbf{ellipsis}.

\subsection{How do I use brackets?}

Square brackets are used to alter source material in two ways. First, in the interest of grammar, 
a bracket may be used to change a verb tense, alter capitalization, or add a word. Secondly, a bracket may
also be used to provide clarifying information for the reader. 

This sentence has been altered to satisfy grammar:
\begin{quote}
Simon Frith believes we "use music to organize our sense of time [and] to make 
our private feelings public" (87). 
\end{quote}

This sentence has been altered to provide the reader with clarifying information:

\begin{quote}
John Cole declared the "Department [of Homeland Security] a waste of money" (97).
\end{quote}

\subsection{How do I use ellipsis?}

Occasionally, we need to remove words or sentences from a quotation. To do this, 
we use a series of dots, known as ellipses, to show when you have altered a source 
by omitting words or sentences. Be careful that the quote you present to the 
reader is an \emph{accurate} reflection of the original text; it is improper 
(and unethical) to hack up a text so that it says something that the author 
didn't really intend. Further, the "hybrid" sentence that you produce must be grammatical.

When omitting words within a single sentence, use three periods in your ellipsis. 
However, if the omission crosses a period and continues into another sentence, 
you must use four dots in your ellipsis to indicate where the period is located 
in the original source. For example:


\textbf{Original source}:

Rock music depends on the myth of authenticity. Paglia is wrong to believe that only contemporary rock musicians play music to make money. Rock music has never been authentic and spontaneous, or particularly revolutionary.

\textbf{You write}:

Gracyk believes that "Rock music depends on the myth of authenticity. . . . Rock music has never been authentic . . . or particularly revolutionary" (23).


 

\section{Using signal phrases}
 
 
\subsection{What are they?}
 
Signal phrases are words used to indicate that the material you are incorporating 
is borrowed from a source. Often, a signal phrase uses the name of the author from 
whom the text is borrowed.
 
\subsection {Why use them?}
 \begin{itemize}

\item They make it clear that you are transitioning from your own ideas and 
writing to the ideas and writing of another. This makes your paper more coherent.
 
\item They make it clear when you have begun to paraphrase or summarize.
Unlike quotations, paraphrases and summaries are not formatted with quotation 
marks; therefore, it is difficult for readers to know when or where you have 
begun to paraphrase or summarize unless you include these phrases.
 
\item They make the tone of your paper more academic and authoritative. 
 
\item They compel you to articulate how your ideas relate to those you 
have borrowed from others. This will direct your attention to the precise 
ways in which authors agree or disagree with you or each other, and allow 
you to make these intersections clear to your reader.
 
\item Using these signal phrases will help you to avoid plagiarism.
 \end{itemize}
 
\subsection{How do you construct a signal phrase?}
 
\begin{itemize}
\item \textbf{Use the author's name}. The first time you mention an author, include the 
author's name, the title of his or her work, and perhaps a brief statement indicating the 
author's credentials. Once you have introduced an author in your paper, only use his or 
her last name if you mention him or her again.

\item \textbf{Use a strong verb to characterize what the author has “done.”} See the list 
below for suggestions. Be sure to pick the verb which most precisely articulates the 
author's action:

\begin{quote}
asserts, argues, believes, claims, emphasizes, insists, observes, reports, 
suggests, acknowledges, admires, agrees, corroborates, endorses, extols, praises, 
verifies, illustrates, expands on, rejects, complicates, contends, contradicts, denies, 
disagrees, refutes, questions, warns, proposes, implores, exhorts, demands, calls for, 
recommends, urges, advocates, wonders, asks, rejects, encourages.
\end{quote}

\end{itemize}
 
The following paragraph demonstrates the appropriate use of signal phrases:
 
 \begin{quote}
In his article, "Writing is Thinking," Dr. Benjamin Warhol, a composition scholar, rejects 
the idea that one must have an outline prepared before one drafts a paper. While 
recognizing that some students must prepare their thoughts beforehand, Warhol warns 
that too much time spent on outlines may actually prevent students from completing 
papers on time (22-5). However, Sally T. Osmond, poet and  creative writing instructor at 
Fordham University, disagrees, arguing that students who make a plan before they 
begin their drafts tend to write more organized and fluent essays (7). \end{quote}

\section{Summarizing}
Summary is a critical skill that allows you to present the ideas of another writer in a condensed form. 
The length of a summary is dictated by your rhetorical needs, however they are always shorter than 
the original text. For example, the summary of a large book could be 20 pages, one paragraph, or one 
sentence. Although a summary sacrifices specificity and detail in the interest of brevity, it must always 
remain a faithful representation of the original text.


\subsection {Why are summaries important?}

Summary is one of the central skills needed for academic writing. Summary is used to 
provide the reader with background knowledge on a particular problem, history, or conversation. 
We also use summary to provide the reader with an appropriate context, so that our ideas may be 
understood within the broader context of writings on the subject in question. An excellent summary 
of this context goes far to establish you as a knowledgeable authority with your reader, someone 
whose views should be trusted and considered. 


\subsection{How do I incorporate them?}

\begin{itemize}
\item Since summaries do not use quotation marks, you must take care to indicate to your readers
that you are borrowing from the work of others. This is primarily accomplished through the use
of a \textbf{signal phrase} and a citation. As you move from your own writing to the summary of others,
use a signal phrase to indicate this transition. 

\item End the summary with an appropriate citation, noting the page(s) summarized.
\end{itemize}

\subsection{What should I avoid?}
\begin{itemize}

\item Plagiarizing\textemdash remember, summarized material is still borrowed material, even 
though you have greatly condensed it and put it entirely in your own words.
\end{itemize}

\subsection{Example of a summary}
The source text below is taken from page 35 of Patricia Nelson Limerick's essay, "Empire of Innocence."

\textbf{Original source}:
\begin{quote}
When academic territories were parceled out in the early twentieth century, anthropology got the tellers of tales and history got the keepers of written records.  As anthropology and history diverged, human differences that hinged on literacy assumed an undeserved significance.  Working with oral, preindustrial, prestate societies, anthropologists acknowledged the power of culture and of a received worldview; they knew that the folk conception of the world was not narrowly tied to proof and evidence.  But with the disciplinary boundary overdrawn, it was easy for historians to assume that literacy, the modern state, and the commercial world had produced a different sort of creature entirely—humans less inclined to put myth over reality, more inclined to measure their beliefs by the standard of accuracy and practicality (35).
\end{quote}

\textbf{Summary}:

\begin{quote}
As Patricia Nelson Limerick argues in her essay “Empire of Innocence,” historians have assumed that literate societies with vibrant economies and systems of governance were never beholden to myth or superstition (35).
\end{quote}



\section{Paraphrasing}

Think of paraphrase as a translation from English into English. It involves taking 
language from a source, putting it in your own words, and arranging it within your own original 
sentence structure(s). Unlike summary, which aims to reduce or distill an idea, a paraphrase should 
be similar in length to the original passage.

\subsection{Why are paraphrases important?}

Accurate paraphrase demonstrates mastery of your source materials and indicates an author
who is in control of his or her own writing and thinking. Whereas excessive quotation may reveal
an uncertain or tentative author, paraphrase demonstrates control and confidence. However, 
ensure that your paraphrases do justice to the original, or risk compromising your authority
with your readers.

\subsection{How do I incorporate them?}

\begin{itemize}
\item Like summaries, paraphrases do not use quotation marks. As a result, you must take care to indicate to your readers that you are borrowing from the work of others with a \textbf{signal phrase} and citation. As you move from your own writing to the paraphrase of others, use a signal phrase to indicate this transition.

\item End the paraphrase with an appropriate citation.

\end{itemize}

\subsection {What should I avoid?}

\begin{itemize}

\item Plagiarizing\textemdash remember, paraphrased material is still borrowed material, even 
though you have put it in your own words.

\item Merely changing a word or a phrase here or there. Instead, read the passage until 
you can put it aside and write your paraphrase without having to look back at it.
\end{itemize}

\subsection{Example of Paraphrase}
The source text below is taken from page 35 of Patricia Nelson Limerick's essay, "Empire of Innocence."

\textbf{Original source}:
\begin{quote}
When academic territories were parceled out in the early twentieth century, anthropology got the tellers of tales and history got the keepers of written records.  As anthropology and history diverged, human differences that hinged on literacy assumed an undeserved significance.  Working with oral, preindustrial, prestate societies, anthropologists acknowledged the power of culture and of a received worldview; they knew that the folk conception of the world was not narrowly tied to proof and evidence.  But with the disciplinary boundary overdrawn, it was easy for historians to assume that literacy, the modern state, and the commercial world had produced a different sort of creature entirely—humans less inclined to put myth over reality, more inclined to measure their beliefs by the standard of accuracy and practicality (35).
\end{quote}

\textbf{Paraphrase}:
\begin{quote}
In "Empire of Innocence," Patricia Nelson Limerick argues that during the early part of the last century the disciplines of anthropology and history separated. While anthropology focused on unlettered and illiterate communities, history became the study of societies who produced texts and records. Within the field of anthropology, a firm belief developed that oral cultures were characterized by mythological worldviews and superstitious beliefs; on the other hand, historians assumed that literate cultures were filled with individuals who only used reason and evidence to guide their thinking (35).
\end{quote}


%----------------------------------------------------------------------------------------
% END OF SECTION
%----------------------------------------------------------------------------------------
