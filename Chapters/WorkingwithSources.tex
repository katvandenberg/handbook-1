
%----------------------------------------------------------------------------------------
%	WORKING WITH SOURCES
%----------------------------------------------------------------------------------------

\chapter{Working with Sources}
\section{Quotation}
 
\subsection{When should I quote something?}

Quoting is word-for-word borrowing from source material. Only quote material when:
 
\begin{itemize}

\item You are writing about the specific language used within a text, such as an 
interpretation of a line of poetry or a passage from a literary text.
 
\item You are calling on the words of a known authority, on whose credibility you are 
depending. For example, if you are writing on racial equality, a quotation from Martin
Luther King, Jr. is appropriate.
 
\item You are referring to specific data such as numbers, dates, or statistics. A quotation
is also necessary when you describe legal discourse (such as a law or court ruling) where words cannot be
paraphrased or summarized without altering the meaning and effect of the language.
 
\end{itemize}
 
\subsection{How do I integrate quoted material?}
 
 \begin{itemize}       	
\item Introduce it with a \textbf{signal phrase} to indicate the author of the quoted
passage.

\item Use quotation marks.

\item Provide a citation in your chosen format, such as MLA or Chicago.
        	
\item If necessary, use square brackets to alter the source to satisfy grammar or provide
clarification.

\textbf{For example}:

\begin{quote} Simon Frith believes we "use music to organize our sense of time [and] 
to make our private feelings public" (87). 

\medskip

John Cole declared the "Department [of Homeland Security] a waste of money" (97).
\end{quote}

\item Use \textbf{ellipsis} to condense a passage of which you only need parts. Use a period 
after a sentence and before the ellipsis if you are omitting a whole sentence (for a 
total of 4 dots).

\textbf{For example}:
 If the source reads as follows:
\begin{quote}
Rock music depends on the myth of authenticity. Paglia is wrong to believe that only 
contemporary rock musicians play music to make money. Rock music has never been 
authentic and spontaneous, or particularly revolutionary.
\end{quote}
         
You might write:
\begin{quote}
Gracyk believes that "Rock music depends on the myth of authenticity. . . . Rock 
music has never been authentic . . . or particularly revolutionary" (23).
 \end{quote}
 \end{itemize}
 
\subsection{What should I avoid?}

\begin{itemize}
\item Avoid excessive use of quotation. If you quote too often it can make it appear 
that you have not fully read or understood the source material. It also makes your
writing appear very lazy and thoughtless.

\item Avoid excessive use of block quotation. Block quotations should be rare, and reserved
for special language that you believe cannot be summarized, paraphrased, or otherwise reduced.

\item Avoid inserting a quote within your writing without providing your commentary. 
Expand on it, clarify or critique it, and connect it to surrounding material and your main argument.

\end{itemize}


\section{Signal Phrases/Integrating Research}
 
 
\subsection{What are they?}
 
Signal phrases are words used to indicate that the material you are incorporating is 
borrowed from a source.
 
\subsection {Why use them?}
 \begin{itemize}

\item They make it clear that you are transitioning from your own material/thinking to the 
material/ideas of another. This makes your paper more coherent.
 
\item They make it clear when you have begun to paraphrase or summarize. Unlike quotations, 
paraphrases and summaries are not formatted with quotation marks; therefore, it is 
difficult for readers to know when or where you have begun to paraphrase or summarize 
unless you include these phrases.
 
\item They make the tone of your paper more academic/make your use of research clear.
 
\item They compel you to articulate how your ideas relate to those you have borrowed from
others. This will direct your attention to the precise ways in which authors agree or 
disagree with you or each other, and allow you to make these intersections clear to 
your reader.
 
\item Using these signal phrases will help you to avoid plagiarism.
 \end{itemize}
 
\subsection{How do you construct a signal phrase?}
 
\begin{itemize}
\item \textbf{Use the author's name}. The first time you mention an author, include the 
author's name, the title of his or her work, and perhaps a brief statement indicating the 
author's credentials. Once you have introduced an author in your paper, only use his or 
her last name if you mention him or her again.

\item \textbf{Use a strong verb to characterize what the author has “done.”} See the list 
below for suggestions. Be sure to pick the verb which most precisely articulates the 
author's action:
 
\begin{quote}asserts, argues, believes, claims, emphasizes, insists, observes, reports, 
suggests, acknowledges, admires, agrees, corroborates, endorses, extols, praises, 
verifies, illustrates, expands on, rejects, complicates, contends, contradicts, denies, 
disagrees, refutes, questions, warns, proposes, implores, exhorts, demands, calls for, 
recommends, urges, advocates, wonders, asks, rejects, encourages.\end{quote}
 \end{itemize}
 
The following paragraph demonstrates the appropriate use of signal phrases:
 
 \begin{quote}
In his article, "Writing is Thinking," Dr. Benjamin Warhol, a composition scholar, rejects 
the idea that one must have an outline prepared before one drafts a paper. While 
recognizing that some students must prepare their thoughts beforehand, Warhol warns 
that too much time spent on outlines may actually prevent students from completing 
papers on time (22-5). However, Sally T. Osmond, poet and  creative writing instructor at 
Fordham University, disagrees, arguing that students who make a plan before they 
begin their drafts tend to write more organized and fluent essays (7). \end{quote}

\section{Summarizing}

\begin{itemize}

\item Reviewing the main idea(s) of a page, pages, chapter/scene/stanza, or entire 
work.

\end{itemize}
\subsection {How is it different than paraphrasing?}

\begin{itemize}
\item A summary is much shorter than the source.
\end{itemize}
\subsection {Why are summaries important?}

\begin{itemize}
\item They demonstrate that you have read and comprehended source material. They 
can provide context for your argument or help you clarify a concept.
\end{itemize}

\subsection{How do I incorporate them?}

\begin{itemize}
\item Employ a \textbf{signal phrase}.

\item End each with an in-text citation (MLA) or footnote/endnote (Chicago), noting 
the pages summarized.
\end{itemize}

\subsection{What should I avoid?}
\begin{itemize}

\item Plagiarizing\textemdash remember, summarized material is still borrowed material, even 
though you have greatly condensed it and put it entirely in your own words.
 \end{itemize}
 
\section{Paraphrasing}

Think of paraphrase as a translation from English into English. It involves taking 
language from a source, putting it in your own words, and arranging it within your own original 
sentence structure(s). Unlike summary, which aims to reduce or distill an idea, a paraphrase should 
be similar in length to the original passage.

\subsection{Why are paraphrases important?}

Accurate paraphrase demonstrates mastery of your source materials and indicate an author
who is in control of his or her own writing and thinking. Whereas excessive quotation may reveal
an uncertain or tentative author, paraphrase demonstrates control and confidence. However, 
ensure that your paraphrases do justice to the original, or risk compromising your authority
with your readers.

\subsection{How do I incorporate them?}

\begin{itemize}
\item Employ a \textbf{signal phrase} to introduce them.

\item End each with an in-text citation (see the \textbf{MLA} or \textbf{Chicago} style
chapters).

\end{itemize}

\subsection {What should I avoid?}

\begin{itemize}

\item Plagiarizing\textemdash remember, paraphrased material is still borrowed material, even 
though you have put it in your own words (see \textbf{Definition of plagiarism}).

\item Merely changing a word or a phrase here or there. Instead, read the passage until 
you can put it aside and write your paraphrase without having to look back at it.
\end{itemize}

%----------------------------------------------------------------------------------------
% END OF SECTION
%----------------------------------------------------------------------------------------
