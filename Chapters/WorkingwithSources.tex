
%----------------------------------------------------------------------------------------
%	WORKING WITH SOURCES
%----------------------------------------------------------------------------------------

\chapter{Working with Sources}
\section{Quotation}
 
\subsection{When should I quote something?}

Quoting is word-for-word borrowing from source material. Only quote material when:
 
\begin{itemize}

\item You are going to be discussing the language used in a specific part of a text (i.e., 
if you are interpreting a line of poetry, or analyzing a portion of a literary text).
 
\item You are calling on the words of a known authority, on whose credibility you are 
depending (i.e., if you are discussing MLK Jr's stance on equality).
 
\item You are referring to highly technical data or specific numbers, dates, or statistics.
 
\item You are relying on a selection of text that is uniquely or precisely worded.
 
 \end{itemize}
 
\subsection{How do I integrate quoted material?}
 
 \begin{itemize}       	
\item Introduce it with a signal phrase (see \emph{Signal Phrases}), so the quote is not 
just "dropped" into the paper.

\item Use quotation marks.

\item Use either an in-text citation (MLA) or a footnote/endnote (Chicago).
        	
\item Use brackets to insert your own words into a quote (this may be necessary to 
ensure that the grammar of the quote matches the grammar of your text).

\textbf{For example}:

\begin{quote} Simon Frith believes we "use music to organize our sense of time [and] 
to make our private feelings public" (87). 
\end{quote}

\item Use ellipsis to condense a passage of which you only need parts. Use a period 
after a sentence and before the ellipsis if you are omitting a whole sentence (for a 
total of 4
dots).

\textbf{For example}:
 If the source reads as follows:
\begin{quote}
Rock music depends on the myth of authenticity. Paglia is wrong to believe that only 
contemporary rock musicians play music to make money. Rock music has never been 
authentic and spontaneous, or particularly revolutionary.
\end{quote}
         
You might write:
\begin{quote}
Gracyk believes that "Rock music depends on the myth of authenticity. \ldots Rock 
music has never been authentic\ldots  or particularly revolutionary" (23).
 \end{quote}
 \end{itemize}
 
\subsection{What should I avoid?}

\begin{itemize}
\item Avoid relying excessively on quotes. If you quote too often it can make it appear 
that you have not fully read, understood, and "digested," the source material or that 
you are not comfortable putting it in your own words.

\item Relying excessively on long quotes (see above).

\item Inserting a quote and then doing nothing to discuss it, expand on it, clarify it, or 
connect it to surrounding material.

\end{itemize}


\section{Signal Phrases/Integrating Research}
 
 
\subsection{What are they?}
 
Signal phrases are words used to signal that the material you are incorporating is 
borrowed.
 
\subsection {Why use them?}
 
They make it clear that you are transitioning from your own material/thinking to the 
material/ideas of another. This makes your paper more coherent.
 
They make it clear when you have begun to paraphrase or summarize (unlike quotations, 
paraphrases and summaries are not formatted with quotation marks; therefore, it is 
difficult for readers to know when/where you have begun to paraphrase or summarize 
unless you include these phrases).
 
They make the tone of your paper more academic/make your use of research clear.
 
They compel you to articulate, precisely, the stances of those from whom you have 
borrowed. This will direct your attention to the precise ways in which authors agree or 
disagree with you or each other, and allow you to make these intersections clear to 
your reader.
 
Using these signal phrases will help you to avoid plagiarism.
 
 
\subsection{How do you construct a signal phrase?}
 
\textbf{Use the author's name}. The first time you mention an author, include the 
author's name, the title of his or her work, and perhaps a brief statement indicating the 
author's credentials. (Once you've introduced an author in your paper, just use his or 
her last name if you mention him or her again).

\textbf{Use a strong verb to characterize what the author has “done.”} See the list 
below for suggestions. Be sure to pick the verb which most precisely articulates the 
author's action:
 
\begin{quote}asserts, argues, believes, claims, emphasizes, insists, observes, reports, 
suggests, acknowledges, admires, agrees, corroborates, endorses, extols, praises, 
verifies, illustrates, expands on, rejects, complicates, contends, contradicts, denies, 
disagrees, refutes, questions, warns, proposes, implores, exhorts, demands, calls for, 
recommends, urges, advocates, wonders, asks, rejects, encourages.\end{quote}
 
 
\textbf{Example}:
 
 \begin{quote}
In his article, "Writing is Thinking," Dr. Benjamin Warhol, a composition scholar, rejects 
the idea that one must have an outline prepared before one drafts a paper. While 
recognizing that some students must prepare their thoughts beforehand, Warhol warns 
that too much time spent on outlines could actually prevent students from completing 
papers on time (22-5). He is correct--writing  is a process, and often one must draft 
before one  can realize what one has to say much less in what order one wants to say it. 
However, Sally T. Osmond, poet and  creative writing instructor at Fordham, disagrees,
 noting that... \end{quote}

\section{Paraphrasing}

A paraphrase involves taking a paragraph or passage from a source, putting it in your 
own words, and arranging it within your own original sentence structures. Although a 
paraphrase is similar in length to the original passage, it is sometimes employed to 
condense and clarify the main points of that passage, and, therefore, may be somewhat 
shorter in length.

\subsection{Why are paraphrases important?}

They allow you to demonstrate your knowledge of and comfort with other sources and 
may provide strong support for your argument or analysis or help you clarify an idea.

\subsection{How do I incorporate them?}

\begin{itemize}
\item Employ a signal phrase (see \emph{Signal Phrases}) to introduce them.

\item End each with an in-text citation (see \emph{MLA Style}) or 
(\emph{Chicago Style}).
\end{itemize}

\subsection {What should I avoid?}

\begin{itemize}

\item Plagiarizing--remember, paraphrased material is still borrowed material, even 
though you have put it in your own words (see \emph{Definition of plagiarism}).

\item Merely changing a word or a phrase here or there. Instead, read the passage until 
you can put it aside and write your paraphrase without having to look back at it.
\end{itemize}


\section{Summarizing}

\begin{itemize}

\item Reviewing the main idea(s) of a page, pages, chapter/scene/stanza, or entire 
work.

\end{itemize}
\subsection {How is it different than paraphrasing?}

\begin{itemize}
\item A summary is much shorter than the source.
\end{itemize}
\subsection {Why are summaries important?}

\begin{itemize}
\item They demonstrate that you have read and comprehended source material. They 
can provide context for your argument or help you clarify a concept.
\end{itemize}

\subsection{How do I incorporate them?}

\begin{itemize}
\item Employ a signal phrase (see \emph{Signal Phrases}).

\item End each with an in-text citation (MLA) or footnote/endnote (Chicago), noting 
the pages summarized.
\end{itemize}

\subsection{What should I avoid?}
\begin{itemize}

\item Plagiarizing--remember, summarized material is still borrowed material, even 
though you have greatly condensed it and put it entirely in your own words.
 \end{itemize}

%----------------------------------------------------------------------------------------
% END OF SECTION
%----------------------------------------------------------------------------------------
