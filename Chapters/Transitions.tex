
%----------------------------------------------------------------------------------------
%Transitions
%----------------------------------------------------------------------------------------

\chapter{Transitions}

\section {What are transitions?}

Transitions are words or phrases that connect ideas within or between sentences--transitions allow readers to follow your ideas across a paper.

\section{Why are they important?}

\begin{itemize}
\item They make your writing coherent (see \emph{Paragraph Coherence}). 

\item They can help you vary your sentence structure, which often improves your style.

\item Their employment often prompts you to consider, carefully, the logic behind the order of your points.

\end{itemize}
         

\textbf{If you want to clarify a causal relationship between ideas}, you might employ the following transition words: \emph{since}, \emph{so}, \emph{consequently}, \emph{as a result}, \emph{therefore}, \emph{then}, \emph{thus}, \emph{because}, or \emph{due to}.
 
 \begin{itemize}
	\item \textbf{Example}: ``As a result of his hard work, he aced the exam.''
\end{itemize}

\textbf{If you want to contrast one idea with another}, you might use these words: \emph{however}, \emph{on the other hand}, \emph{in contrast}, \emph{on the contrary}, \emph{yet}, \emph{rather}, or \emph{while}.

\begin{itemize}
	\item \textbf{Example}: ``Silver endorses human cloning, while Krauthammer does not.''
\end{itemize}
 
\textbf{If you wish to show that ideas are similar}, you might depend on the following words: \emph{likewise}, \emph{similarly}, \emph{in much the same way}, or \emph{also}.

\begin{itemize}
	\item  \textbf{Example}:  ``In much the same way that Frith values pop music, Berry loves rap.''      
\end{itemize}

\textbf{If you wish to use an example}, you might employ these words: \emph{for example}, \emph{for instance}, or \emph{in illustration}.

\begin{itemize}
	\item \textbf{Example}: ``He neglected his paperwork; for instance, he forgot to file his taxes.''
\end{itemize}
 

\textbf{If you wish to show how multiple points build on each other}, you might want these: \emph{Futhermore}, \emph{in addition}, \emph{also}, \emph{and}, \emph{moreover}

\begin{itemize}
	\item  \textbf{Example}: ``It important to analyze film.  And it is essential to attend to film style.''
\end{itemize}

\textbf{If you wish to elaborate on a point}, you might try these: \emph{in short}, \emph{ultimately}, \emph{by extension}, \emph{that is}, \emph{in other words}.

\begin{itemize}
	\item \textbf{Example}: ``Cloning human is immoral. That is, it undermines the moral value humans place on life.''
\end{itemize}

\textbf{If you wish to concede a point}, you might employ the following: \emph{although}, \emph{admittedly}, \emph{granted}, \emph{of course}, \emph{naturally}.

\begin{itemize}
	\item  \textbf{Example}: ``Granted, not all tax breaks are good for the economy.''
\end{itemize}
        	
\textbf{Finally, if you wish to sum up an idea}, these will be of use: \emph{in conclusion}, \emph{as a result}, \emph{in sum}, \emph{finally}, \emph{therefore}, \emph{thus}, \emph{in short}.

\begin{itemize}
	\item  \textbf{Example}: ``In sum, transitions are an important part of a coherent essay.''
\end{itemize}


%----------------------------------------------------------------------------------------
% END OF SECTION
%----------------------------------------------------------------------------------------