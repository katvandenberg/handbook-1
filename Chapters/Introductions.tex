
%----------------------------------------------------------------------------------------
%	Introductions
%----------------------------------------------------------------------------------------

\section{\textcolor{ForestGreen}{Introductions}}


 
\subsection {Why is the introduction important?}
 
\begin{itemize}
\item It is your first chance to establish your ethos [ethos=the sense of yourself that you create through your writing/the impression your reader gets of you due to your writing].
 
\item It is the part of the paper where you orient your reader to what is to follow in the rest of your paper.
 
\item It is your opportunity to hook/engage your audience, to make them interested.
\end{itemize}

\subsection{How do I write a strong introduction?}

\begin{enumerate}
\item \textbf{Hook your reader with a vivid and engaging opening sentence by avoiding general or vague sentences}.
        	               	        	
\begin{quote}

\textbf{Instead of}: ``Drugs are a big problem in the U.S. today.''

\textbf{Try}:``When Shay Howard was 5, she watched her mother
overdose on  the bathroom floor.''
\end{quote}
        	    	
\item {\textbf{Begin with a question, provocative statement, interesting data or statistic.}}

\begin{quote}
                            	                   
\textbf{Instead of}: ``There are a lot of debates about the sale of cells today.''

\textbf{Try}: ``Although Henrietta Lacks died in the 1950s, trillions
      of her living cells have since gone to the moon and  
      been used to cure cancer, all without her prior
      consent.''     
\end{quote}
                    	 
\item \textbf{Offer a brief introduction to the material in your paper, probably around 4-8 lines}. 

If you have not done so in your opening sentence, and will not do so in your thesis, be sure to mention any works, authors, or concepts central to the paper, though you should avoid doing so in a list-like fashion.
 
\item \textbf{Write your thesis} (see: \emph{Thesis Statements}).

After you have written your thesis, put it next to the prompt/assignment question to see if it addresses exactly what it is the professor has asked you to address.  
  
 \end{enumerate}

 
\subsection{Helpful hints on introductions}

\begin{itemize}
\item Avoid the clich\'ed phrase ``In today's world.''
        	
\item Do not employ a dictionary definition.
       	
\item In general, work to keep your introduction to half a page or less.    	  
 
\item Proofread your introduction very carefully and work on your style--remember, this is where you are making your first impression. You want to sound educated, on-topic, clear, and coherent.
 \end{itemize}
