%----------------------------------------------------------------------------------------
% Grading Rubric
%----------------------------------------------------------------------------------------

\chapter{Grading Rubric}


The following grading rubric has been adopted from 
\href{http://www2.winthrop.edu/english/WritingProgram/rubric.htm}{\{Winthrop University's Writing Program\}}:

\section{The A paper}
The A paper is superior work that far exceeds the requirements of the assignment. This 
paper tackles the topic in an innovative way, with an appropriate sense of audience 
and an effective plan of organization. The writer has clearly used the elements of 
reasoning to generate material for developing as well as structuring the paper. When 
evaluated, the paper meets the highest standards of critical thinking. For example, the 
point of view and purpose are clear. It outlines the question at issue and assumptions 
with precision. Its information is accurate, and its concepts are relevant. It reaches 
sufficient conclusions and interpretations. Its implications and consequences are deep 
and broad. All elements can be judged against the standards very favorably. When the 
writer uses source materials in the A paper, he/she demonstrates complete 
understanding of potential critical thinking impediments and also illustrates an insightful 
understanding of the elements of reasoning present in the source. The writer reveals 
an understanding of the stated and unstated concepts and/or assumptions that are an 
essential aspect of alternative perspectives and shows a sophisticated analysis of the 
information and conclusions used to support the source’s main argument. The writing 
style is energetic and precise; how the writer says things is as excellent as what the 
writer says. There is evidence of careful editing, since the paper contains few 
grammatical and/or mechanical errors. Research materials, if used, are correctly 
documented \dots.

\section{The B paper}

The B paper is above-average work that more than meets the requirements of the 
assignment. It has a clear sense of topic, audience, and purpose, with sound 
organization, good evidence, and good analysis and/or argument. As in the A paper, this 
writer has clearly used elements of reasoning to generate material for developing the 
paper, although not to the extent of the superior A paper. Similarly, all elements can be 
judged against the standards favorably. When the writer uses source materials, he/she 
demonstrates complete understanding of potential critical thinking impediments, but 
illustrates a less sophisticated understanding of the elements of reasoning present in 
the source. The writer reveals an understanding of the source’s assumptions and 
conclusions, but falls somewhat short in recognizing the stated and unstated concepts 
and/or assumptions of the argument. The writing style is clear and precise, and the 
paper shows evidence of careful editing, since the essay contains few grammatical 
and/or mechanical errors (although an otherwise superior paper can have errors that 
lower the overall grade). Research materials, if used, are correctly documented \dots.
\section{The C paper}
The C paper is adequate, average work that meets the requirements of the assignment. 
The paper has a clear sense of topic, audience, and purpose, with a generally sound 
organization, although problems with focus may exist. The evidence is adequate, but 
not as specific as in an A or B paper, and the analysis and/or argument is not as fully 
developed. This paper deals with the elements of reasoning as outlined above, although 
not to the extent of the A and/or B paper, and some elements may not be fully 
addressed. While the paper minimally meets the standards of critical thinking, it may 
not address some of these standards fully. When the writer uses source materials, 
he/she demonstrates a basic understanding of potential critical thinking impediments, 
but reveals only a rudimentary understanding of the elements of reasoning present in 
the source. The writing style is clear, although there may be problems with sentence 
construction or word choice. Though the writer has edited the paper, remaining errors 
may affect the ability to communicate. Research materials, if used, are correctly 
documented \dots.

\section{The D paper}

The D paper is below average work that demonstrates an attempt to fulfill the 
assignment and shows some promise, but does not meet the requirements of the 
assignment. The paper may have one or more of the following weaknesses. There may 
be problems with the sense of topic, audience, or purpose. The paper may not 
adequately address the elements of reasoning and the standards of critical thinking as 
outlined above. It may have a general or implied thesis, but the idea may be too broad, 
vague, or obvious. The organizational plan may be inappropriate or inconsistently carried 
out. Evidence may be too general, missing, irrelevant to the thesis, or inappropriately 
repetitive. The analysis and/or argument may be underdeveloped. The style may be 
compromised by repetitive or flawed sentence patterns and/or inappropriate word 
choice and confusing syntax. Grammatical and mechanical errors may interfere with 
readability and indicate a less-than-adequate attempt at editing or an unfamiliarity with 
some aspects of Standard Written English. Research materials, if used, may not be 
completely or accurately documented \dots.

\section{The F paper}

The F paper is unacceptable work that does not meet the requirements of the 
assignment. It exhibits one or more of the following weaknesses. It may not meet the 
purpose of the assignment. It may be an attempt to meet the requirements of the 
assignment, but have no apparent thesis or a self-contradictory one, or its point may 
be general or obvious and suggest no serious engagement with the topic. The paper 
fails to address the elements of reasoning and the standards of critical thinking as 
outlined above. It may display little or no apparent sense of organization; it may lack 
development; evidence may be inadequate, inappropriate, or may consist of 
generalizations, faulty assumptions, or errors of fact. The style suggests serious 
difficulties with fluency, which may be revealed in short, simple sentences and 
ineffective word choice. Grammatical/mechanical errors may interfere with reader 
comprehension or indicate problems with basic literacy or a lack of understanding of 
Standard English Usage. Research materials, if used, may not be handled responsibly 
and/or documented appropriately \dots.

%----------------------------------------------------------------------------------------
% END OF SECTION
%----------------------------------------------------------------------------------------
