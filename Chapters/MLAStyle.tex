
%----------------------------------------------------------------------------------------
%	MLA
%----------------------------------------------------------------------------------------
\chapter{MLA style} % Sub-section


\section{Formatting the MLA essay}
When setting up your word processor for an MLA-formatted document, use the 
following settings:

\begin{itemize}
\item Set 1" margins on all sides of the document.
\item Double-space the entire document, including block quotes.
\item In the top left portion of the first page, type your name, instructor's name, 
course title, and date on separate, double-spaced lines.
\item Include your last name and a page number on each page in the top right corner 
of the header.
\item Include a centered title on the first page.
\item Indent the first line of each paragraph with a tab set to 0.5".
\end{itemize}

When a quotation runs more than four typed lines, use a \textbf{block quote}. A block 
quote is a free-standing block of text, set apart from the rest of the text. (See the 
following page for a visual example of block quote formatting in MLA). When formatting 
blockquotes in MLA, use the following rules: 

\begin{itemize}
\item Begin the block quote on a new line. 
\item Indent every line of the quote 1" from the left margin (two tabs). 
\item Do not use quotation marks around the quoted material. 
\item Place the parenthetical citation \emph{after} the final punctuation of the quoted 
passage. For example:
\end{itemize}

The following page is a model for the MLA-formatted document:

%-----------MLA FIRST PAGE EXAMPLE--------
\newpage

\thispagestyle{empty}
\begin{flushright}Johnson 1\end{flushright}
\bigskip
\begin{Spacing}{2}
Jeff Johnson\\
Profesor Smith\\
English 130\\
January 28, 2010
\end{Spacing}
\begin{center}
America's Foreign Policy Future?
\end{center}
\begin{Spacing}{2}

\hspace{.4in}We currently live in an era of great political uncertainty. After the Cold War, as 
historian Dave Smith has argued, America suddenly became the stand-alone world power, 
dominating international treaties, coalitions, and movements (34). This "unipolarity" has resulted in one of the 
most wholly economically prosperous and peaceful periods of time in history. As one 
economist explains:

\hspace{.8in}The most remarkable result of the fall of the Soviet Union has been the

\hspace{.8in}incredible proliferation of capital markets to every corner of the globe.

\hspace{.8in}From Bangalore to Brazil, Columbia to China, capital flows have  

\hspace{.8in}circled the globe, all protected by the military forces of the world's last 

\hspace{.8in}remaining superpower. (Johnson 77)

However, this economic prosperity comes at a great cost. American military 
deployments at bases across the world cost the American taxpayers dearly. As much as


\end{Spacing}
\newpage
%------DONE------------------

\section{In-text citations}
The MLA style uses parenthetical citations to indicate the author and page number of 
sources. These parenthetical citations take two forms. The first form is used when the 
source you are citing \emph{is known or understood} by your audience. The second 
form is used when the author being cited is \emph{unknown or unclear}. 

In the following sentence, the author of the source in question is obvious:

\begin{quote}According to scholar James Frey, "Americans eat five pounds of ice cream 
annually" (78).
\end{quote}

Since the author is known to the reader, the citation uses \textbf{only the page 
number} of the source.

In these versions of the sentence, however, the author is not stated:

\begin{quote}
According to one scholar, "Americans eat five pounds of ice cream annually" (Frey 78).
\end{quote}
Or:

\begin{quote}
Studies have shown that the American people consume an average of five pounds of ice 
cream every year (Frey 78).
\end{quote}
In the first sentence, the author is referred to only as a generic "scholar." To give the 
reader information on which scholar is being cited, Frey's name is included in the 
citation. In the second sentence, the author describes the report, not its author; as a 
result, the student has included Frey's name to indicate whose report is being referenced. 

%----DONE-----------

\section{List of Works Cited}

MLA requires a bibliography at the conclusion of the essay that includes the full 
citation of the sources cited within the essay. In MLA, the bibliography is known as the 
Works Cited page. When setting up a Works Cited page, use the following rules and 
characteristics:

\begin{itemize}
\item Center the words "Works Cited" at the top of the page.
\item Use your last name and the page number on the right side of the page's header.
\item Double-space the Works Cited entries.
\item Alphabetize the entries by the author's last name.
\item If an entry runs more than one typed line, indent the second (and any 
subsequent) line with a .5" tab.
\item If two or more works by the same author are used, list the entries alphabetically 
by title. After the first entry, replace the author's name with three dashes followed by a 
period. (See the entries from St. Augustine on the following page).
\end{itemize}


Here is an example of a MLA formatted Works Cited page:

%-----------MLA Works Cited EXAMPLE--------
\newpage
\thispagestyle{empty}
\begin{flushright}Johnson 23\end{flushright}
\begin{center}
Works Cited
\end{center}
\begin{Spacing}{2}
Adams, Robert M. \emph{James Joyce and Ireland's Geography}. Woodward: 

\hspace{.4in}Classic Nonfiction Library, 1967. Print.

Augustine. \emph{Tractates on the Gospel of John}. Trans. J. W. Rettig. Washington D.C.:

\hspace{.4in}Catholic U of America P, 1988. Print.

- - -. \emph{Confessions}. Trans. Henry Chadwick. New York: Oxford UP, 1992. Print.

Carver, Craig. "Molly: Bloom's Preservative; Correspondence and Function in \emph{Ulysses}."

\hspace{.4in}\emph{James Joyce Quarterly} 12 (1975): 414-22. Print.

Garrett, Peter K. Introduction. \emph{Twentieth Century Interpretations of} Dubliners.

\hspace{.4in}Ed. Peter K. Garrett. New York: Penguin, 1968. 1-17. Print.

\end{Spacing}

\newpage




%------DONE------------------
										                

\section{The MLA Bibliography}

The following section provides examples for citing sources that are commonly found in
academic writing. We have separated them into sections on \textbf{books}, \textbf{periodicals}, \textbf{electronic
sources}, and \textbf{other types of sources} that are less common.


\section{Book forms}

\subsection{A book by one author}

\begin{quote}
\textbf{Author's Last Name, First Name. \emph{Title}. City: Publisher, Year. Medium.}

\medskip

Taylor, Herman. \emph{A Tale of One City}. New York: Little and Sons, 1998.
 
\hspace{.4in}Print.
\end{quote}
 
\subsection{Two or more works by the same author(s)}

If you cite two works by the same author, use the author's first and last name in the first instance. For any additional works by the same author, use three dashes followed by a period in place of the
author's name:

\begin{quote}
San, Rathanak. \emph{Escaping Vietnam}. Atlanta: Peach Pit Publishing, 1988. 

\hspace{.4in}Print.

\medskip

- - -. \emph{The Golden Triangle}. New York: Gray and Long, 1999. Print.
\end{quote}


\subsection{Two or three authors}

\begin{quote}
\textbf{First author's Last name, First name, Second author's First name and}

\hspace{.4in}\textbf{Last name, and Third author's First and Last name. \emph{Title}. City}

\hspace{.4in}\textbf{Publisher, Year. Medium}
\medskip

Roberts, John, Philip Glass and Jane Hinds. \emph{Recovering the City of Boston}. 

\hspace{.4in}Boston: U of Massachusetts P, 2000.
\end{quote}

\subsection{Four or more authors}
If a work has four or more authors, you may give the first author's name and then replace the other authors with the Latin term "et al," or "and others."

\begin{quote}

Bankston, Jonathan, et al. \emph{On Barns}. Bourne: Woodcraft Publishing,

\hspace{.4in}2013. Print

\end{quote}

\subsection{A book with an editor}
\begin{quote}
James, Henry. \emph{Portrait of a Lady}. Ed. Leon Edel. Boston: Houghton,

\hspace{.4in}1963. Print.
\end{quote}

\subsection{An edition (other than the first)}
\begin{quote}
Thompson, Fred. \emph{Why I Fight}. 3rd ed. New York: Vanity Publications,

\hspace{.4in}2000. Print.
\end{quote}

\subsection{A republished book}
\begin{quote}
\textbf{Author's last name, First name. \emph{Title}. Year originally published. City: }

\hspace{.4in}\textbf{Publisher, Year. Medium.}
\medskip

James, Esther. \emph{My Life}. 1946. New York: Random House, 2001. Print.
\end{quote}

\subsection{Corporate author (written by organization or government)}
\begin{quote}

\textbf{Organization name. \emph{Title}. City: Publisher, Year. Medium.}
\medskip

John Bigam Society. \emph{The Religions of Kenya}. New York: John Bigam 

\hspace{.4in}Publishing, 2000. Print.
\end{quote}

If the author is a government, include the name of the department or agency after the name of the government:
\begin{quote}

United States. Dept. of Transportation. \emph{State Highway Sinage}

\hspace{.4in} \emph{Regulations}. Washington: GPO, 2002. Web.

\end{quote}

\subsection{An anthology}
\begin{quote}

\textbf{Editor's Last name, First name, ed. \emph{Title}. City: Publisher, Year. Medium.}

\medskip

Foner, Eric, ed. \emph{An American Voice: A Collection of America's Finest}

\hspace{.4in}\emph{Essays}. Boston: McKinley and Smith, 2011. Print.

\end{quote}

\subsection{Work in an anthology}

\begin{quote}

\textbf{Author's Last name, First name. "Title." Ed. Editor's First and Last}

\hspace{.4in}\textbf{Name. \emph{Title}. City: Publisher, Year. Pages. Medium.}

\medskip

Graves, Thomas. "The History of our National Anthem." Ed. Eric Foner. 

\hspace{.4in}\emph{An American Voice: A Collection of America's Finest Essays.}

\hspace{.4in}Boston: McKinley and Smith, 2011. 20-41. Print.

\end{quote}

\subsection{No author or editor}

\begin{quote}
\emph{\textbf{Title}}. \textbf{City: Publisher, Year. Medium}

\medskip
\emph{A Guide to Boston}. Boston: Beantown Publishing, 2000. Print.
\end{quote}

\subsection{Forward, introduction, preface, or afterward}
\begin{quote}

\textbf{Author's last name, First name. Portion of book. \emph{Title}. By Author's} 

\hspace{.4in}\textbf{First and Last names. City: Publisher, Year. Pages. Medium.}

\medskip
Knox, John. Introduction. \emph{The Life of James}. By Elders Johnson.

\hspace{.4in}New York: Random House, 2009. 1-8. Print.

\end{quote}

\subsection{A book with a translator} 
\begin{quote}
McDougle, Astrid. \emph{The Basics of Gaelic}. Trans. Paddy Maloney. New

\hspace{.4in}York: Vintage, 1990. Print.
\end{quote}


\subsection{Multivolume work}
\begin{quote}

\textbf{Author's Last name, First name. \emph{Title}. Volume Number. City: }

\hspace{.4in}\textbf{Publisher, Year. Medium.}
\medskip

Graves, Johanna. \emph{Ronald Reagan and the Iran-Contra Affair}. Vol. 7. New 

\hspace{.4in}York: Greenstalk, 1988. Print.
\end{quote}

\subsection{Book in a series}

\begin{quote}

\textbf{Editor's Last name, First name, ed. \emph{Title}. By Author's Last name,}

\hspace{.4in}\textbf{First name. City: Publisher, Year. Medium. Abbreviated Series}

\hspace{.4in}{\textbf{Title}.}
\medskip

Smith, Rod, ed. \emph{American Economic Expansion in the Gilded Age}. By

\hspace{.4in} Andrew Stills. New York: Grim and Drang, 1988. Print.

\hspace{.4in}American Economics.
\end{quote}

\subsection{Article in a reference work, such as a dictionary or encyclopedia}
\begin{quote}
\textbf{Author's Last name, First name (if any). "Title." \emph{Title of Reference}}

\hspace{.4in}\textbf{\emph{Book}. Edition. Year. Medium.}

\medskip

"Suzerian." \emph{Merriam-Webster's Collegiate Dictionary}. 10th ed. 2008. Print.

\end{quote}

\subsection{Sacred text}

For sacred texts such as the Bible, Koran, or Torah:

\begin{quote}
\emph{\textbf{Title}}, \textbf{version(if any). City: Publisher, Year.}
\medskip

\emph{The Holy Bible, King James Version}. New York: American Bible Society, 

\hspace{.4in}1999.

\end{quote}

\subsection{Book with title within the title}
If a book title contains the title of another book, remove the italics to indicate the title
of the other work:

\begin{quote}
Hixson, Fred. \emph{On Cormac McCarthy's }Blood Meridian. New York: 

\hspace{.4in}Plainspeak Press, 2000.
\end{quote}


%--------------------------------------------------------------------------------------------
%Periodicals
%-------------------------------------------------------------------------------------------


\section{Periodical forms}

\subsection{Article in a scholarly journal with volume and issue numbers}
\begin{quote}
Taylor, James. "The Indian Matter of Charles Brockden Brown's

\hspace{.4in}\emph{Edgar Huntly}." \emph{American Literature} 45.6 (1998): 432-45. Print.
\end{quote}


\subsection{Article in a scholarly journal with only volume numbers}
\begin{quote}
Johnston, Johanna. "A Reading of \emph{Moby Dick}." \emph{North Dakota Quarterly}  

\hspace{.4in}45 (1978): 45-56. Print.
\end{quote}

\subsection{Article in a newspaper}
\begin{quote}
McKinley, Robert. "Cat Saved from Dog." \emph{New York Times} 7 Oct. 2011:

\hspace{.4in}A12+. Print.
\end{quote}


\subsection{Editorial in a newspaper}
\begin{quote}
"How to Reduce Crime." \emph{Sunapee Lake Times} 13 Nov. 2012: A4. Print.
\end{quote}

\subsection{Letter to the editor of a newspaper}

\begin{quote}
Johnson, Smitty. "Reduce our Property Taxes." \emph{Henniker Telegraph} 14 

\hspace{.4in}Oct. 2013: A1. Print. 
\end{quote}

\subsection{A review}
\begin{quote}
Smith, James. Rev. of \emph{The Orchard Revival}, by Cormac Freedman. 

\hspace{.4in}\emph{Oregon Magazine} 23 Oct. 2011: 34-36. Print. 
\end{quote}

\subsection{An unsigned article in a newspaper}

\begin{quote}
"A Walk Down Nostalgia Lane." \emph{Bloomington Sun} 28 

\hspace{.4in}Oct. 2013: B6. Print. 
\end{quote}


\subsection{Article in a magazine}
\begin{quote}
Smith, Jim. "Remembering Tony." \emph{New Yorker} Jan. 2010: 12-18. Print.
\end{quote}

%--------------------------------------------------------------------
\section{Online sources}

\subsection{Article in an online database}
Cite the source as you would a print article then include the database name, medium of access, and date of access:

\begin{quote}
Taylor, Hayden. "\emph{Moby Dick} and the Cold War." \emph{American Literature} 45.6 

\hspace{.4in}(2010): 45-57. JSTOR. Web. 12 July 2012.
\end{quote}

\subsection{An entire website} 

\begin{quote}
\textbf{Author's Last Name, First Name. \emph{Title of Website}. Publisher or}

\hspace{.4in}\textbf{Sponsor, Date published or last updated. Medium. Day} 

\hspace{.4in}\textbf{Month Year accessed.}

\medskip
Zimmerman, Constantine. \emph{The Moose Report}. New Hampshire Hiking

\hspace{.4in} Club, 2013. Web. 13 March 2013.
\end{quote}

\subsection{A work from a website}

\begin{quote}
\textbf{Author's Last Name, First Name. "Title of Work." \emph{Title of Site}. Ed.} 

\hspace{.4in}\textbf{Editor's First and Last Names. Publisher or Sponsor, Date} 

\hspace{.4in}\textbf{posted or last updated. Medium. Day Month Year accessed.}
\medskip

Reagan, John. "The Judo Champion Parent." \emph{Parenthood Online}. 

\hspace{.4in}Ed. Jessie McCarver. The Parent Institute of Boston. Oct. 2011. 

\hspace{.4in}Web. 5 Oct. 2012.

\end{quote}

\subsection{Online book or book chapter}
\begin{quote}
\textbf{Author's Last Name, First Name. \emph{Title}. City of Publication: Publisher,}

\hspace{.4in}\textbf{Year. Name of site or database. Medium. Day Month Year} 

\hspace{.4in}\textbf{accessed.}

\medskip
Melville, Herman. \emph{Moby Dick}. New York: RP Johnson, 1864. Bartleby.com. 

\hspace{.4in}Web. 7 Nov. 2000.
\end{quote}

\subsection{Article in an online scholarly journal}

\begin{quote}
\textbf{Author's Last name, First name. "Title." \emph{Title of Journal}. Volume.Issue} 

\hspace{.4in}\textbf{(Year): Pages. Medium. Day Month Year accessed.}

\medskip

Nelson, Grady. "Electronic Literature Comes of Age." \emph{e-Lit Quarterly}. 8.2 

\hspace{.4in}(2012): 2-12. Web. Oct. 28 2013.

\end{quote}

\subsection{Article in an online newspaper}

\begin{quote}
\textbf{Author's Last name, First name. "Title." \emph{Title of Newspaper}.}

\hspace{.4in}\textbf{Publisher, Day Month Year. Medium. Day Month Year}

\hspace{.4in}\textbf{accessed.}

\medskip

Taylor, Robert C. "Harvesting Undersea Sponges." \emph{New York Times}. New 

\hspace{.4in}York Times, 23 Nov. 2000. Web. 10 Dec. 2012.

\end{quote}

\subsection{Article in an online magazine}

\begin{quote}
\textbf{Author's Last name, First name. "Title of Article." \emph{Title of Magazine}. }

\hspace{.4in}\textbf{Publisher, Date. Medium. Day Month Year accessed.}

\medskip

James, Brian. "The New War on Terror." \emph{Foreign Affairs Monthly}. FAW 

\hspace{.4in}Publishing, 2 Oct. 2009. Web. 12 Nov. 2012.

\end{quote}


\subsection{Blog entry}
\begin{quote}
\textbf{Author's Last Name, First name. "Title." \emph{Title of Blog}. Sponsor, Day}

\hspace{.4in}\textbf{Month Year posted. Medium. Day Month Year accessed.}

\medskip

Hardeman, Chad. "The Recent Healthcare Debate." \emph{The Health Blog}. 

\hspace{.4in}4 July 2010. Web. 12 Dec. 2010.

\end{quote}

%Online editorial
%Online film review

\subsection{Email}
\begin{quote}
\textbf{Writer's Last name, First name. "Subject Line." Message to the author.} 

\hspace{.4in}\textbf{Day Month Year of email. Medium.}

\medskip
Cooledge, John. "My Election Thoughts." Message to the author. 

\hspace{.4in}12 Nov. 2012. Email.
\end{quote}

%Posting from an online discussion board

\subsection{Article from an online reference work, such as Wikipedia}
\begin{quote}

\textbf{"Title of Article." \emph{Title of Reference Work}. Sponsor, Date of work.} 

\hspace{.4in}\textbf{Medium. Day Month Year of access.}

\medskip

"Al-Qaeda." Wikipedia. Wikimedia Foundation, 3 April 2004. Web. 

\hspace{.4in}12 Oct. 2013.

\end{quote}

\subsection{Podcast:}

\begin{quote}
\textbf{Performer/Host's Last name, First name. "Title of Podcast."} 

\hspace{.4in}\textbf{Host's First and Last Name. \emph{Title of Podcast}. Sponsor,}

\hspace{.4in}\textbf{Day Month Year posted. Medium. Day Month Year accessed.}

\medskip

Zeender, Nathan and Michael Tonsmeire. "Dark Lagers." James Spenser. 

\hspace{.4in}\emph{Basic Brewing Radio}. 31 Jan. 2013. Web. 12 Mar. 2013.

\end{quote}


\section{Other types of sources}

\subsection{A dissertation}

\begin{quote}
\textbf{Author's Last name, First name. \emph{Title}. Diss. Institution Granting}

\hspace{.4in}\textbf{Degree, Year.}

\medskip

Redburn, Marcus. "A Study of Melville's Aesthetics." Diss. Boston 

\hspace{.4in}University, 1978.

\end{quote}


\subsection{An advertisement}

\begin{quote}
\textbf{Product or Company. Advertisement. \emph{Title of Publication} Date or}

\hspace{.4in}\textbf{Volume.Issue (Year): Page(s). Medium.}

\medskip

Dove Body Wash. Advertisement. \emph{Fortune Monthly} Oct. (2012): 23. 

\hspace{.4in}Print.


\end{quote}



\subsection{Artwork}

\begin{quote} 

\textbf{Artist's Last name, First name. \emph{Title}. Medium. Year. Institution, City.}

\medskip

Freeman, Dianna. \emph{Still Life 7}. Watercolor. 2009. Hunter Museum of Art, 

\hspace{.4in}Chattanooga.

\end{quote}

\subsection{Film or video clip}

\begin{quote}

\textbf{\emph{Title}. Dir. Director's First and Last names. Perf. Lead Actors'} 

\hspace{.4in}\textbf{First and Last names. Distributor, Year of release. Medium. }

\medskip

\emph{Rushmore}. Dir. Wes Anderson. Perf. Bill Murray, Jason Schwartzman, 

\hspace{.4in}and Olivia Williams. Buena Vista International, 1998. DVD.

\end{quote}


%Broadcast interview
%Published interview
%Unpublished letter
%Published letter
%Map or chart
%Musical score
%Sound recording
%Oral presentation
%Paper from a conference
%Performance
%Television or radio program
%Pamphlet, brochure, or press release
%Legal source
%A digital file, such as .mp3, .pdf, etc.


%----------------------------------------------------------------------------------------
% END OF SECTION
%----------------------------------------------------------------------------------------