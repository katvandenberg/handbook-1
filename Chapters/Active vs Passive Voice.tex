%----------------------------------------------------------------------------------------
% Active vs. Passive Voice
%----------------------------------------------------------------------------------------

\chapter{Active vs. Passive Voice}



In the active voice, the subject of the sentence is performing the action expressed in the verb; in the passive voice, the subject of the sentence is acted upon. 

While the passive voice is often a requirement in scientific or technical writing---where the focus is on a particular process or experiment---it should be avoided in most other forms of academic writing. As the following sentences illustrate, the active voice produces less wordy and more precise sentences:

\begin{quote}
 
\textbf{Passive voice}: My first visit to San Francisco will always be remembered by me.

\textbf{Active voice}: I will always remember my first visit to San Francisco.

\textbf{Passive voice}: 	An A was given to Johnnie by professor James.

\textbf{Active voice}: 	Professor James gave Johnnie an A.
\end{quote}

\begin{quote}
\textbf{Passive voice}: The take-home exam was cheated on by over 100 Harvard students.

\textbf{Active voice}: Over 100 Harvard students cheated on the take-home exam.

\textbf{Passive voice}: Mistakes were made.

\textbf{Active voice}: We made mistakes.

\end{quote}

%----------------------------------------------------------------------------------------
% END OF SECTION
%----------------------------------------------------------------------------------------