
%----------------------------------------------------------------------------------------
%Proofreading for Common Errors
%----------------------------------------------------------------------------------------
 
\section{\textcolor{ForestGreen}{Proofreading for Errors}}

\begin{enumerate}

\item \textbf{Comma splice}

A comma splice occurs when the writer combines two or more independent clauses (i.e., clauses that could stand alone as sentences) with only a comma. To revise, use a comma with a coordinating conjunction (and, or, but, for, so, yet); employ a semicolon; or break the clauses up and punctuate them as separate sentences.
\begin{quote}
\textbf{[Wrong]}: The Cherokee Indians once ruled the Tennessee valley, they are all gone now.

\textbf{[Revised]}:  The Cherokee Indians once ruled the Tennessee valley, but they are all gone now.

\textbf{[Revised]}: The Cherokee Indians once ruled the Tennessee valley; they are all gone now.
\end{quote}
\item \textbf{Run-on}

A run-on is very similar to a comma splice; it is also created when the writer combines two or more independent clauses incorrectly, in this case by not providing any punctuation between them.  To revise, do as you would with a comma splice and use a comma with a coordinating conjunction; employ a semi-colon; or break the clauses up and punctuate them as separate sentences.
\begin{quote}
\textbf{[Wrong]}: The medication had a side effect it caused severe dry mouth.

\textbf{[Revised]}: The medication had a side effect; it caused severe dry mouth.

\textbf{[Revised]}: The medication had a side effect. It caused severe dry mouth.
\end{quote}
\item \textbf{Sentence Fragment}

A sentence fragment is a dependent clause treated as an independent clause or sentence. It does not express a complete thought. Often, though not always, these start with the following words: Although, While, Because, Whether, Such, and If.
\begin{quote}
\textbf{[Wrong]}: BU offers many classes. Such as Accounting and English.

\textbf{[Revised]}: BU offers many classes, such as Accounting and English.
\end{quote}
\item \textbf{Misplaced or dangling modifier}

A misplaced or dangling modifier occurs when a word or phrased is not placed next to the word it modifies, thus altering the meaning of a sentence, often in comical ways.
\begin{quote}
\textbf{[Wrong]}: Covered with hot, melted cheese, we ate the pizza.

\textbf{[Revised]}: We ate the pizza, which was covered in hot, melted cheese.
\end{quote}

\item \textbf{Mixed construction}

Mixed construction occurs when you start with one sentence structure and then shift to another.
\begin{quote}
\textbf{[Wrong]}: There is so much going on in the world today is why it is so hard to keep up with everything.

\textbf{[Revised]}: With so much going on in the world today, it is hard to keep up with everything.
\end{quote}

\item \textbf{Wrong preposition}

Many prepositions have developed a particular usage that will be expected by readers.

\begin{quote}
\textbf{[Wrong]}: The candidate compared his opponent with an orangutan.

\textbf{[Revised]}: The candidate compared his opponent to an orangutan.

\textbf{[Wrong]}: The tree was at the background of the photograph.

\textbf{[Revised]}: The tree was in the background of the photograph.

\end{quote}

\item \textbf{Shifts in tense, person, number, and mood}

Avoid unexpected changes in tense, person, number, and mood.

\emph{Tense}:
\begin{quote}
\textbf{[Wrong]}: She ran to the store and picks up some milk. 

\textbf{[Revised]}:  She ran to the store and picked up some milk.
\end{quote}

\emph{Person}:
\begin{quote}
\textbf{[Wrong]}:  When one visits Boston's Museum of Fine Arts, you get an education.

\textbf{[Revised]}:  When one visits Boston's Museum of Fine Arts, one gets an education.
\end{quote}

\emph{Number} (shifting between singular and plural):
\begin{quote}
\textbf{[Wrong]}: Because people make errors, he or she receives poor grades.

\textbf{[Revised]}:  Because people make errors, they receive poor grades.
\end{quote}

\emph{Mood} (indicative=make a statement/pose a question; imperative=request/command; subjunctive=making a wish):
\begin{quote}
\textbf{[Wrong]}: The teacher gave the students two guidelines: do not chew gum in class (imperative) and students should not be late for class (indicative).

\textbf{[Revised]}: The teacher gave the students two guidelines: do not chew gum in class and be on time.
\end{quote}

\item \textbf{Missing Commas}

\emph{After an introductory element}:
 \begin{quote}
\textbf{[Wrong]}: To tell the truth I never really liked the Yankees.

\textbf{[Revised]}: To tell the truth, I never really liked the Yankees.
\end{quote}

\emph{With a nonrestrictive element} (a part of the sentence not essential to the meaning of the sentence):

\begin{quote}
\textbf{[Wrong]}: Jeff who owned the corporation was a big gambler.

\textbf{[Revised]}: Jeff, who owned the corporation, was a big gambler.
\end{quote}

\emph{
In a series}:
\begin{quote}
\textbf{[Wrong]}: He bought eggs, milk, cheese and shampoo.

\textbf{[Revised]}: He bought eggs, milk, cheese, and shampoo. 
\end{quote}


\item \textbf{Unclear pronoun reference}
\begin{quote}
\textbf{[Wrong]}: The teacher gave her notes to her. (Whose notes are they?)

\textbf{[Revised]}: The teacher gave Jane's notes to her.
\end{quote}
\item \textbf{Misspellings [don't count on spell check]/Wrong words}
\begin{quote}
\textbf{[Wrong]}: He drank the bear.

\textbf{[Revised]}: He drank the beer.
\end{quote}
\end{enumerate}
