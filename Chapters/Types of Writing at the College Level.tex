
%----------------------------------------------------------------------------------------
%	TYPES OF WRITING AT THE COLLEGE LEVEL
%----------------------------------------------------------------------------------------
\chapter{Types of Writing at the College Level}

Writing in college takes many forms. Your writing instrutor may ask you to complete 
assignments that focus on particular "modes" of writing such as narrative, exposition, 
synthesis, or analysis. However, there are seldom clear boundaries between these 
rhetorical forms. For instance, a research paper often includes synthesis and argument; 
an analysis can be an argument; a proposal is an argument often based in research; and 
arguments often include narrative features. The following are the most common forms 
of writing that you will encounter in your college coursework:


\section{Expository}
A paper in which you report on, define, summarize, clarify and/or explain a concept or 
a process. The purpose of this paper is to provide information to an audience unfamiliar 
with the subject or to demonstrate for a professor that you have understood course 
material.  Research is often required. Clarity and organization are key.

\section{Synthesis}

A paper in which you pull together information from different sources on one topic or 
in which you examine the connections between the perspectives of various authors on 
one issue. It is essential that you move beyond summarizing each source and/or author 
discreetly, and clarify the relationships between ideas (whether from different sources 
or different authors). Thus, you must focus on where authors agree and disagree, or 
where ideas conflict or complement each other. Transitional words and phrases are 
extremely important for synthesis papers.

\section{Analysis/Close reading}
Analysis requires you to look at a work (a reading, film, musical piece, etc.), a process, 
a person, or an issue and break it into smaller units in order to explain how those units 
function independently and how they work together to create the whole. You might, 
for example, do a rhetorical analysis to analyze how successfully an author persuades his 
audience, a visual analysis to explain how a film creates a certain mood or communicates 
a theme, or a poem to examine how figures of speech work to convey meaning at the 
level of the sentence.

\section{Argument}

A paper that requires you to make a claim about a debatable issue, which you then 
back up with evidence (examples, data, quotes from experts, etc.). An argument also 
usually requires you to recognize (and refute, concede, or undermine) opposing 
arguments and to qualify your claim.

\section{Response}

A paper written in response to a specific question or prompt. It may involve any of the 
other modes of writing. You should be sure you have a strong sense of the professor's 
expectations as well as a good grounding in the subject matter (which often comes 
from lectures, course readings, and discussions).

\section{Proposal}

An argument in which you propose that something should be done in the future. This 
often requires you to anticipate possible constraints and obstacles, qualify your claims, 
and pay close attention to the needs of the audience.

\section{Research paper}

Any paper that requires you to draw on sources other than your own thinking. Be sure 
you are familiar with the citation format required by each discipline/professor/class 
(i.e. APA, MLA, and Chicago Style), and that you understand what plagiarism is and how 
it can be avoided.

\section{Narrative}

A paper that tells a story, whether fictional, nonfictional, or a blend. Narratives can be 
used to entertain, inform, or persuade (or some blend).


%----------------------------------------------------------------------------------------
% END OF SECTION
%----------------------------------------------------------------------------------------
