%----------------------------------------------------------------------------------------
%Thesis Statements
%----------------------------------------------------------------------------------------

\chapter{Thesis Statements}

\section{Why do I need a thesis statement?}

A thesis statement is your way of introducing your readers to the main ideas of your 
paper while indicating to them the order in which you will address these ideas. It serves 
as a miniature outline for your paper.

\section{Where does it belong?}

Readers are typically expecting to see the thesis statement at the end of the 
introduction. For papers under 10-12 pages, this usually means it will be on the first 
page of your paper.

\section{When should I write it?}

Unless your professor has required you to submit a thesis before a draft is due, you 
may write it at any point during the drafting process. Many students do not know exactly 
what they want their paper to cover until they have completed a first draft. Thus, do 
not feel obligated to have a thesis composed before you begin drafting.

\section{How long can it be?}

Many high school students are taught to keep thesis statements to one sentence, 
listing 3 ideas for 3 main body paragraphs (i.e. "The death penalty should be eliminated 
because it is expensive, ineffective, and immoral.") However, most college papers are 
much longer and more complex and thus require a more lengthy thesis statement.

As a general guideline, 3-4 sentences should be plenty. You don't want to overwhelm 
the reader or go into too much detail before it is necessary.

\section{What should it do?}

\begin{itemize}
\item List the main ideas of your paper as precisely as possible.

\item List those points in the order in which they appear in the paper.

\end{itemize}

\section{What makes a thesis statement strong?}

\begin{itemize}
\item Being precise (see \emph{Precision}) and concise (see \emph{Concision}).

\item Using parallel structure (see \emph{Parallel structure}.

\item Employing transitions.
\end{itemize}

\section{What should I avoid?}

\begin{itemize}
\item Avoid being vague about your actual points or narrating what you will do in the 
paper (i.e., "In this paper I will make some points about how the film works and the 
many effects of the film style").

\item Don't raise points that you do not develop in your paper.

\end{itemize}

\textbf{For example}:

\textbf{[A weak thesis]}:  “All of these authors discuss many points. I will talk about 
which points I agree with. There are also many points with which I disagree. These will 
also be talked about in this paper.”


\textbf{[Revised]}: While Pauline Erera believes a family can and should take many forms, 
Wade Horn and James Wilson disagree, arguing for the importance of the nuclear family 
to the proper upbringing of children. Although Erera's willingness to embrace diversity is 
admirable, Horn and Wade are correct in asserting that the traditional two-parent home 
ensures safe, happy, and educated children in a way nontraditional families cannot.

%----------------------------------------------------------------------------------------
% END OF SECTION
%----------------------------------------------------------------------------------------

