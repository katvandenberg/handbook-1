%----------------------------------------------------------------------------------------
%	INTRODUCTION
%----------------------------------------------------------------------------------------

\section{\textcolor{ForestGreen}{Rhetoric And Composition}}


\subsection{What is rhetoric?}

\textbf{At a glance}: 

Rhetoric is the study of oral, written, and visual persuasion. Oral speeches were an important part of social, political, and judicial life in ancient Greece, which gave birth to democratic government. Aristotle's work, \emph{The Art of Rhetoric}, is one of the earliest known books on the subject.

 
For the next 2,300 years, classical rhetoric was an integral part of Western culture: most educated citizens were required to take classes in it.
 
\textbf{Why do we need it in this class?} 

Rhetoric helps us determine how to use reason, emotion, and the strength of our own character to persuade others to change their views. It allows us to think critically about how others attempt to persuade us. It further provides methods for generating, arranging, and styling content. It directs our attention, as writers, to the importance of rhetorical situations, audience awareness, and voice.

\textbf{Where can I find more online information about rhetoric?}

Try the website \emph{Silva Rhetoricae}, or ``The Forest of Rhetoric,'' which explains the rhetorical proofs of ethos, pathos, and logos and lists and defines figures of speech.

\url{http://rhetoric.byu.edu/} 
%-------done-------
