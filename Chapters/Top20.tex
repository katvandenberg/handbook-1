 %----------------------------------------------------------------------------------------
% Top 20 Grammatical Errors
%----------------------------------------------------------------------------------------

 
 
\chapter{Top 20 Grammatical Errors}

In 1993, a statistical study of student writing performed by scholars Andrea Lunsford and Robert Connors demonstrated that nearly 100 percent of student writing mistakes are limited to 20 discrete errors. Eliminating these errors in your writing therefore offers the quickest path to precise, error-free prose.

\begin{enumerate}

\item \textbf{1 Wrong word}.

\begin{quote}
The workmen assembled at the \sout{cite}. \ding{55}     

The workmen assembled at the site. \ding{51}     

\medskip

I have a bad case of \sout{ammonia}. \ding{55}    

I have a bad case of pneumonia. \ding{51}    
\end{quote}

\item \textbf{2 Missing comma after an introductory element}.

Introductory words or clauses are usually set off with a comma. 

\begin{quote}

To tell the truth I never really liked the Yankees. \ding{55} 

To tell the truth, I never really liked the Yankees. \ding{51} 

\medskip

However I will say that he had remarkable hair. \ding{55} 

However, I will say that he had remarkable hair. \ding{51} 
\end{quote}

\item \textbf{3 Incomplete or missing documentation}

Missing documentation for quotations, summaries, or paraphrases of other texts may result in charges
of \textbf{plagiarism}, a serious academic offense.

\begin{quote}
Taylor argues that the response to terrorism should not be a curtailing of freedoms. \ding{55}

\medskip

Taylor argues that the response to terrorism should not be a curtailing of freedoms (29). \ding{51}
\end{quote}

\item \textbf{4 Vague pronoun reference}.

\begin{quote}
The teacher gave her notes to her. \ding{55} 

The teacher gave her notes to Jane. \ding{51} 

\end{quote}

\item \textbf{5 Spelling error}.

\begin{quote}
I frequently \sout{loose} my keys. \ding{55}

I frequently lose my keys. \ding{51}

\end{quote}

\item \textbf{6 Mechanical error with a quotation}

\begin{quote}
"We do not choose our faith", (89) says Wendell Berry. \ding{55}

"We do not choose our faith," (89) says Wendell Berry. \ding{51}
\end{quote}

\ding{96} The comma should be placed \emph{inside} the quotation marks.

\medskip

\begin{quote}
Roberts describes the candidate as "troubling in his memoir (7). \ding{55}

Roberts describes the candidate as "troubling" in his memoir (7). \ding{51}

\end{quote}

\ding{96} In the first sentence, the writer has neglected to close the quotation.

\item \textbf{7 Unnecessary comma}.

\begin{quote}
The legal language applies to carnivals, and to amusement parks.  \ding{55} 

The legal language applies to carnivals and to amusement parks.  \ding{51} 
\end{quote}

\ding{96} A comma is not necessary here since both phrases modify the verb "applies."

\medskip

\begin{quote}
Dinosaurs, of the Cretaceous era, were likely covered in feathers.  \ding{55} 

Dinosaurs of the Cretaceous era were likely covered in feathers.  \ding{51} 

\end{quote}

\ding{96} In this sentence, the phrase "of the Cretaceous era" is a \textbf{restrictive element.} A restrictive element
is a part of a sentence that is necessary for the meaning of the sentence. Were we to remove this phrase, the meaning
of the sentence would radically change. Restrictive elements are never set off with commas. 

\item \textbf{11 Missing comma with a nonrestrictive element}.

A nonrestrictive element is a part of a sentence that is not essential to its meaning. Commas are used to set off these nonessential portions of the sentence.

\begin{quote}
Jeff who owned the corporation was a big gambler. \ding{55}     

Jeff, who owned the corporation, was a big gambler. \ding{51}     
\end{quote}





\item \textbf{13. Missing comma in compound sentence}.

A compound sentence contains two or more clauses that can stand alone as complete sentences (otherwise known as "independent clauses"). However, to connect them you must either employ a semicolon \emph{or} use a comma and coordinating conjunction such as \emph{and}, \emph{but}, or \emph{yet}. Failing to punctuate the compound sentence results in a fused, or run-on, sentence.

\begin{quote}
I've given her all that I own and I can't see myself giving more. \ding{55} 

I've given her all that I own; I can't see myself giving more.  \ding{51}     

I've given her all that I own, and I can't see myself giving more. \ding{51}     

\end{quote}



\item \textbf{Wrong or missing verb ending}.

\begin{quote}
The Mars lander \sout{land} safely on the surface of the planet. \ding{55}     

The Mars lander landed safely on the surface of the planet. \ding{51}     
\end{quote}

\item \textbf{Wrong or missing preposition}.

Many prepositions have developed a particular usage that will be expected by readers. For example, one would never say "Let's meet in Harvard street." Rather, it is customary to say "Let's meet on Harvard street."
 
 \begin{quote}
The candidate compared his opponent \sout{with} an orangutan. \ding{55}     

The candidate compared his opponent to an orangutan. \ding{51}     
\end{quote}

\item \textbf{Comma splice}.

A comma splice occurs when two independent clauses are joined by a comma. To revise, use a comma with a coordinating conjunction or a semicolon.

\begin{quote}
Indians once ruled the valley they are all gone now. \ding{55}     

Indians once ruled the valley, but they are all gone now. \ding{51}     

Indians once ruled the valley; they are all gone now. \ding{51}     
 
\end{quote}

\item \textbf{Missing or misplaced possessive apostrophe}. 

\begin{quote}
The farm stand is proud to offer it's vegetables for sale now. \ding{55}     

The farm stand is proud to offer its vegetables for sale now. \ding{51}     
\end{quote}

\begin{quote}

The Indian's best player is Ubaldo Jimenez. \ding{55}     

The Indians' best player is Ubaldo Jimenez. \ding{51}     
	
\end{quote}


\item \textbf{Unnecessary shift in tense}.

\begin{quote}
She ran to the store and picks up some milk. \ding{55}     

She ran to the store and picked up some milk. \ding{51}     
\end{quote}

\item \textbf{Unnecessary shift in pronoun}.

\begin{quote}
When one visits the Museum, you get an education. \ding{55}     

When one visits the Museum, one gets an education. \ding{51}     

\end{quote}

\item  \textbf{Sentence fragment}.

A fragment is an incomplete thought. It is a dependent clause treated as an independent clause. Often, these start with the following words: \emph{Although}, \emph{While}, \emph{Because}, \emph{Whether}, \emph{Such}, and \emph{If}.

\begin{quote}
BU offers many classes. Such as Accounting and English. \ding{55}     

BU offers many classes, such as Accounting and English. \ding{51}     
\end{quote}

\item \textbf{Wrong tense or verb form}.

\begin{quote}
By the time the UPS man arrived, I left. \ding{55}     

By the time the UPS man arrived, I had left. \ding{51}     
\end{quote}
\item\textbf{ Lack of subject-verb agreement}.

Verbs must agree with their subject in number and person. 

\begin{quote}
The running backs listens for an audible. \ding{55}     

The running backs listen to the audible. \ding{51}     

Jeff was one of the firemen who was killed. \ding{55}     

Jeff was one of the firemen who were killed. \ding{51}     
\end{quote}

\item \textbf{Missing comma in a series}.
\begin{quote}
He bought eggs milk cheese and shampoo. \ding{55}     

He bought eggs, milk, cheese, and shampoo. \ding{51}     
\end{quote}

\item \textbf{Lack of agreement between pronoun and antecedent}.

Indefinite pronouns (such as \emph{everyone} and \emph{each}) are always treated as singular (example A). If antecedents are joined by \emph{or} or \emph{nor}, the pronoun must agree with the closer antecedent (example B). Collective nouns can be either singular or plural depending on whether the people are seen as a single unit or a group of individuals (example C).


A) [\textbf{Wrong}]: Each of the prisoners found happiness in their new work program. 

[\textbf{Revised}]: Each of the prisoners found happiness in his new work program. 


B) [\textbf{Wrong}]: Either Jeff or Robert will be required to give up their car. 

[\textbf{Revised}]:Either Jeff or Robert will be required to give up his car. 
\medskip

C) [\textbf{Singular}]: The campaign constantly changed its positions in the weeks before the election. 

[\textbf{Plural}]: The campaign constantly changed their positions in the weeks before the election. 


\item \textbf{Unnecessary commas with restrictive element}.

A restrictive element is one that is \emph{essential} to the meaning of the sentence. No commas are used with them. 

\begin{quote}
[\textbf{Wrong}]: The voters, who wanted the dam construction blocked, showed up in droves.

 [\textbf{Revised}]: The voters who wanted the dam construction blocked showed up in droves.
\end{quote}

\item \textbf{Fused sentence}.

A fused sentence is also known as a ``run on'' sentence. It occurs when two clauses that could stand alone as complete sentences are placed together without punctuation.
 
 \begin{quote}
[\textbf{Wrong}]: The medication had a side effect it caused severe dry mouth. 

[\textbf{Revised}]: The medication had a side effect; it caused severe dry mouth. 
\end{quote}

\item \textbf{Misplaced or dangling modifier}.

	[\textbf{Wrong}]: Covered with hot melted cheese, we ate the pizza. 
	
	[\textbf{Revised}]: We ate the pizza that was covered with hot, melted cheese. 
	
\item \textbf{It's / Its confusion}.

It's is a contraction and means "it is" or "it has." \emph{Its} is the possessive form of it.

\begin{quote}
[\textbf{Wrong}]: Its unfair to make him pay for all the damages. 

[\textbf{Revised}]: It's unfair to make him pay for all the damages. 
\end{quote}

\end{enumerate}
