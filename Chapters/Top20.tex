%----------------------------------
% Common Sentence Errors
%-----------------------------------



\chapter{Common Sentence Errors}

A \href{http://www.jstor.org/discover/10.2307/357695?uid=3739800&uid=2129&uid=2&uid=70&uid=4&uid=3739256&sid=21102412978541}{\{statistical study of student writing\}} performed in 1988 by scholars Andrea
Lunsford and Robert Connors demonstrated that virtually all student
writing mistakes are limited to 20 formal errors. Eliminating these 
errors in your writing therefore offers the quickest path to error-free prose.

\begin{enumerate}

\item \textbf{Wrong word}

\begin{quote}
The workmen assembled at the \sout{cite}. \ding{55}

The workmen assembled at the site. \ding{51}

\medskip

I have a bad case of \sout{ammonia}. \ding{55}

I have a bad case of pneumonia. \ding{51}
\end{quote}

\item \textbf{Missing comma after an introductory element}

Introductory words or clauses are usually set off with a comma:

\begin{quote}

To tell the truth I never really cared for him. \ding{55}

To tell the truth, I never really cared for him. \ding{51}

\medskip

However I will say that he had remarkable hair. \ding{55}

However, I will say that he had remarkable hair. \ding{51}
\end{quote}

\item \textbf{Incomplete or missing documentation}

Missing documentation for a quotation, summary, or paraphrase of another text
may result in charges of \textbf{plagiarism}, a serious academic offense.

\begin{quote}
Taylor argues that the response to terrorism should not be a curtailing of
freedoms. \ding{55}

\medskip

Taylor argues that the response to terrorism should not be a curtailing of
freedoms (29). \ding{51}

\end{quote}

\item \textbf{Vague pronoun reference}

\begin{quote}
The teacher gave her notes to her. \ding{55}

The teacher gave her notes to Jane. \ding{51}

\end{quote}

\item \textbf{Spelling error}.

\begin{quote}
I frequently \sout{loose} my keys. \ding{55}

I frequently lose my keys. \ding{51}

\end{quote}

\item \textbf{Missing Word}

\begin{quote}

We drove to lake in the mountains. \ding{55}

We drove to \textbf{the} lake in the mountains. \ding{51}
\end{quote}

\item \textbf{Mechanical error with a quotation}

\begin{quote}
"We do not choose our faith\hl{",} (89) says Wendell Berry. \ding{55}

"We do not choose our faith," (89) says Wendell Berry. \ding{51}
\end{quote}

\ding{96} The comma should be placed \emph{inside} the quotation marks.

\begin{quote}
Roberts describes the candidate as "troubling in his memoir (7). \ding{55}

Roberts describes the candidate as "troubling" in his memoir (7). \ding{51}

\end{quote}

\ding{96} In the first sentence, the writer has neglected to close the
quotation.

\item \textbf{Unnecessary comma}


\begin{quote}
The legal language applies to carnivals, and to amusement parks.  \ding{55}

The legal language applies to carnivals and to amusement parks.  \ding{51}

\medskip

The cemetery on the hill, is haunted. \ding{55}

The cemetery on the hill is haunted. \ding{51}

\end{quote}

\ding{96} A comma is not necessary here since both phrases 
modify the verb "applies." 

\begin{quote}
Dinosaurs, of the Cretaceous era, were likely covered in feathers.  \ding{55}

Dinosaurs of the Cretaceous era were likely covered in feathers.  \ding{51}

\end{quote}

\ding{96} The phrase "of the Cretaceous era" is a \textbf{restrictive element.} 
A restrictive element is a part of a sentence that is essential to the meaning 
of the sentence. Were we to remove this phrase the meaning of the sentence would 
radically change. Restrictive elements are never set off with commas.

\item \textbf{Missing comma with a nonrestrictive element}

A \textbf{nonrestrictive element} is a part of a sentence that is not essential to its
meaning. Commas are used to set off these nonessential portions of the
sentence.

\begin{quote}
Jeff who owned the corporation was a big gambler. \ding{55}

Jeff, who owned the corporation, was a big gambler. \ding{51}
\end{quote}

\item \textbf{Missing comma in compound sentence}

A compound sentence contains two or more clauses that can stand alone as
complete sentences (otherwise known as "independent clauses"). However,
to connect them you must either use a semicolon or use a comma and
coordinating conjunction such as \emph{and}, \emph{but}, or \emph{yet}. Failing
to punctuate the compound sentence results in a fused, or run-on, sentence.

\begin{quote}
I've given him all that I own and I can't see myself giving more. \ding{55}

I've given him all that I own; I can't see myself giving more.  \ding{51}

I've given him all that I own, and I can't see myself giving more. \ding{51}

\end{quote}

\item \textbf{Faulty sentence structure}

When a sentence begin with a certain structure, then abruptly shifts to a different
one, it becomes disorderly and difficult to follow. 

\begin{quote}
There is so much going on in the world today is why it is so hard to 
keep up with everything.

With so much going on in the world, it can be difficulty to keep up.

\end{quote}


\item \textbf{Comma splice}

A comma splice occurs when two independent clauses are joined by a comma. To
revise, use a comma with a coordinating conjunction or a semicolon.

\begin{quote}
Indians once ruled the valley they are all gone now. \ding{55}

Indians once ruled the valley, but they are all gone now. \ding{51}

Indians once ruled the valley; they are all gone now. \ding{51}

\end{quote}

\item \textbf{Missing or misplaced possessive apostrophe}

\begin{quote}
The farm stand is proud to offer it's vegetables for sale now. \ding{55}

The farm stand is proud to offer its vegetables for sale now. \ding{51}
\end{quote}

\begin{quote}

The Indian's best player is Ubaldo Jimenez. \ding{55}

The Indians' best player is Ubaldo Jimenez. \ding{51}

\end{quote}


\item \textbf{Unnecessary shift in tense}.

\begin{quote}
She ran to the store and picks up some milk. \ding{55}

She ran to the store and picked up some milk. \ding{51}
\end{quote}


\item \textbf{Sentence fragment}

A fragment is an incomplete thought. It is a dependent clause treated
as an independent clause. Often, these start with the following words:
\emph{Although}, \emph{While}, \emph{Because}, \emph{Whether}, \emph{Such},
and \emph{If}.

\begin{quote}
BU offers many classes. Such as Accounting and English. \ding{55}

BU offers many classes, such as Accounting and English. \ding{51}
\end{quote}

\item\textbf{Lack of subject-verb agreement}.

Verbs must agree with their subject in number and person.

\begin{quote}
The running backs listens for an audible. \ding{55}

The running backs listen to the audible. \ding{51}

\medskip

Jeff was one of the firemen who was killed. \ding{55}

Jeff was one of the firemen who were killed. \ding{51}
\end{quote}

\item \textbf{Lack of agreement between pronoun and antecedent}

Pronouns and their antecedents must always agree in number. Three 
rules govern the choice between singular or plural pronouns. 1) Sentences that begin
with an indefinite pronoun (such as \emph{everyone} and \emph{each}) are \emph{always}
treated as singular. 2) If antecedents are joined by \emph{or} or \emph{nor}, 
the pronoun must agree with the closer antecedent. 3) Collective nouns can be either 
singular or plural depending on whether the people are seen as a single unit or a 
group of individuals.

\begin{quote}
Each of the prisoners found happiness in their work. \ding{55}

Each of the prisoners found happiness in his work. \ding{51}
\medskip

Either Jeff or Robert will be required to give up their car. \ding{55}

Either Jeff or Robert will be required to give up his car. \ding{51}
\medskip

The campaign constantly changed its positions in the weeks before the election. \ding{51}

The campaign constantly changed their positions in the weeks before the election. \ding{51}
\end{quote}

\item \textbf{Fused (run-on) sentence}

A fused sentence is also known as a "run-on" sentence. It occurs when two
clauses that could stand alone as complete sentences are placed together
without punctuation.

 \begin{quote}
The medication had a side effect it caused severe dry mouth. \ding{55}

The medication had a side effect; it caused severe dry mouth. \ding{51}
\end{quote}

\item \textbf{It's / Its confusion}

\textbf{It's} is a contraction and means "it is" or "it has." \textbf{Its} is the
possessive form of it.

\begin{quote}
Its unfair to make him pay for all the damages. \ding{55}

It's unfair to make him pay for all the damages. \ding{51}

\end{quote}

\item \textbf{Poorly integrated quotation}

When you integrate borrowed material into your own writing, the 
"hybrid" sentence you create must satisfy grammar. The quoted material should
blend seamlessly into your sentence structure.

\begin{quote}
Scholar Rod Andrews argues "I argue that there can be no 
reasonable discussions of Shakespeare's biography" (99). \ding{55}

Scholar Rod Andres argues that "there can be no reasonable discussions 
of Shakespeare's biography" (99). \ding{51}
\end{quote}


\end{enumerate}

